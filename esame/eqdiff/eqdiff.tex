\documentclass[11pt,twoside,a4paper]{article}
\usepackage{amsmath}
\usepackage{amsfonts}
\usepackage[T1]{fontenc}
\usepackage[utf8]{inputenc}
\usepackage[top=20mm, head=6mm, headsep=6mm, foot=6mm, bottom= 15mm, left=15mm, right=15mm]{geometry}
\usepackage{calc} 
\usepackage{pythontex}
\usepackage{dsfont}
\newcommand{\mylabel}[1]{#1\hfill}
\renewenvironment{itemize}
  {\begin{list}{$\triangleright$}{%
   \setlength{\parskip}{0mm}
   \setlength{\topsep}{.4\baselineskip}
   \setlength{\rightmargin}{0mm}
   \setlength{\listparindent}{0mm}
   \setlength{\itemindent}{0mm}
   \setlength{\labelwidth}{2ex}
   \setlength{\itemsep}{.4\baselineskip}
   \setlength{\parsep}{0mm}
   \setlength{\partopsep}{0mm}
   \setlength{\labelsep}{1ex}
   \setlength{\leftmargin}{\labelwidth+\labelsep}
   \let\makelabel\mylabel}}{%
   \end{list}\vspace*{-1.3mm}}
\parindent0ex
\parskip2ex
\newcounter{quesito}
\newenvironment{question}{\bigskip\addtocounter{quesito}{1}\bigskip\bigskip\par\textbf{Quesito \thequesito.\kern1ex}}{\vspace{\parskip}}
\newenvironment{xquestion}{\bigskip\addtocounter{quesito}{1}\bigskip\bigskip\par\textbf{Quesito \thequesito.\kern1ex}}{\vspace{\parskip}}
\newenvironment{answer}{\par\textbf{Risposta\quad}}{\vspace{\parskip}}

\pagestyle{empty} 

\begin{document}
\begin{pycode}
import random
random.seed(2548445)
\end{pycode}

%1
\begin{question}
\def\RR{{\mathds R}}
\begin{pycode}
from sympy import *
x = symbols('x')
k = [Rational( i ) for i in random.sample([3,5,7],1) ]
\end{pycode}
Si consideri la seguente equazione differenziale a variabili separabili \(y' = x^3 y^2\).
\begin{itemize}
\item[1.] Determinarne eventuali soluzioni costanti.
\item[2.] Trovare la soluzione del problema di Cauchy
\[\begin{cases} y' = x^3 y^2 \cr y(0) = \py{latex(k[0])}  \end{cases}\]
\end{itemize}
\begin{answer}

{\color{blue}
La soluzione costante \`e data dalla funzione $y(x) = 0$, $\forall x \in \RR$.
\hfill Risposta 1\kern0ex}

\smallskip
{\color{blue} La soluzione del problema di Cauchy \`e data dalla funzione $y(x) = \cfrac{\py{latex(4*k[0])}}{4-\py{latex(k[0])}x^4}$.
\hfill Risposta 2\kern0ex}

\end{answer}

\end{question}
%2
\begin{question}
\def\RR{{\mathds R}}
Si consideri il seguente problema di Cauchy:
\[\begin{cases} y' = x^3 y^2 \cr y(0) = 4 \end{cases}\]
\begin{itemize}
\item[1.] Trovare la soluzione del problema di Cauchy.
\item[2.] Determinare l'intervallo massimale di esistenza della soluzione.

\end{itemize}
\begin{answer}

{\color{blue}
La soluzione del problema di Cauchy \`e data dalla funzione $y(x) = \cfrac{4}{1-x^4}$.
\hfill Risposta 1\kern0ex}

\smallskip
{\color{blue} L'intervallo massimale \`e $(-1,1)$.
\hfill Risposta 2\kern0ex}

\end{answer}
\end{question}
%3
\begin{question}
\def\RR{{\mathds R}}
\begin{pycode}
from sympy import *
x = symbols('x')
k = [Rational( i ) for i in random.sample([1,2,3,4,5,6],1) ]
h = [Rational( i ) for i in random.sample([0,1,2,3,4,5,6],1) ]
\end{pycode}
Si consideri il seguente problema di Cauchy:
\[\begin{cases} y' = \cfrac{y^2 - \py{latex(k[0]**2)}}{xy} \cr y(\py{latex(h[0])}) = \py{latex(-k[0])} \end{cases}\]
\begin{itemize}
\item[1.] Trovare la soluzione del problema di Cauchy.
\item[2.] Determinare l'intervallo massimale di esistenza della soluzione.

\end{itemize}
\begin{answer}

{\color{blue}
La soluzione del problema di Cauchy \`e data dalla funzione $y(x) = -\py{latex(k[0])}$.
\hfill Risposta 1\kern0ex}

\smallskip
{\color{blue} L'intervallo massimale \`e $(0, +\infty)$.
\hfill Risposta 2\kern0ex}

\end{answer}
\end{question}
%4
\begin{question}
\def\RR{{\mathds R}}
\begin{pycode}
from sympy import *
x = symbols('x')
h1 = [Rational( i ) for i in random.sample([2,3,4,5,6],1) ]
h2 = [Rational( i ) for i in random.sample([1,2,3,4],1) ]
\end{pycode}
Si consideri la seguente equazione differenziale a variabili separabili \(xy' = y\).
\begin{itemize}
\item[1.] Determinarne eventuali soluzioni costanti.
\item[2.] Trovare la soluzione del problema di Cauchy
\[\begin{cases} xy' = y \cr y(\py{latex(h1[0])}) = \py{latex(h2[0])}  \end{cases}\]
\end{itemize}
\begin{answer}

{\color{blue}
La soluzione costante \`e data dalla funzione $y(x) = 0$, $\forall x \in \RR$.
\hfill Risposta 1\kern0ex}

\smallskip
{\color{blue} La soluzione del problema di Cauchy \`e data dalla funzione $y(x) = \py{latex(h2[0]/h1[0])}x$.
\hfill Risposta 2\kern0ex}

\end{answer}
\end{question}
%5
\begin{question}
\def\RR{{\mathds R}}
\begin{pycode}
from sympy import *
x = symbols('x')
h = [Rational( i ) for i in random.sample([2,3,4],1) ]
\end{pycode}
Si consideri il seguente problema di Cauchy:
\[\begin{cases} y' =-xe^{-y} \cr y(0) = \py{latex(h[0])} \end{cases}\]
\begin{itemize}
\item[1.] Trovare la soluzione del problema di Cauchy.
\item[2.] Determinare l'intervallo massimale di esistenza della soluzione.

\end{itemize}
\begin{answer}

{\color{blue}
La soluzione del problema di Cauchy \`e data dalla funzione $y(x) = \ln(-\cfrac{x^2}{2}+ e^{\py{latex(h[0])}})$.
\hfill Risposta 1\kern0ex}

\smallskip
{\color{blue} L'intervallo massimale \`e $(-\sqrt{2e^{\py{latex(h[0])}}}, \sqrt{2e^{\py{latex(h[0])}}})$.
\hfill Risposta 2\kern0ex}

\end{answer}
\end{question}
%6
\begin{question}
\def\RR{{\mathds R}}
Si consideri la seguente equazione differenziale a variabili separabili \(y' = \cfrac{1-e^{-y}}{2x+1}\).
\begin{itemize}
\item[1.] Determinarne eventuali soluzioni costanti.
\item[2.] Trovare la soluzione del problema di Cauchy
\[\begin{cases} y' = \cfrac{1-e^{-y}}{2x+1} \cr y(0) = 0  \end{cases}\]
\end{itemize}
\begin{answer}

{\color{blue}
La soluzione costante \`e data dalla funzione $y(x) = 0$.
\hfill Risposta 1\kern0ex}

\smallskip
{\color{blue} La soluzione del problema di Cauchy \`e data dalla funzione $y(x) = 0$.
\hfill Risposta 2\kern0ex}

\end{answer}
\end{question}
%7
\begin{question}
\def\RR{{\mathds R}}
\begin{pycode}
from sympy import *
x = symbols('x')
k = [Rational( i ) for i in random.sample([1,2,3,4],1) ]
\end{pycode}
Si consideri la seguente equazione differenziale a variabili separabili \(y' = \cfrac{4x^3}{y}\).
\begin{itemize}
\item[1.] Determinarne eventuali soluzioni costanti.
\item[2.] Trovare la soluzione del problema di Cauchy
\[\begin{cases} y' = \cfrac{4x^3}{y} \cr y(0) = \py{latex(k[0])}  \end{cases}\]
\end{itemize}
\begin{answer}

{\color{blue}
Non ci sono soluzioni costanti.
\hfill Risposta 1\kern0ex}

\smallskip
{\color{blue} La soluzione del problema di Cauchy \`e data dalla funzione $y(x) = \sqrt{2x^4 + \py{latex(k[0]**2)}}$.
\hfill Risposta 2\kern0ex}

\end{answer}
\end{question}
%8
\begin{question}
\def\RR{{\mathds R}}
\begin{pycode}
from sympy import *
x = symbols('x')
k = [Rational( i ) for i in random.sample([1,2,3,4],1) ]
\end{pycode}
Si consideri il seguente problema di Cauchy:
\[\begin{cases} y' = \cfrac{4x^3}{y} \cr y(0) = -\py{latex(k[0])} \end{cases}\]
\begin{itemize}
\item[1.] Trovare la soluzione del problema di Cauchy.
\item[2.] Determinare l'intervallo massimale di esistenza della soluzione.

\end{itemize}
\begin{answer}

{\color{blue}
La soluzione del problema di Cauchy \`e data dalla funzione $y(x) = -\sqrt{2x^4 + \py{latex(k[0]**2)}}$.
\hfill Risposta 1\kern0ex}

\smallskip
{\color{blue} L'intervallo massimale \`e $\RR$.
\hfill Risposta 2\kern0ex}

\end{answer}
\end{question}


\end{document}
