\documentclass[11pt,twoside,a4paper]{article}
\usepackage[T1]{fontenc}
\usepackage[utf8]{inputenc}
\usepackage[top=20mm, head=6mm, headsep=6mm, foot=6mm, bottom= 15mm, left=20mm, right=20mm]{geometry}
\usepackage{amsmath}
\usepackage{pythontex}
\usepackage{fancyhdr}
\usepackage{comment}
\usepackage{graphicx}
\usepackage{mathtools}
\newcounter{compito}
\newcounter{foglio}
\newcounter{quesito}
\pagestyle{fancy}
%\chead{}
\lhead{Matematica e Biostatistica data_esame}
\cfoot{}
\fancyfoot[RO]{\noindent\rlap{\hskip0mm\thecompito~\Alph{foglio}1}}
\fancyfoot[RE]{\noindent\rlap{\hskip0mm\thecompito~\Alph{foglio}2}}
\newenvironment{quesito}{\par\noindent\textbf{Quesito \thequesito.\/}}{\addtocounter{quesito}{1}}
\newenvironment{quesito*}{\par\noindent\textbf{Quesito \thequesito.\/}}{\addtocounter{quesito}{1}}
\newenvironment{answer}{\par\noindent\textbf{Risposta.\ }}{}
\parindent0ex
\parskip1ex
\def\epsilon{\varepsilon}
\excludecomment{answer}
%%%%%%%%%%%%%%%%%%
%%%%%%%%%%%%%%%%%%%%
%%%%%%%%%%%%%%%%%%
\begin{document}

\begin{pycode}
import random
random.seed('data_esame'+"\studente")
\end{pycode}


\addtocounter{compito}{1}
\setcounter{foglio}{1}
\setcounter{quesito}{1}
\rhead{\student}


\vfill
\setcounter{foglio}{1}
\begin{quesito}
t_test_1t
t_test_2t
\end{quesito}
\vfill
\begin{quesito}
EVar
\end{quesito}
\vfill

\clearpage
\begin{quesito}
bayes
\end{quesito}
\vfill
\begin{quesito}
binom
z_dist
\end{quesito}
\vfill
{\hrulefill\scriptsize

Si assumano noti i valori delle seguenti funzioni




}


\clearpage\setcounter{foglio}{2}
\begin{quesito}
eq_diff
\end{quesito}
\vfill

\clearpage
\begin{quesito}
eq_diff_stab
\end{quesito}
\vfill

\clearpage\setcounter{foglio}{3}
\begin{quesito}
dinamica_lineare
\end{quesito}
\vfill

\end{document}

\begin{students}
\end{students}
