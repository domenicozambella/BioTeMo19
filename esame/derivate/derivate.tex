\documentclass[11pt,twoside,a4paper]{article}
\usepackage[T1]{fontenc}
\usepackage[utf8]{inputenc}
\usepackage[top=20mm, head=6mm, headsep=6mm, foot=6mm, bottom= 15mm, left=15mm, right=15mm]{geometry}
\usepackage{amsmath}
\usepackage{calc} 
\usepackage{pythontex}
\usepackage{dsfont}
\newcommand{\mylabel}[1]{#1\hfill}
\renewenvironment{itemize}
  {\begin{list}{$\triangleright$}{%
   \setlength{\parskip}{0mm}
   \setlength{\topsep}{.4\baselineskip}
   \setlength{\rightmargin}{0mm}
   \setlength{\listparindent}{0mm}
   \setlength{\itemindent}{0mm}
   \setlength{\labelwidth}{2ex}
   \setlength{\itemsep}{.4\baselineskip}
   \setlength{\parsep}{0mm}
   \setlength{\partopsep}{0mm}
   \setlength{\labelsep}{1ex}
   \setlength{\leftmargin}{\labelwidth+\labelsep}
   \let\makelabel\mylabel}}{%
   \end{list}\vspace*{-1.3mm}}
\parindent0ex
\parskip2ex
\newcounter{quesito}
\newenvironment{question}{\bigskip\addtocounter{quesito}{1}\bigskip\bigskip\par\textbf{Quesito \thequesito.\kern1ex}}{\vspace{\parskip}}
\newenvironment{xquestion}{\bigskip\addtocounter{quesito}{1}\bigskip\bigskip\par\textbf{Quesito \thequesito.\kern1ex}}{\vspace{\parskip}}
\newenvironment{answer}{\par\textbf{Risposta\quad}}{\vspace{\parskip}}

\pagestyle{empty} 

\begin{document}
\begin{pycode}
import random
random.seed(2548445)
\end{pycode}
%1
\begin{question}
\def\RR{{\mathds R}}
\begin{pycode}
from sympy import *
x = symbols('x')
n =[Rational( i ) for i in random.sample([1,2,3,4,5],1) ]
\end{pycode}
Si consideri una funzione $f(x)$ la cui derivata prima è data dalla funzione $f'(x) = e^{\py{latex(-n[0]*x)}}$.
\begin{itemize}
\item[1.] Indicare gli intervalli in cui la funzione $f(x)$ cresce e quelli in cui la funzione decresce.
\item[2.] Trovare massimi e minimi locali di $f(x)$.
\end{itemize}
\begin{answer}

{\color{blue}
$f(x)$ cresce in $(-\infty, \infty) = \RR$}, infatti $e^{\py{latex(-n[0]*x)}} > 0$ per ogni $x \in \RR$

{\color{blue}
\hfill Risposta 1\kern19ex}

{\color{blue}
$f(x)$ non ha massimi e minimi locali
\hfill Risposta 2\kern19ex}

\end{answer}
\end{question}
%2
\begin{question}
\def\RR{{\mathds R}}
\begin{pycode}
from sympy import *
x = symbols('x')
n =[Rational( i ) for i in random.sample([2,3,5,6,7],1) ]
\end{pycode}
Si consideri una funzione $f(x)$ definita su tutto $\RR$ la cui derivata prima è data dalla funzione $f'(x) = \cfrac{x^2 - \py{latex(n[0])}}{x}$.
\begin{itemize}
\item[1.] Indicare gli intervalli in cui la funzione $f(x)$ cresce e quelli in cui la funzione decresce.
\item[2.] Trovare massimi e minimi locali di $f(x)$.
\end{itemize}
\begin{answer}

{\color{blue}
$f(x)$ cresce in $(-\sqrt{\py{latex(n[0])}}, 0)$ e $(\sqrt{\py{latex(n[0])}}, +\infty)$; $f(x)$ decresce in $(-\infty, -\sqrt{\py{latex(n[0])}})$ e $(0, \sqrt{\py{latex(n[0])}})$}

{\color{blue}
\hfill Risposta 1\kern19ex}

{\color{blue}
$f(x)$ ha minimi locali in $-\sqrt{\py{latex(n[0])}}$ e $\sqrt{\py{latex(n[0])}}$
\hfill Risposta 2\kern19ex}

\end{answer}
\end{question}
%3
\begin{question}
\def\RR{{\mathds R}}
\begin{pycode}
from sympy import *
x = symbols('x')
n =[Rational( i ) for i in random.sample([1,2,3,4,5,6,7],1) ]
\end{pycode}
Si consideri una funzione $f(x)$ definita su tutto $\RR$ la cui derivata prima è data dalla funzione $f'(x) = \cfrac{x^2 + \py{latex(n[0])}}{x}$.
\begin{itemize}
\item[1.] Indicare gli intervalli in cui la funzione $f(x)$ cresce e quelli in cui la funzione decresce.
\item[2.] Trovare massimi e minimi locali di $f(x)$.
\end{itemize}
\begin{answer}

{\color{blue}
$f(x)$ cresce in $(0, +\infty)$; $f(x)$ decresce in $(-\infty, 0)$ }

{\color{blue}
\hfill Risposta 1\kern19ex}

{\color{blue}
$f(x)$ non ha minimi e massimi locali.
\hfill Risposta 2\kern19ex}

\end{answer}
\end{question}
%4
\begin{question}
\def\RR{{\mathds R}}
\begin{pycode}
from sympy import *
x = symbols('x')
n =[Rational( i ) for i in random.sample([2,3,4,5,6,7],1) ]
\end{pycode}
Si consideri una funzione $f(x)$ definita sull'intervallo $(0, +\infty)$ la cui derivata prima è data dalla funzione $f'(x) = \py{latex(n[0])}\log (x)$.
\begin{itemize}
\item[1.] Indicare gli intervalli in cui la funzione $f(x)$ cresce e quelli in cui la funzione decresce.
\item[2.] Trovare massimi e minimi locali di $f(x)$.
\end{itemize}
\begin{answer}

{\color{blue}
$f(x)$ cresce in $(1, +\infty)$; $f(x)$ decresce in $(0, 1)$ }

{\color{blue}
\hfill Risposta 1\kern19ex}

{\color{blue}
$f(x)$ ha minimo locale in $1$
\hfill Risposta 2\kern19ex}

\end{answer}
\end{question}
%5
\begin{question}
\begin{pycode}
from sympy import *
x = symbols('x')
n =[Rational( i ) for i in random.sample([1,2,3,4,5,6,7],1) ]
\end{pycode}
Si consideri un corpo lasciato cadere da una torre alta 500 metri. Sia  $f(t) = 5 t^2$ la funzione che ne descrive la distanza dalla cima della torre ad ogni secondo (quando $t=0$, $f(t) = 0$ ovvero il corpo si trova in cima alla torre).
\begin{itemize}
\item[1.] Qual è la velocità istantanea del corpo dopo $\py{latex(n[0])}$ secondi?
\item[2.] Qual è la velocità istantanea del corpo quando tocca terra?
\end{itemize}
\begin{answer}

La funzione $f'(t) = 10 t$ descrive l'andamento della velocità istantanea, quindi
{\color{blue}
$f'(\py{latex(n[0])}) = \py{latex(10*n[0])}$ }

{\color{blue}
\hfill Risposta 1\kern19ex}

Il corpo tocca terra quando $f(t) = 500$, ovvero quando $5t^2 = 500$, ovvero a $t = 10$, quindi
{\color{blue}
$f'(10) = 100$
\hfill Risposta 2\kern19ex}

\end{answer}
\end{question}
%6
\begin{question}
Sia data la funzione $f(x) = x^2 + 3 x$
\begin{itemize}
\item[1.] Scrivere l'equazione della retta tangente nel punto (-5, 10).
\item[2.] In quali intervalli la funzione è decrescente?
\end{itemize}
\begin{answer}

{\color{blue}
$y = -7x -25$ }

{\color{blue}
\hfill Risposta 1\kern19ex}

{\color{blue}
$f(x)$ decresce in $(-\infty, -\frac{3}{2})$
\hfill Risposta 2\kern19ex}

\end{answer}
\end{question}
%7
\begin{question}
\def\RR{{\mathds R}}
\begin{pycode}
from sympy import *
x = symbols('x')
n =[Rational( i ) for i in random.sample([1,2,3,4,5,6,7],1) ]
\end{pycode}
Sia data la funzione $f(x) = x^3 + \py{latex(n[0])}$
\begin{itemize}
\item[1.] Scrivere l'equazione della retta tangente nel punto $(1, \py{latex(1+n[0])})$.
\item[2.] In quali intervalli la funzione è crescente?
\end{itemize}
\begin{answer}

{\color{blue}
$y = \py{latex(3*x+1+n[0]-3)}$ }

{\color{blue}
\hfill Risposta 1\kern19ex}

{\color{blue}
$f(x)$ cresce in $(-\infty,+\infty) = \RR$
\hfill Risposta 2\kern19ex}

\end{answer}
\end{question}
%8
\begin{question}
\begin{pycode}
from sympy import *
t = symbols('t')
n =[Rational( i ) for i in random.sample([2,3,4,5,6,7],1) ]
m =[Rational( i ) for i in random.sample([2,3,4,5,6],1) ]
u =[Rational( i ) for i in random.sample([1,2,3,4,5],1) ]
s =[Rational( i ) for i in random.sample([100,101,102,103,104,105],1) ]
\end{pycode}
Si consideri una particella che si muove lungo una retta. Sia $f(t) = \py{latex(m[0])} t^3 + \py{latex(n[0])} t$ la funzione che ne descrive la distanza in metri dal punto di partenza ogni secondo.
\begin{itemize}
\item[1.] Qual è la velocità istantanea del corpo dopo $\py{latex(u[0])}$ secondi?
\item[2.] Quando la velocità del corpo è superiore a $\py{latex(s[0])}$ metri al secondo?
\end{itemize}
\begin{answer}

La velocità istantanea è descritta dalla funzione $f'(t) = \py{latex(3*m[0])} t^2 + \py{latex(n[0])}$ quindi
{\color{blue}
$f'(\py{latex(u[0])}) = \py{latex(3*m[0]*u[0]**2+n[0])}$ }

{\color{blue}
\hfill Risposta 1\kern19ex}

La velocità supera $\py{latex(s[0])}$ quando $f'(t) = \py{latex(3*m[0])} t^2 + \py{latex(n[0])} > \py{latex(s[0])}$ quindi
{\color{blue}
$t > \sqrt{\py{latex((s[0]-n[0])/(3*m[0]))}}$
\hfill Risposta 2\kern19ex}

\end{answer}
\end{question}
%9
\begin{question}
\begin{pycode}
from sympy import *
n =[Rational( i ) for i in random.sample([2,3,4,5,6,7],1) ]
\end{pycode}
Si consideri la funzione $f(x) = x^3 + \py{latex(n[0])}x + 1$.
\begin{itemize}
\item[1.] Determinare la derivata prima $f'(x)$.
\item[2.] Trovare massimi e minimi locali di $f(x)$.
\end{itemize}
\begin{answer}

{\color{blue}
$f'(x) = 3x^2 + \py{latex(n[0])}$ }

{\color{blue}
\hfill Risposta 1\kern19ex}

La derivata è sempre positiva quindi
{\color{blue}
$f(x)$ non ha massimi e minimi locali
\hfill Risposta 2\kern19ex}

\end{answer}
\end{question}
%10
\begin{question}
Si consideri la funzione $f(x) = x \cos (x)$.
\begin{itemize}
\item[1.] Determinare la derivata $f'(x)$.
\item[2.] Scrivere l'equazione della retta tangente a $f(x)$ nel punto $(\pi, -\pi)$.
\end{itemize}
\begin{answer}

{\color{blue}
$f'(x) = x \sin x + \cos x$ 
\hfill Risposta 1\kern19ex}

{\color{blue}
$y = -x - 2 \pi$
\hfill Risposta 2\kern19ex}

\end{answer}
\end{question}
%11
\begin{question}
Si consideri la funzione $f(x) = \sqrt{x}$.
\begin{itemize}
\item[1.] Scrivere l'approssimazione lineare di $f(x)$ in 4.
\item[2.] Usare il risultato precedente per approssimare i valori di $\sqrt{4.1}$ e $\sqrt{3.9}$.
\end{itemize}
\begin{answer}

L'approssimazione lineare di $f(x)$ in 4 è data da $f'(4)(x-4) + f(4)$, essendo $f'(x) = \frac{1}{2\sqrt{x}}$ si ha
{\color{blue}
$\frac{1}{4} x + 1$ 
\hfill Risposta 1\kern19ex}

{\color{blue}
$\sqrt{4.1} \cong 2.025$ e $\sqrt{3.9} \cong 1.975$
\hfill Risposta 2\kern19ex}

\end{answer}
\end{question}
%12
\begin{question}
Si consideri la funzione $f(x) = e^x$.
\begin{itemize}
\item[1.] Scrivere l'approssimazione lineare di $f(x)$ in 1.
\item[2.] Usare il risultato precedente per approssimare i valori di $e^{0.9}$ e $e^{1.1}$ (si approssimi $e$ con $2.7$).
\end{itemize}
\begin{answer}

L'approssimazione lineare di $f(x)$ in 1 è data da $f'(1)(x-1) + f(1)$, essendo $f'(x) = e^x$ si ha
{\color{blue}
$e \cdot x$ 
\hfill Risposta 1\kern19ex}

{\color{blue}
$e^{0.9} \cong 2.43$ e $e^{1.1} \cong 2.97$
\hfill Risposta 2\kern19ex}

\end{answer}
\end{question}
%13
\begin{question}
Si consideri la funzione $f(x) = \sqrt[3]{x}$.
\begin{itemize}
\item[1.] Scrivere l'approssimazione lineare di $f(x)$ in 1.
\item[2.] Usare il risultato precedente per approssimare i valori di $\sqrt[3]{1.1}$ e $\sqrt[3]{1.2}$.
\end{itemize}
\begin{answer}

L'approssimazione lineare di $f(x)$ in 1 è data da $f'(1)(x-1) + f(1)$, essendo $f'(x) = \frac{1}{3\sqrt[3]{x^2}}$ si ha
{\color{blue}
$\frac{1}{3}x+\frac{2}{3}$ 
\hfill Risposta 1\kern19ex}

{\color{blue}
$\sqrt[3]{1.1} \cong 1.0\overline{3}$ e $\sqrt[3]{1.2} \cong 1.0\overline{6}$
\hfill Risposta 2\kern19ex}

\end{answer}
\end{question}

\end{document}