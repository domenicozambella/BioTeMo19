\documentclass[11pt,twoside,a4paper]{article}
\usepackage[T1]{fontenc}
\usepackage[utf8]{inputenc}
\usepackage[top=20mm, head=6mm, headsep=6mm, foot=6mm, bottom= 15mm, left=15mm, right=15mm]{geometry}
\usepackage{amsmath}
\usepackage{calc} 
\usepackage{pythontex}
\newcommand{\mylabel}[1]{#1\hfill}
\renewenvironment{itemize}
  {\begin{list}{$\triangleright$}{%
   \setlength{\parskip}{0mm}
   \setlength{\topsep}{.4\baselineskip}
   \setlength{\rightmargin}{0mm}
   \setlength{\listparindent}{0mm}
   \setlength{\itemindent}{0mm}
   \setlength{\labelwidth}{2ex}
   \setlength{\itemsep}{.4\baselineskip}
   \setlength{\parsep}{0mm}
   \setlength{\partopsep}{0mm}
   \setlength{\labelsep}{1ex}
   \setlength{\leftmargin}{\labelwidth+\labelsep}
   \let\makelabel\mylabel}}{%
   \end{list}\vspace*{-1.3mm}}
\parindent0ex
\parskip2ex
\def\Pr{{\rm Pr\,}}
\def\Ex{{\rm E\,}}
\def\Var{{\rm Var\,}}
\newcounter{quesito}
\newenvironment{question}{\bigskip\addtocounter{quesito}{1}\par\textbf{Quesito \thequesito.\kern1ex}}{\vspace{\parskip}}
\newenvironment{xquestion}{\bigskip\addtocounter{quesito}{1}\par\textbf{Quesito \thequesito.\kern1ex}}{\vspace{\parskip}}
\newenvironment{answer}{\par\textbf{Risposta\quad}}{\vspace{\parskip}}

\pagestyle{empty} 

\begin{document}
\colorbox{blue!10}{\begin{minipage}{\textwidth}
Domande per verificare la comprensione del significato di errori del II tipo e di potenza.\bigskip

N.B. Le domande potrebbero contenere informazioni irrilevanti.
\end{minipage}}

\bigskip\bigskip

\begin{pycode}
import random
random.seed(25846456445)
ESAME = False
\end{pycode}

% \begin{xquestion}
% \begin{pycode}
% from sympy import Rational, Integer
% from scipy.stats import norm
% mu = random.choice( [ [1,2,4], [2,4,7], [2,5,9] ] )
% n = random.choice( [2, 3, 4, 5 ] ) # = sqrt(n)
% sigma = random.choice( [5, 7] )
% alpha = random.choice( [0.1, 0.05, 0.02, 0.01] )
% zalpha = - norm.ppf(alpha) / 2
% xalpha1 = round( mu[1] - zalpha*sigma / n, 2 )
% xalpha2 = round( mu[1] + zalpha*sigma / n, 2 )
% \end{pycode}
% Preleviamo un campione di rango $n=\py{n**2}$ da una popolazione con distribuzione $N(\mu,\sigma^2)$. Sappiamo che la deviazione standard è $\sigma=\py{sigma}$. La media $\mu$ invece potrebbe avere uno qualsiasi dei tre valori $\py{mu[0]}$, $\py{mu[1]}$, o $\py{mu[2]}$. 
% 
% Vogliamo testare $H_0:\mu=\py{mu[1]}$ contro $H_A:\mu\in\{\py{mu[0]},\py{mu[2]}\}$. Fissiamo come significatività $\alpha=\py{alpha}$ otteniamo che l'intervallo critico è per uno z-test a due code è $[\py{xalpha1}, \ \py{xalpha2}]$.
% 
% Qual è la potenza del test? Esprimere il risutato numerico tramite (solo) le funzioni elencate in calce.
% 
% \begin{answer}
% 
% Notazione: $\bar X\sim N(\py{mu[0]},\ \sigma^2/n)$
% 
% Il caso in cui $H_A$ meno distante da $H_0$ si verifica quando $\mu = \py{mu[0]}$. Calcoliamo la probabilità di un errore del II tipo (falso negativo) in questo caso
% 
% 
% $\beta
% \ =\ 
% \Pr\big(\py{xalpha1}\le \bar X \le\py{xalpha2} \big)
% \ =\ 
% \Pr\bigg(\dfrac{\py{xalpha1-mu[0]} }{\sigma/\sqrt{n} }\ \le\ \dfrac{\bar X - \py{mu[0]} }{\sigma/\sqrt{n}}\ \le\ \dfrac{\py{xalpha2-mu[0]} }{\sigma/\sqrt{n} }\bigg)
% \ =\ 
% \Pr\big(\py{n * (xalpha1-mu[0] ) / sigma }\le Z\le \py{n * (xalpha2-mu[0] ) / sigma }\big)$
% 
% 
% {\color{blue}$\beta
% \ \le\ $}
% {\color{blue}\tt norm.cdf(\py{n * (xalpha2-mu[0] ) / sigma }) - norm.cdf(\py{n * (xalpha1-mu[0] ) / sigma }) }{\tt\ =\ \py{round( norm.cdf(n * (xalpha2-mu[0] ) / sigma ) - norm.cdf(n * (xalpha1-mu[0] ) / sigma ), 4 ) } }{\color{blue}\hfill Risposta}
% 
% \end{answer}
% \end{xquestion}

% 11111111111111111
\begin{question}
\def\pyl#1{\py{latex(#1)}}
\def\pyr#1{\py{round(#1,4)}}
\begin{pycode}
from sympy import *
from scipy.stats import norm
mu = random.choice( [ [1,3,6], [2,4,7], [2,5,9] ] )
n = Integer( random.choice( [4, 5, 6 ] )**2 )
sigma = Integer( random.choice( [6, 7] ) )
alpha = random.choice( [0.1, 0.05, 0.02, 0.01] )
zalpha = - norm.ppf(alpha) / 2
xalpha1 = mu[1] - 1#round( mu[1] - zalpha*sigma / n, 2 )
xalpha2 = mu[1] + 1#round( mu[1] + zalpha*sigma / n, 2 )
\end{pycode}
Consideriamo una popolazione normale con $\sigma=\py{sigma}$ e media $\mu\in\{\py{mu[0]}, \py{mu[1]}, \py{mu[2]}\}$ ignota. 

Vogliamo testare $H_0:\mu=\py{mu[1]}$ contro $H_A:\mu\in\{\py{mu[0]},\py{mu[2]}\}$. Preleviamo un campione di rango $n=\py{n}$ e decidiamo di rifiutare $H_0$ se la media campionaria $\bar x$ non appartiene all'intervallo $[\py{xalpha1}, \ \py{xalpha2}]$.

\begin{itemize}
\item[1.] Qual è la significatività del test?
\item[2.] Qual è la potenza del test?
\end{itemize}

Esprimere il risutato numerico tramite (solo) le funzioni elencate in calce.

\begin{answer}

$\displaystyle\alpha\ =\ 1 - \Pr\big(\py{xalpha1}\le\bar X\le\py{xalpha2}\big)$ dove $\displaystyle\bar X\sim N(\mu,\frac{\sigma^2}{n})$ \ con \ $\mu=\py{mu[1]}$ \ e \ $\displaystyle\frac{\sigma^2}{n}=\pyl{sigma**2/n}$

$\displaystyle\phantom{\alpha}\ =\ 1- \Pr\bigg(\pyl{(xalpha1-mu[1])/(sigma/sqrt(n))}\le Z\le\pyl{(xalpha2-mu[1])/(sigma/sqrt(n))}\bigg)\ =\ $
{\color{blue}{\tt 2 * norm.cdf( \py{ (xalpha1-mu[1])/(sigma/sqrt(n)) } ) }
\hfill Risposta 1}

\smallskip
$\displaystyle\phantom{\alpha}\ =\ ${\tt\ \pyr{ 2 * norm.cdf( float(xalpha1-mu[1])/float(sigma/sqrt(n)) ) } }

\smallskip
Per calcolare la potenza assumo $H_A$ valga nel caso più sfaforevole $\mu=\py{mu[0]}$  

$\displaystyle1-\beta\ =\ 1 - \Pr\big(\py{xalpha1}\le\bar X\le\py{xalpha2}\big)$ dove $\displaystyle\bar X\sim N(\mu,\frac{\sigma^2}{n})$ \ con \ $\mu=\py{mu[0]}$ \ e \ $\displaystyle\frac{\sigma^2}{n}=\pyl{sigma**2/n}$

$\displaystyle\phantom{1-\beta}\ =\ 1- \Pr\bigg(\pyl{(xalpha1-mu[0])/(sigma/sqrt(n))}\le Z\le\pyl{(xalpha2-mu[0])/(sigma/sqrt(n))}\bigg)\ =\ $
{\color{blue}{\tt norm.cdf( \py{ (xalpha1-mu[0])/(sigma/sqrt(n)) } ) + norm.cdf( -\py{ (xalpha2-mu[0])/(sigma/sqrt(n)) } ) }
\hfill Risposta 2}

\smallskip
$\displaystyle\phantom{1-\beta}\ =\ ${\tt\ \pyr{ norm.cdf( float(xalpha1-mu[0])/float(sigma/sqrt(n)) ) + norm.cdf( -float(xalpha2-mu[0])/float(sigma/sqrt(n)) ) } }


% Il caso in cui $H_A$ meno distante da $H_0$ si verifica quando $\mu = \py{mu[0]}$. Calcoliamo la probabilità di un errore del II tipo (falso negativo) in questo caso
% 
% 
% $\beta
% \ =\ 
% \Pr\big(\py{xalpha1}\le \bar X \le\py{xalpha2} \big)
% \ =\ 
% \Pr\bigg(\dfrac{\py{xalpha1-mu[0]} }{\sigma/\sqrt{n} }\ \le\ \dfrac{\bar X - \py{mu[0]} }{\sigma/\sqrt{n}}\ \le\ \dfrac{\py{xalpha2-mu[0]} }{\sigma/\sqrt{n} }\bigg)
% \ =\ 
% \Pr\big(\py{n * (xalpha1-mu[0] ) / sigma }\le Z\le \py{n * (xalpha2-mu[0] ) / sigma }\big)$
% 
% 
% {\color{blue}$\beta
% \ \le\ $}
% {\color{blue}\tt norm.cdf(\py{n * (xalpha2-mu[0] ) / sigma }) - norm.cdf(\py{n * (xalpha1-mu[0] ) / sigma }) }{\tt\ =\ \py{round( norm.cdf(n * (xalpha2-mu[0] ) / sigma ) - norm.cdf(n * (xalpha1-mu[0] ) / sigma ), 4 ) } }{\color{blue}\hfill Risposta}

\end{answer}
\end{question}

% 222222222222
\begin{question} 
\begin{pycode}
from sympy import Rational, Integer
from scipy.stats import norm
mu = random.choice( [ [1,4], [2,7], [2,8] ] )
n = random.choice( [3, 4, 5 ] ) # = sqrt(n)
sigma = random.choice( [2, 3] )
alpha = random.choice( [0.1, 0.05, 0.02, 0.01] )
zalpha = - norm.ppf(alpha) 
xalpha1 = round( mu[0] - zalpha*sigma / n, 3 )
xalpha2 = round( mu[0] + zalpha*sigma / n, 3 )
delta = random.choice( [0.5, 1, 1.5, 2] )
muA = mu[0] + delta
risposta_numerica1 = round( norm.cdf(n * (xalpha2-(mu[0]+delta) ) / sigma ), 4 )
risposta_numerica2 = 1 - risposta_numerica1
\end{pycode}
Preleviamo un campione di rango $n=\py{n**2}$ da una popolazione con distribuzione $N(\mu,\sigma^2)$. Sappiamo che la deviazione standard è $\sigma=\py{sigma}$. La media $\mu$ invece potrebbe avere uno qualsiasi valori dei nell'intervallo $[\py{mu[0]},\ \py{mu[1]}  ]$.

Vogliamo testare $H_0:\mu=\py{mu[0]}$ contro $H_A:\mu\in (\py{mu[0]},\py{mu[1]}]$. Fissiamo come significatività $\alpha=\py{alpha}$ otteniamo che per uno z-test a coda superiore la zona di rifiuto per la media campionaria è $[\py{xalpha2}, \ +\infty)$.

\begin{itemize}
\item[1.] Nel caso $H_A:\mu\in [\py{muA},\ \py{mu[1]}]$ qual'è la massima probabilità $\beta$ di non rigettare $H_0$ (errore II tipo)?
\item[2.] Calcolare la potenza del test con l'effect-size suggerito nel punto precedente.
\end{itemize}

Esprimere il risutato numerico tramite (solo) le funzioni elencate in calce.

\begin{answer}

Il caso più sfavorevole si ottiene quando $\mu = \py{muA}$. Sia $\bar X\sim N(\py{muA},\ \sigma^2/n)$

$\beta
\ =\ 
\Pr\big(\bar X < \py{xalpha2} \big)
\ =\ 
\Pr\bigg( \dfrac{\bar X - \py{muA} }{\sigma/\sqrt{n}}\ <\ \dfrac{\py{xalpha2-muA} }{\sigma/\sqrt{n} }\bigg)
\ =\ 
\Pr\big( Z<  \py{round( n * (xalpha2-muA ) / sigma, 4) }\big)$


{\color{blue}$\beta
\ =\ $}
{\color{blue}\tt norm.cdf(\py{ round( n * (xalpha2-muA ) / sigma , 4) }) }{\tt\ =\ \py{risposta_numerica1} }{\color{blue}\hfill Risposta 1}

\medskip
Con un effect size $\delta=\py{delta}$ la potenza del test è $1-\beta=${\color{blue}\tt\ 1 - norm.cdf(\py{round( n * (xalpha2-muA ) / sigma , 4) }) }{\tt\ =\ \py{risposta_numerica2 } }{\color{blue}\hfill Risposta 2}

\end{answer}
\end{question}


\clearpage
% 33333333333333
\begin{question} 
\begin{pycode}
# dis = random.choice( [ '>', '<'])
dis =  '>'
if ESAME: dis = '<'

if dis == '<':
   dis_eq  = '\\le'
   adis    = '>'
   adis_eq = '\\ge'
   sign = '+'

if dis == '>':
   dis_eq  = '\\ge'
   adis    = '<'
   adis_eq = '\\le'
   sign = '-'
\end{pycode}
Vogliamo testare $H_0:\mu=\mu_0$ contro $H_A:\mu\py{dis}\mu_0$ per una popolazione distribuita normalmente con deviazione standard nota $\sigma$. Fissiamo una significatività $\alpha$ e una potenza $1-\beta$. L'effect-size che ci interessa è $\delta$.  Esprimere, in funzione dei parametri che assumiamo noti, le condizioni cui deve soddisfare il rango $n$ del campione.

\begin{answer}

{\color{blue}Il rango necessario è il minimo $n$ tale che $\displaystyle\Pr\bigg(Z\py{adis}\frac{x_\alpha-\mu_0\py{sign}\delta}{\sigma/\sqrt{n}}\bigg)\le\beta$

dove $x_\alpha$ tale che $\displaystyle\Pr\bigg(Z\py{dis_eq} \frac{x_\alpha-\mu_0}{\sigma/\sqrt{n}}\bigg)=\alpha$\hfill Risposta}


\end{answer}

\end{question}















\vfill
\hrulefill

Si assuma noto il valore delle seguenti funzioni della libreria {\tt scipy.stats\/}

{\tt norm.cdf(z)} = $\Pr\big(Z<{\tt z}\big)$ per $Z\sim N(0,1)$ 

{\tt norm.ppf($\alpha$)} = $z_\alpha$ dove $z_\alpha$ è tale che $\Pr\big(Z<z_\alpha\big)=\alpha$ per $Z\sim N(0,1)$ 



\end{document}

