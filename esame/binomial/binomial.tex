\documentclass[11pt,twoside,a4paper]{article}
\usepackage[T1]{fontenc}
\usepackage[utf8]{inputenc}
\usepackage[top=20mm, head=6mm, headsep=6mm, foot=6mm, bottom= 15mm, left=15mm, right=15mm]{geometry}
\usepackage{amsmath}
\usepackage{calc} 
\usepackage{pythontex}
\newcommand{\mylabel}[1]{#1\hfill}
\renewenvironment{itemize}
  {\begin{list}{$\triangleright$}{%
   \setlength{\parskip}{0mm}
   \setlength{\topsep}{.4\baselineskip}
   \setlength{\rightmargin}{0mm}
   \setlength{\listparindent}{0mm}
   \setlength{\itemindent}{0mm}
   \setlength{\labelwidth}{2ex}
   \setlength{\itemsep}{.4\baselineskip}
   \setlength{\parsep}{0mm}
   \setlength{\partopsep}{0mm}
   \setlength{\labelsep}{1ex}
   \setlength{\leftmargin}{\labelwidth+\labelsep}
   \let\makelabel\mylabel}}{%
   \end{list}\vspace*{-1.3mm}}
\parindent0ex
\parskip2ex
\newcounter{quesito}
\newenvironment{question}{\bigskip\addtocounter{quesito}{1}\bigskip\bigskip\par\textbf{Quesito \thequesito.\kern1ex}}{\vspace{\parskip}}
\newenvironment{xquestion}{\bigskip\addtocounter{quesito}{1}\bigskip\bigskip\par\textbf{Quesito \thequesito.\kern1ex}}{\vspace{\parskip}}
\newenvironment{answer}{\par\textbf{Risposta\quad}}{\vspace{\parskip}}

\pagestyle{empty} 

\begin{document}
\colorbox{blue!10}{\begin{minipage}{\textwidth}
Domande  per verificare il riconoscimento di esperimenti che si modellano con distribuzione binomiale.
\end{minipage}}

\bigskip\bigskip


\begin{pycode}
import random
random.seed(258466445)
ESAME = False
\end{pycode}


\begin{question}
\def\Pr{{\rm Pr\,}}
\def\Ex{{\rm E\,}}
\def\Var{{\rm Var\,}}
\begin{pycode}
from scipy.stats import binom
p = random.choice([0.6,0.7,0.8])
n_vittoria = random.choice([8, 10])
n_partite = n_vittoria + random.choice([1, 2, 3, 4])
if not ESAME :
   p = 0.6
   n_vittoria = 8
   n_partite = 11
risultato = 1 - binom.cdf(n_vittoria-1, n_partite, p)
risultato = round(risultato, 4)
\end{pycode}
In un gioco a due giocatori, $A$ e $B$, ogni partita vale un punto che è vinto da uno dei due giocatori (non ci sono patte). Vince il gioco chi per primo raggiunge \py{n_vittoria} punti. In ciascuna partita vince $A$ con probabilità $\py{p}$.

Qual è la probabilità che $A$ vinca il gioco in $\le\py{n_partite}$ partite~?
\begin{answer}

Sia $X\sim B(\py{n_partite},\py{p})$. Il giocatore $A$ vince se $X\ge 8$. 

$\displaystyle\Pr(X\ge\py{n_vittoria})\ =\ 1-\Pr( X\le\py{n_vittoria-1} )\ =\ ${\tt{\color{blue} 1 - binom.cdf( \py{n_vittoria-1}, \py{n_partite}, \py{p} ) }= \py{risultato} }{\color{blue}\hfill Risposta}
\end{answer}
\end{question}



\begin{question}
\begin{pycode}
n = random.choice([8,9,10,11,12])
d = random.choice([4,5])
k = random.choice([3,6])
risposta = round( 1- binom.cdf(k-1,n, 1/d), 4)
\end{pycode}
Suppose that you take a $\py{n}$-question multiple-choice quiz by randomly guessing. Each question has $\py{d}$ possible answers and only one is correct. What is the probability that answering at random you correctly guess at least $\py{k}$ answers? 

Leave the answer in impicit form using one of the functions listed below.
\begin{answer}

$X\sim B(\py{n}, 1/\py{d})$

$\Pr(X\ge\py{k})\ =\ 1-\Pr(X\le \py{k-1})\ =\ ${\tt{\color{blue}  1 - binom.cdf( \py{k-1}, \py{n}, 1/\py{d} )} = \py{risposta}}{\color{blue}\hfill Risposta}

\end{answer}
\end{question}


\begin{question}
\begin{pycode}
n = random.choice([26,27,28,29])
k = random.choice([12,13,14,15])
t = random.choice(["t","u"])
risposta = round( 1- binom.cdf(k-1,n, 1/4), 4)
\end{pycode}
Consideriamo sequenze di $\py{n}$ caratteri dell'alfabeto $\{a,g,c,\py{t}\}$. Assumiamo che tutti i caratteri occorrano con la stessa probabilità indipendentemente dalla posizione. Qual è la probabilità che due sequenze coincidano in $\ge\py{k}$ posizioni?  

Esprimere il risutato numerico tramite (solo) le funzioni elencate in calce. 
\begin{answer}


$X\sim B(\py{n}, 1/4)$

$\Pr(X\ge \py{k})\ =\ 1-\Pr(X\le \py{k-1})\ =\ ${\tt{\color{blue}  1 -  binom.cdf( \py{k-1}, \py{n}, 1/4 )} = \py{risposta}}{\color{blue}\hfill Risposta}
\end{answer}
\end{question}



\vfill
\hrulefill

Si assuma noto il valore delle seguenti funzioni della libreria {\tt scipy.stats\/} di  {\tt Python\/}

{\tt binom.pmf(k,n,p)} = $\Pr\big(X={\tt k}\big)$ dove $X\sim B({\tt n},{\tt p})$ 

{\tt binom.cdf(k,n,p)} = $\Pr\big(X\le{\tt k}\big)$ dove  $X\sim B({\tt n},{\tt p})$ 

{\tt bimom.ppf($\alpha$, n, p)} = ${\tt x_\alpha}$ dove ${\tt x_\alpha}$ è tale che $\Pr\big(X\le{\tt x_\alpha}\big)=\alpha$ per $X\sim B({\tt n},{\tt p})$ 

\end{document}

