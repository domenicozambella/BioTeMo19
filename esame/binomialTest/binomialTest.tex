\documentclass[11pt,twoside,a4paper]{article}
\usepackage[T1]{fontenc}
\usepackage[utf8]{inputenc}
\usepackage[top=20mm, head=6mm, headsep=6mm, foot=6mm, bottom= 15mm, left=15mm, right=15mm]{geometry}
\usepackage{amsmath}
\usepackage{calc} 
\usepackage{pythontex}
\newcommand{\mylabel}[1]{#1\hfill}
\renewenvironment{itemize}
  {\begin{list}{$\triangleright$}{%
   \setlength{\parskip}{0mm}
   \setlength{\topsep}{.4\baselineskip}
   \setlength{\rightmargin}{0mm}
   \setlength{\listparindent}{0mm}
   \setlength{\itemindent}{0mm}
   \setlength{\labelwidth}{2ex}
   \setlength{\itemsep}{.4\baselineskip}
   \setlength{\parsep}{0mm}
   \setlength{\partopsep}{0mm}
   \setlength{\labelsep}{1ex}
   \setlength{\leftmargin}{\labelwidth+\labelsep}
   \let\makelabel\mylabel}}{%
   \end{list}\vspace*{-1.3mm}}
\parindent0ex
\parskip2ex
\newcounter{quesito}
\newenvironment{question}{\bigskip\addtocounter{quesito}{1}\bigskip\bigskip\par\textbf{Quesito \thequesito.\kern1ex}}{\vspace{\parskip}}
\newenvironment{xquestion}{\bigskip\addtocounter{quesito}{1}\bigskip\bigskip\par\textbf{Quesito \thequesito.\kern1ex}}{\vspace{\parskip}}
\newenvironment{answer}{\par\textbf{Risposta\quad}}{\vspace{\parskip}}

\pagestyle{empty} 

\begin{document}
\colorbox{blue!10}{\begin{minipage}{\textwidth}
Domande  per verificare il riconoscimento di esperimenti che si modellano con distribuzione binomiale.
\end{minipage}}

\bigskip\bigskip


\begin{pycode}
import random
random.seed(258466445)
ESAME = False
\end{pycode}



\begin{question}
\def\Pr{{\rm Pr\,}}
\def\Ex{{\rm E\,}}
\def\Var{{\rm Var\,}}
\begin{pycode}
from scipy.stats import binom
irr = random.sample([4,5,6], 2)
n = 10 * random.choice([8,9,10])
p0 = random.choice([40, 50, 60])
p1 = p0-10
alpha =  random.choice([2,3,4,5])
beta =  random.choice([2,3,4,5])
k = int( n*p0/100 ) - 1
risultato1 = binom.cdf(k,n,p0/100)
risultato1 = round(risultato1, 4)
risultato2 = 1 - binom.cdf(k,n,p1/100)
risultato2 = round(risultato2, 4)
risultato3 = binom.ppf(alpha/100,n*10,p0/100)
risultato3 = int(risultato3)
risultato4 = binom.ppf(1 - beta/100, n*10, p1/100)
risultato4 = int(risultato4)
\end{pycode}
Una fabbrica produce confezioni di biglie rosse e blu. Una confezione corretta contiene $\py{irr[0]}\cdot 10^{\py{irr[1]}}$ biglie con circa il $\py{p0}\%$ di biglie rosse. 

Vogliamo essere ragionevolmente sicuri che la percentuale non scenda mai sotto $\py{p1}\%$.  Stabiliamo quindi due livelli di controllo. 
Al primo controllo preleviamo $\py{n}$ biglie a caso da ogni confezione e se $\le\py{k}$ biglie sono rosse la confezione viene sottoposta a ulteriori controlli. Altrimenti viena dichiarata soddisfacente.

1. Si calcoli la probabilità che una confezione con $\py{p0}\%$ di biglie rosse venga sottoposta al secondo controllo.

2. Si calcoli la probabilità che una confezione con $\py{p1}\%$ di biglie rosse venga dichiarata soddisfacente.

Il secondo controllo comporta l'estrazione di altre biglie,  $\py{n*10}$ in totale. Se meno di $x\%$ è rosso la confezione viene scartata definitivamente, altrimenti viene dichiarata soddisfacente.

3. A quanto dovremmo fissare $x$ per non scartare al secondo controllo più del $\py{alpha}\%$ di confezioni con $\py{p0}\%$ di biglie rosse?

4. A quanto dovremmo fissare $x$ per non  dichiare soddisfacente al secondo controllo più del $\py{beta}\%$ di confezioni con $\py{p1}\%$ di bigie rosse?

Si trattino tutte le estrazioni come estrazioni con \textit{reimbussolamento}.

\begin{answer}

{\tt {\color{blue} binom.cdf( \py{k}, \py{n}, \py{p0/100} )} = \py{risultato1}}\hfill  {\color{blue}Risposta 1}

{\tt {\color{blue} 1 - binom.cdf( \py{k-1}, \py{n}, \py{p1/100} )} = \py{risultato2}}\hfill  {\color{blue}Risposta 2}

{\tt {\color{blue} binom.ppf( \py{alpha/100}, \py{n*10}, \py{p0/100} )} = \py{risultato3}}\hfill  {\color{blue}Risposta 3}

{\tt {\color{blue} binom.ppf( \py{1- beta/100}, \py{n*10}, \py{p1/100} )} = \py{risultato4}}\hfill {\color{blue}Risposta 4}

\end{answer}
\end{question}



\vfill
\hrulefill

Si assuma noto il valore delle seguenti funzioni della libreria {\tt scipy.stats\/} di  {\tt Python\/}

{\tt binom.pmf(k,n,p)} = $\Pr\big(X={\tt k}\big)$ dove $X\sim B({\tt n},{\tt p})$ 

{\tt binom.cdf(k)} = $\Pr\big(X\le{\tt k}\big)$ dove  $X\sim B({\tt n},{\tt p})$ 

{\tt bimom.ppf(q, n, p)} = ${\tt k}$ dove ${\tt k}$ è tale che $\Pr\big(X\le{\tt k}\big)\cong{\tt q}$ per $X\sim B({\tt n},{\tt p})$ 

\end{document}


