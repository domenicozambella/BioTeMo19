\documentclass[11pt,twoside,a4paper]{article}
\usepackage[T1]{fontenc}
\usepackage[utf8]{inputenc}
\usepackage[top=20mm, head=6mm, headsep=6mm, foot=6mm, bottom= 15mm, left=15mm, right=15mm]{geometry}
\usepackage{amsmath}
\usepackage{xcolor}
\usepackage{pythontex}
\parindent0ex
\parskip2ex
\def\Pr{{\rm Pr\,}}
\def\Ex{{\rm E\,}}
\def\Var{{\rm Var\,}}
\newcounter{quesito}
\newenvironment{question}{\addtocounter{quesito}{1}\bigskip\bigskip\par\textbf{Quesito \thequesito.\kern1ex}}{\vspace{\parskip}}
\newenvironment{xquestion}{\addtocounter{quesito}{1}\bigskip\bigskip\par\textbf{Quesito \thequesito.\kern1ex}}{\vspace{\parskip}}
\newenvironment{answer}{\par\textbf{Risposta\quad}}{\vspace{\parskip}}
\pagestyle{empty} 

\begin{document}
\colorbox{blue!10}{\begin{minipage}{\textwidth}
Domande per verificare la comprensione del significato di distribuzione continua (solo caso distribuzione normale). Richiede anche le nozioni di standardizzazione e di media campionaria.\medskip

N.B. Alcune domande potrebbero contenere informazioni irrilevanti.
\end{minipage}}
\bigskip\bigskip


\begin{pycode}
import random
random.seed('daxtxsdsxssme')
ESAME = False
\end{pycode}
%1
\begin{question}
\begin{pycode}
from scipy.stats import norm
mu = random.choice([6,7,8,9])
sigma = random.choice([3,5])
sqrt_n = random.choice([4,8])
n = sqrt_n**2
xbar =  random.choice([4,5,6,7,8])
conf = random.choice([90, 95, 99])
alpha = round(1-conf/100, 3)
risultato = round(- norm.ppf(alpha/2, n-1) * (sigma/sqrt_n), 4)
\end{pycode}
La variabile aleatoria $X$ ha distribuzione normale con deviazione standard $\sigma=\py{sigma}$ e media $\mu$ ignota.

Da un campione di rango $n=\py{n}$ otteniamo una media $\bar x=\py{xbar}$. Si stimi un intervallo di confienza al $\py{conf}\%$ per $\mu$.

Esprimere il risutato numerico tramite (solo) le funzioni elencate in calce.
\begin{answer}

% RISPOSTA 1

L'intervallo è $(\bar x-\varepsilon,\ \bar x+\varepsilon)$ dove

$\varepsilon\ =\ z_{\alpha/2}\cdot\dfrac{\sigma}{\sqrt{n}}$ 

$\alpha = \py{alpha}$

$z_{\alpha/2}$ \ è tale che \ $\alpha/2\ =\ \Pr(Z<-z_{\alpha/2})$

$z_{\alpha/2}\ =\ $\ {\tt - norm.ppf(\py{alpha/2})}

$\varepsilon\ =\ z_{\alpha/2}\cdot\dfrac{\sigma}{\sqrt{n}}\ =\ ${\color{blue}\tt - norm.ppf(\py{alpha/2}) $\cdot$ \py{sigma/sqrt_n}}

$\phantom{\varepsilon}$\ =\ {\tt \py{risultato}}
\end{answer}
\end{question}


%2
\begin{question} 
\begin{pycode}
from scipy.stats import t
mu = random.choice([6,7,8,9])
sd = random.choice([3,5])
sqrt_n = random.choice([4,8])
n = sqrt_n**2
xbar =  random.choice([4,5,6,7,8])
conf = random.choice([90, 95, 99])
alpha = round(1-conf/100, 2)
risultato = round(- t.ppf(alpha/2, n-1) * (sd/sqrt_n), 4)
\end{pycode}
La variabile aleatoria $X$ ha distribuzione normale con deviazione standard $\sigma$ e media $\mu$ ignote.

Da un campione di rango $\py{n}$ otteniamo una media $\bar x=\py{xbar}$ e un deviazione standard $s=\py{sd}$. Si stimi un intervallo di confienza al $\py{conf}\%$ per $\mu$.

Esprimere il risutato numerico tramite (solo) le funzioni elencate in calce.
\begin{answer}

% RISPOSTA 1

L'intervallo è $(\bar x-\varepsilon,\ \bar x+\varepsilon)$ dove

$\varepsilon\ =\ t_{\alpha/2}\cdot\dfrac{s}{\sqrt{n}}$ 

$\alpha = \py{alpha}$

$t_{\alpha/2}$ \ è tale che \ $\alpha/2\ =\ \Pr(T<-t_{\alpha/2})$ \ dove \ $T\sim t(n-1)$

$t_{\alpha/2}\ =\ $\ {\tt - t.ppf(\py{alpha/2}, \py{n-1})}

$\varepsilon\ =\ t_{\alpha/2}\cdot\dfrac{s}{\sqrt{n}}\ =\ ${\color{blue}\tt - t.ppf(\py{alpha/2}, \py{n-1}) $\cdot$ \py{sd/sqrt_n}}

$\phantom{\varepsilon}$\ =\ {\tt \py{risultato}}
\end{answer}
\end{question}








\vfill
\hrulefill

Si assuma noto il valore delle seguenti funzioni della libreria {\tt scipy.stats\/}

{\tt norm.cdf(z)} = $\Pr\big(Z<{\tt z}\big)$ per $Z\sim N(0,1)$ 

{\tt norm.ppf($\alpha$)} = $z_\alpha$ dove $z_\alpha$ è tale che $\Pr\big(Z<z_\alpha\big)=\alpha$ per $Z\sim N(0,1)$ 


\end{document}

