\documentclass[11pt,twoside,a4paper]{article}
\usepackage[T1]{fontenc}
\usepackage[utf8]{inputenc}
\usepackage[top=20mm, bottom= 15mm, left=15mm, right=15mm]{geometry}
\usepackage{amsmath}
\usepackage{dsfont}
\usepackage{calc} 
\usepackage{comment}
\usepackage{pythontex}
\newcommand{\mylabel}[1]{#1\hfill}
\renewenvironment{itemize}
  {\begin{list}{$\triangleright$}{%
   \setlength{\parskip}{0mm}
   \setlength{\topsep}{.4\baselineskip}
   \setlength{\rightmargin}{0mm}
   \setlength{\listparindent}{0mm}
   \setlength{\itemindent}{0mm}
   \setlength{\labelwidth}{2ex}
   \setlength{\itemsep}{.4\baselineskip}
   \setlength{\parsep}{0mm}
   \setlength{\partopsep}{0mm}
   \setlength{\labelsep}{1ex}
   \setlength{\leftmargin}{\labelwidth+\labelsep}
   \let\makelabel\mylabel}}{%
   \end{list}\vspace*{-1.3mm}}
\parindent0ex
\parskip1.5ex
\newcounter{quesito}
\newenvironment{question}{\addtocounter{quesito}{1}\par\textbf{Quesito \thequesito.\kern1ex}}{\vspace{0.5\parskip}}
\newenvironment{xquestion}{\bigskip\addtocounter{quesito}{1}\bigskip\bigskip\par\textbf{Quesito \thequesito.\kern1ex}}{\vspace{\parskip}}
\newenvironment{answer}{\par\textbf{Risposta\quad}}{\vspace{\parskip}}

\pagestyle{empty} 

\excludecomment{xquestion}
\excludecomment{answer}

\begin{document}
\colorbox{blue!10}{\begin{minipage}{\textwidth}
Matematica e BioStatistica con Applicazioni Informatiche\\
Esercitazione in aula del 20 novembre 2018
\end{minipage}}




\begin{question}
Calcolare un valore $\alpha$ tale che la seguente funzione è una corretta densità di probabilità

$\displaystyle\qquad f(x)\ =\ \left\{
\begin{array}{ll}
0& \textrm{se }x<1\\
\alpha\cdot x& \textrm{se } 1\le x\le 3\\
0& \textrm{se }x>3\\
\end{array}\right.$
\begin{answer}

$\quad\displaystyle\sum^{\infty}_{i=0}a^2_i
\ =\ \sum^{\infty}_{i=2}a^2_i+2^2+1^2
\ =\ 14+4+1={\color{blue}19}$.
\end{answer}
\end{question}



\begin{question}
Un urna contiene biglie il cui diametro è distribuito noralmente con media $30mm$ e deviazione standard $3$mm. Preleviamo $10$ biglie a caso dall'urna. 

\begin{itemize}
\item[1.] Quante biglie con diametro $\le 33$mm ci aspettiamo di ottenere? 
\item[2.] Qual'è la probabilità che almeno $7$ abbiano diametro $\le 33$mm?
\end{itemize}

Estraiamo casualmente biglie dall'urna fino a trovarne una di diametro compreso tra $29$ e $31$mm

\begin{itemize}
\item[1.] Quante estrazioni ci aspettiamo di dover fare? 
\item[2.] Qual'è la probabilità di trovarne una prima della quinta estrazione?
\end{itemize}

\begin{answer}
\end{answer}
\end{question}




\vfill\hrulefill\par
\begin{tabular}{@{}lll}
Formulario:& se $X\sim B({\tt n},{\tt p})$ & allora $E(X)=np$\\
           & se $X\sim NB({\tt n},{\tt p})$& allora $E(X)=n(1-p)/p$
\end{tabular}

Si assuma noto il valore delle seguenti funzioni della libreria {\tt scipy.stats\/} di  {\tt Python\/}\\
{\tt binom.pmf(k, n, p)} = $\Pr\big(X={\tt k}\big)$ dove $X\sim B({\tt n},{\tt p})$\\
{\tt binom.cdf(k, n, p)} = $\Pr\big(X\le{\tt k}\big)$ dove  $X\sim B({\tt n},{\tt p})$ \\
{\tt bimom.ppf(q, n, p)} = ${\tt k}$ dove ${\tt k}$ è tale che $\Pr\big(X\le{\tt k}\big)\cong{\tt q}$ per $X\sim B({\tt n},{\tt p})$ 

{\tt nbinom.xxx(k, n, p)}, è l'analogo per $X\sim NB({\tt n},{\tt p})$.

{\tt norm.xxx(k, n, p)}, è l'analogo per $X\sim N(\mu,\sigma)$.
\end{document}

