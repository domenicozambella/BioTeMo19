\documentclass[11pt,twoside,a4paper]{article}
\usepackage[T1]{fontenc}
\usepackage[utf8]{inputenc}
\usepackage[top=20mm, head=6mm, headsep=6mm, foot=6mm, bottom= 15mm, left=15mm, right=15mm]{geometry}
\usepackage{amsmath}
\usepackage{dsfont}
\usepackage{calc} 
\usepackage{comment}
\usepackage{pythontex}
\newcommand{\mylabel}[1]{#1\hfill}
\renewenvironment{itemize}
  {\begin{list}{$\triangleright$}{%
   \setlength{\parskip}{0mm}
   \setlength{\topsep}{.4\baselineskip}
   \setlength{\rightmargin}{0mm}
   \setlength{\listparindent}{0mm}
   \setlength{\itemindent}{0mm}
   \setlength{\labelwidth}{2ex}
   \setlength{\itemsep}{.4\baselineskip}
   \setlength{\parsep}{0mm}
   \setlength{\partopsep}{0mm}
   \setlength{\labelsep}{1ex}
   \setlength{\leftmargin}{\labelwidth+\labelsep}
   \let\makelabel\mylabel}}{%
   \end{list}\vspace*{-1.3mm}}
\parindent0ex
\parskip2ex
\newcounter{quesito}
\newenvironment{question}{\addtocounter{quesito}{1}\bigskip\bigskip\par\textbf{Quesito \thequesito.\kern1ex}}{\vspace{\parskip}}
\newenvironment{xquestion}{\bigskip\addtocounter{quesito}{1}\bigskip\bigskip\par\textbf{Quesito \thequesito.\kern1ex}}{\vspace{\parskip}}
\newenvironment{answer}{\par\textbf{Risposta\quad}}{\vspace{\parskip}}

\pagestyle{empty} 

\excludecomment{answer}

\begin{document}
\colorbox{blue!10}{\begin{minipage}{\textwidth}
Matematica e BioStatistica con Applicazioni Informatiche\\
Esercitazione in aula di martedì 16 ottobre 2018
\end{minipage}}

\bigskip


\begin{pycode}
import random
random.seed(258466445)
\end{pycode}


\begin{question}
\def\Pr{{\rm Pr\,}}
\def\Ex{{\rm E\,}}
\def\Var{{\rm Var\,}}
\begin{pycode}
from sympy import *
x = random.sample([1,2,3],2)
x1 = -x[0]
x2 = x[1]
x3 = x[0]
p1 = Rational(1,random.choice([2,3,4]))
p2 = Rational(1,random.choice([3,4]))
p3 = 1- p1 - p2
\end{pycode}
La v.a.\@ discreta $X$ ha distribuzione di probabilità 

\hfil$\displaystyle\Pr(X=\py{x1}) =\py{latex(p1)}$,\hfil  $\displaystyle\Pr(X=\py{x2}) =\py{latex(p2)}$,\hfil $\displaystyle\Pr(X=\py{x3}) =\py{latex(p3)}$. 

\begin{itemize}
\item[1.] Calcolare la distribuzione di probabilità di $X^2$
\item[2.] Calcolare $\Var(X)$. 
\end{itemize}

Esprimere i numeri razionali come frazioni.


\begin{answer}

{\color{blue}$\displaystyle\Pr(X^2=\py{x1**2})\ =\ \py{latex(p1+p3)}$ 
\ \ e \ \ 
$\displaystyle\Pr(X^2=\py{x2**2})\ =\ \py{latex(p2)}$\hfill Risposta 1\kern19ex} 

$\displaystyle\Ex(X) = \py{x1}\cdot\Pr(X=\py{x1})+\py{x2}\cdot\Pr(X=\py{x2})+\py{x3}\cdot\Pr(X=\py{x3})=\py{latex(x1*p1)} + \py{latex(x2*p2)} + \py{latex(x3*p3)}=\py{latex(x1*p1 + x2*p2 + x3*p3)}$

$\displaystyle\Ex(X^2) = \py{x1**2}\cdot\Pr(X^2=\py{x1**2})\ +\ \py{x2**2}\cdot\Pr(X^2=\py{x2**2})=\py{latex((x1**2)*(p1+p3))} + \py{latex((x2**2)*p2)} = \py{latex((x1**2)*(p1+p3)+ (x2**2)*p2)}$

$\displaystyle{\color{blue}\Var(X)}=\Ex(X^2)-\Ex(X)^2=\py{latex((x1**2)*(p1+p3)+ (x2**2)*p2)} - \py{latex((x1*p1 + x2*p2 + x3*p3)**2)}$ {\color{blue}$\ =\ \displaystyle\py{latex((x1**2)*(p1+p3)+ (x2**2)*p2 - (x1*p1 + x2*p2 + x3*p3)**2)} $\hfill Risposta 2\kern19ex} 
\end{answer}
\end{question}


\begin{question}
\def\RR{{\mathds R}}
\def\dom{{\rm dom}}
\def\range{{\rm im}}
\begin{pycode}
from sympy import *
x = symbols('x')
n =[Rational( i ) for i in random.sample([1,2,3,4,5],4) ]
g = ( n[0]*x - n[1] ) / ( n[2]*x + n[3] )
f = cos( abs( g ) )
\end{pycode}
Si consideri la funzione $\displaystyle f(x) = \py{latex(f)}$.
\begin{itemize}
\item[1.] Determinare dominio e immagine della funzione.
\item[2.] Determinare il punto di massimo assoluto per $x \geq 0$.
\end{itemize}
\begin{answer}

{\color{blue}
$\displaystyle \dom f \ =\ \RR\ \setminus\ \left\{\py{latex(-n[3]/n[2])}\right\} $
\quad e\quad 
$\range f \ =\ [-1, 1]$ 
\hfill Risposta 1\kern19ex}

{\color{blue}
$\displaystyle x\ =\ \py{latex(solve(g,x)[0])}$
\hfill Risposta 2\kern19ex}

\end{answer}
\end{question}





\begin{question}
\def\Pr{{\rm Pr\,}}
\begin{pycode}
rain_days = random.choice([5,6,7,8])
prev = rain_days * 100 / 365
sens =  random.choice([85, 90,95])
spec = random.choice([85, 90,95])
pos =  sens * prev + (100-spec) * (100-prev)
pos = pos / 100
ppv =  prev * sens  / pos
prev = round(prev, 1)
ppv = round(ppv, 1)
pos = round(pos, 1)
\end{pycode}
Marie is getting married tomorrow at an outdoor ceremony in the desert. In recent years it has rained only $\py{rain_days}$ days each year. But the weatherman has predicted rain for tomorrow. When it actually rains, the weatherman correctly forecasts rain $\py{sens}\%$ of the time. When it doesn’t rain, he incorrectly forecasts rain $\py{100-spec}\%$ of the times. What is the probability that it will rain on the day of Marie’s wedding?

\begin{answer}

$R$\hfill event: it rains on Marie’s wedding\kern22ex

$T_+$\hfill event: the weatherman predicts rain\kern22ex

$\Pr(R) =\py{rain_days}/365 = \py{prev}\%$\hfill it rains $\py{rain_days}$ days out of $365$\kern22ex

$\Pr(T_+|R) = \py{sens}\%$\hfill when it rains, rain is predicted\kern22ex

$\Pr(T_+|\neg R) = \py{100-spec}\%$\hfill when it does not rain,  rain is predicted\kern22ex

$\Pr(T_+) = \Pr(T_+|R)\cdot \Pr(R)+ \Pr(T_+|\neg R)\cdot \Pr(\neg R) = \py{pos}\%$ 

$\displaystyle \Pr(R\,| T_+)\ =\ \frac{\Pr(R)\cdot \Pr(T_+|R)}{\Pr(T_+)\vphantom{\Big[}}\ =\kern1ex${\color{blue}$\py{ppv}\%$\hfill Risposta\kern19ex}

\end{answer}
\end{question}





\begin{question}
\def\<div style="vertical-align: top;">
<div style="width:600px;height:200px;background-color:coral;display:inline;">
$T\cap H_0$ falso positivo / errore I tipo
<br/><br/>
$\Pr(T|H_0)=\alpha$ significatività
</div>
<div style="width:600px;height:300px;background-color:green!20%;display:inline;">
$T\cap H_A$ corretto positivo
<br/><br/>
$\Pr(T_+|H_A)=1-\beta$ sensibilità / potenza
</div>
</div>RR{{\mathds R}}
\def\dom{{\rm dom}}
\def\range{{\rm im}}
\begin{pycode}
from sympy import *
x = symbols('x')
n = [Rational( i ) for i in random.sample([2,3,4,5,6],1) ]
f = 1/(n[0]*x)
g = log(x)
\end{pycode}
Si considerino le funzioni $\displaystyle f(x) = \py{latex(f)}$ e $\displaystyle g(x) =\py{latex(g)}$.
\begin{itemize}
\item[1.] Scrivere esplicitamente le funzioni $f \circ g$ e $g \circ f$.
\item[2.] Determinare dominio di $f \circ g$ e $g \circ f$.
\end{itemize}
\begin{answer}

{\color{blue}
$\displaystyle (f \circ g) (x)\ =\ \py{latex(f.subs(x,g))}$
\qquad e\qquad 
$\displaystyle (g \circ f) (x)\ =\ \py{latex(g.subs(x,f))}$
\hfill Risposta 1\kern19ex}

\smallskip
{\color{blue}
$\dom (f \circ g)\ =\ (0, +\infty)$
\qquad e\qquad 
$\dom (g \circ f)\ =\ (0, 1) \cup (1, +\infty)$
\hfill Risposta 2\kern19ex}

\end{answer}
\end{question}




\begin{question}
\def\RR{{\mathds R}}
\def\E{{\rm E}}
\def\dom{{\rm dom}}
\def\range{{\rm im}}
If $20\%$ of the bolts produced by a machine are defective.
\begin{itemize}

\item[1.] Determine the probability that out of 4 bolts chosen at random less than 2, bolts will be defective.

\item[2.] Out of $2000$ bolts how many would you expect to be defective.
\end{itemize}
\begin{answer}

$p=0.2\quad n=2000\quad \E(X)=$
{\color{blue}
400
\hfill Risposta\kern19ex}
\end{answer}
\end{question}




\begin{question}
\def\Pr{{\rm Pr\,}}
\def\Ex{{\rm E\,}}
\def\Var{{\rm Var\,}}
\begin{pycode}
from sympy import *
x = random.sample([1,2,3,4,5,6],2)
y = random.choice([2,3])
mu = x[0]
var = x[1]
\end{pycode}
La v.a.\@ discreta $X$ ha valore atteso $\Ex(X)=\py{mu}$ e varianza $\Var(X)=\py{var}$. Qual è il valore atteso di $X(X-\py{y})$?  
\begin{answer}

$\displaystyle{\color{blue}\Ex\big(X(X-\py{y})\big)}=\Ex(X^2)-\py{y}\cdot\Ex(X)=\Var(X)+\Ex(X)^2-\py{y}\cdot\Ex(X)${\color{blue}$\ =\ \displaystyle\py{var+mu**2-y*mu}$\hfill Risposta} 
\end{answer}
\end{question}


\end{document}

