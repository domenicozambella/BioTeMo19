\documentclass[11pt,twoside,a4paper]{article}
\usepackage[T1]{fontenc}
\usepackage[utf8]{inputenc}
\usepackage[top=20mm, bottom= 15mm, left=15mm, right=15mm]{geometry}
\usepackage{amsmath}
\usepackage{dsfont}
\usepackage{calc} 
\usepackage{comment}
\usepackage{pythontex}
\newcommand{\mylabel}[1]{#1\hfill}
\renewenvironment{itemize}
  {\begin{list}{$\triangleright$}{%
   \setlength{\parskip}{0mm}
   \setlength{\topsep}{.4\baselineskip}
   \setlength{\rightmargin}{0mm}
   \setlength{\listparindent}{0mm}
   \setlength{\itemindent}{0mm}
   \setlength{\labelwidth}{2ex}
   \setlength{\itemsep}{.4\baselineskip}
   \setlength{\parsep}{0mm}
   \setlength{\partopsep}{0mm}
   \setlength{\labelsep}{1ex}
   \setlength{\leftmargin}{\labelwidth+\labelsep}
   \let\makelabel\mylabel}}{%
   \end{list}\vspace*{-1.3mm}}
\parindent0ex
\parskip1.5ex
\newcounter{quesito}
\newenvironment{question}{\addtocounter{quesito}{1}\par\textbf{Quesito \thequesito.\kern1ex}}{\vspace{0.5\parskip}}
\newenvironment{xquestion}{\bigskip\addtocounter{quesito}{1}\bigskip\bigskip\par\textbf{Quesito \thequesito.\kern1ex}}{\vspace{\parskip}}
\newenvironment{answer}{\par\textbf{Risposta\quad}}{\vspace{\parskip}}

\pagestyle{empty} 

\excludecomment{xquestion}
\excludecomment{answer}

\begin{document}
\colorbox{blue!10}{\begin{minipage}{\textwidth}
Matematica e BioStatistica con Applicazioni Informatiche\\
Esercitazione in aula del 6 dicembre 2018
\end{minipage}}



\begin{pycode}
import random
random.seed('daxtxsdsxssme')
ESAME = False
\end{pycode}


\bigskip\bigskip
\begin{question}
Una macchina è calibrata in modo da fare un taglio in un punto di altezza $\mu_0=20$. Se calibrata bene, l'altezza del taglio è distribuita normalmente con media $\mu_0$ deviazione standard $\sigma=2$. 

Ogni tanto (per effetto delle vibrazioni) la macchina si sposta, va quindi fermata e ricalibrata. Idealmente vorremmo fermare la macchina quando la nuova media $\mu$ differisce più di $3$ da $\mu_0$.

\begin{itemize}
\item[1.] Misuriamo quindi la posizione del taglio. Chiamiamo $\bar x$ la media fatta su un campione di $5$. Calibreremo la macchina se $|\mu_0-\bar x|$ è maggiore di un valore critico. Quale dev'essere questo valore per non fermare inutilmente ma macchina più del $10\%$ delle volte?

\item[2.] Dato il valore critico al punto 1, qual'è la probabilità di non ricalibrare una macchina che necessita di essere ricalibrata?

\item[3.] Dopo $500$ tagli la probabilità che $|\mu-\mu_0|>3$ è del $5\%$.  Su un campione di dimensione $5$ misuriamo una distanza media $\bar x=17$. Qual'è la probabilità che  $|\mu-\mu_0|$ sia davvero $>3$?

\item[4] alcune delle quantita calcolate nelle domande precedenti vengono generalmente denominate $\alpha$, $\beta$, $\delta$, e p-valore. Specificare quali.
\end{itemize}
\begin{answer}
$ =\ ${\tt {\color{blue}   }}\hfill{\color{blue} Risposta} 
\end{answer}
\end{question}


\vfill\hrulefill\par
\begin{tabular}{@{}lll}
Formulario:& se $X\sim B({\tt n},{\tt p})$ & allora $E(X)=np$\\
           & se $X\sim NB({\tt n},{\tt p})$& allora $E(X)=n(1-p)/p$
\end{tabular}

Si assuma noto il valore delle seguenti funzioni della libreria {\tt scipy.stats\/} di  {\tt Python\/}\\
{\tt binom.pmf(k, n, p)} = $\Pr\big(X={\tt k}\big)$ dove $X\sim B({\tt n},{\tt p})$\\
{\tt binom.cdf(k, n, p)} = $\Pr\big(X\le{\tt k}\big)$ dove  $X\sim B({\tt n},{\tt p})$ \\
{\tt bimom.ppf(q, n, p)} = ${\tt k}$ dove ${\tt k}$ è tale che $\Pr\big(X\le{\tt k}\big)\cong{\tt q}$ per $X\sim B({\tt n},{\tt p})$ 

{\tt nbinom.xxx(k, n, p)}, è l'analogo per $X\sim NB({\tt n},{\tt p})$.

{\tt norm.xxx(z)}, è l'analogo per $Z\sim N(0,1)$.
\end{document}


