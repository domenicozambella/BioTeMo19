\documentclass[11pt,twoside,a4paper]{article}
\usepackage[T1]{fontenc}
\usepackage[utf8]{inputenc}
\usepackage[top=20mm, bottom= 15mm, left=15mm, right=15mm]{geometry}
\usepackage{amsmath}
\usepackage{dsfont}
\usepackage{calc} 
\usepackage{comment}
\usepackage{pythontex}
\newcommand{\mylabel}[1]{#1\hfill}
\renewenvironment{itemize}
  {\begin{list}{$\triangleright$}{%
   \setlength{\parskip}{0mm}
   \setlength{\topsep}{.4\baselineskip}
   \setlength{\rightmargin}{0mm}
   \setlength{\listparindent}{0mm}
   \setlength{\itemindent}{0mm}
   \setlength{\labelwidth}{2ex}
   \setlength{\itemsep}{.4\baselineskip}
   \setlength{\parsep}{0mm}
   \setlength{\partopsep}{0mm}
   \setlength{\labelsep}{1ex}
   \setlength{\leftmargin}{\labelwidth+\labelsep}
   \let\makelabel\mylabel}}{%
   \end{list}\vspace*{-1.3mm}}
\parindent0ex
\parskip1.5ex
\newcounter{quesito}
\newenvironment{question}{\addtocounter{quesito}{1}\par\textbf{Quesito \thequesito.\kern1ex}}{\vspace{0.5\parskip}}
\newenvironment{xquestion}{\bigskip\addtocounter{quesito}{1}\bigskip\bigskip\par\textbf{Quesito \thequesito.\kern1ex}}{\vspace{\parskip}}
\newenvironment{answer}{\par\textbf{Risposta\quad}}{\vspace{\parskip}}

\pagestyle{empty} 

\excludecomment{xquestion}
\excludecomment{answer}

\begin{document}
\colorbox{blue!10}{\begin{minipage}{\textwidth}
Matematica e BioStatistica con Applicazioni Informatiche\\
Esercitazione in aula del 8 novembre 2018
\end{minipage}}

\bigskip

%10
\begin{xquestion}
\def\RR{{\mathds R}}
\def\dom{{\rm dom}}
\def\range{{\rm im}}
\begin{pycode}
from sympy import *
x = symbols('x')
n = [Rational( i ) for i in random.sample([1,2,3,4,5],1) ]
f = x**2 - n[0]
g = 1/x
\end{pycode}
Si considerino le funzioni $\displaystyle f(x) = \py{latex(f)}$ e $\displaystyle g(x) =\py{latex(g)}$.
\begin{itemize}
\item[1.] Scrivere esplicitamente le funzioni $f \circ g$ e $g \circ f$.
\item[2.] Determinare dominio $f \circ g$ e $g \circ f$.
\end{itemize}
\begin{answer}

{\color{blue}
$\displaystyle (f \circ g) (x)\ =\ \py{latex(f.subs(x,g))}$
\qquad e\qquad 
$\displaystyle (g \circ f) (x)\ =\ \py{latex(g.subs(x,f))}$
\hfill Risposta 1\kern0ex}

\smallskip
{\color{blue}
$\dom (f \circ g)\ =\ \RR - \{0\}$
\qquad e\qquad 
$\dom (g \circ f)\ =\ \RR - \{\pm \py{latex(sqrt(n[0]))}\}$
\hfill Risposta 2\kern0ex}

\end{answer}
\end{xquestion}


\begin{question}
La concentrazione di un farmaco nel sangue dopo $12$ ore è il $70\%$ della concentrazione iniziale. Vogliamo che la concentrazione massima a regime sia $4$. Somministriamo il farmaco giornalmente (ogni 24 ore). Di quanto deve aumentare la concentrazione ad ogni somministrazione? Ricordiamo che l'equazione $x_{n+1}=ax_n +b$ ha come soluzione generale $Ca^n+b/(1-a)$.
\begin{answer}

Conviene ragionare in unità di tempo giornaliere. Ogni giorno la concentrazione si riduce del $(70\%)^2=49\%$. Quindi $a=0.49$. Posto $4=b/(1-a)$ otteniamo $b=2.04$.

\end{answer}
\end{question}

\begin{question}
Al momento un paziente prende giornalmente $2$ unità al giorno di una medicina. La concentrazione massima nel sangue con questa dose è di $3.4$. Si rende necessario farla salire a $4.2$. Quanto dovrà essere la nuova dose? Ricordiamo che l'equazione $x_{n+1}=ax_n +b$ ha come soluzione generale $Ca^n+b/(1-a)$.
\begin{answer}

La concentrazione massima a regime è $b/(1-a)=2/(1-a)=3.4$. Quindi $a=0.41$. La nuova dose $b'$ deve soddisfare $b'/(1-a)=4.2$. Quindi $b'=2.48$
\end{answer}
\end{question}

\begin{question} Per i valori $-1<r<1$, la serie geometrica 

\hfil$\displaystyle \sum^\infty_{i=0}r^i$ 

converge a $1/(1-r)$. Per quali valori di $q$ la serie 

\hfil$\displaystyle \sum^\infty_{i=2}(1+q)^{i-1}$

converge? A cosa converge?
\begin{answer}

$\displaystyle \sum^\infty_{i=2}(1+q)^{i-1}=\sum^\infty_{i=1}(1+q)^{i}=\sum^\infty_{i=0}(1+q)^{i}-1=(1-q)/q -1$ per $2<q<0$

\end{answer}
\end{question}



\begin{question}
Due monetine, una con probabilità di dare testa $0.8$, l'altra $0.9$ vengono laciate simultaneamente. Qual è la probabilità che il primo lancio a cui differiscono sia $\ge 4$? 

Esprimere il risutato numerico tramite (solo) le funzioni elencate in calce.
\begin{answer}
Chiamiamo successo la divergenza tra le due monete. Questa si ottiene con probabilità $p=0.2\cdot0.9+0.8\cdot0.1$. Sia $Y\sim NB(1,p)$. La risposta è $\Pr(Y\ge 4)=1-\Pr(Y\le 5)$.
\end{answer}
\end{question}

\begin{question} Per i valori $-1<r<1$, la serie geometrica 

\hfil$\displaystyle \sum^\infty_{i=0}r^i$ 

converge a $1/(1-r)$. Per quali valori di $q$ la serie 

\hfil$\displaystyle \sum^\infty_{i=2}(1+q)^{i-1}$

converge? A cosa converge?
\begin{answer}

$\displaystyle \sum^\infty_{i=2}(1+q)^{i-1}=\sum^\infty_{i=1}(1+q)^{i}=\sum^\infty_{i=0}(1+q)^{i}-1=(1-q)/q -1$ per $2<q<0$

\end{answer}
\end{question}



\vfill
\hrulefill

Si assuma noto il valore delle seguenti funzioni della libreria {\tt scipy.stats\/} di  {\tt Python\/}

{\tt nbinom.pmf(k, 1, p)} = $\Pr\big(X={\tt k}\big)$ dove $X\sim NB({\tt 1},{\tt p})$ 

{\tt nbinom.cdf(k, 1, p)} = $\Pr\big(X\le{\tt k}\big)$ dove  $X\sim NB({\tt 1},{\tt p})$ 

{\tt nbimom.ppf(q, 1, p)} = ${\tt k}$ dove ${\tt k}$ è tale che $\Pr\big(X\le{\tt k}\big)\cong{\tt q}$ per $X\sim NB({\tt 1},{\tt p})$ 




\end{document}

