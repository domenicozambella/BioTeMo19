\documentclass[11pt,twoside,a4paper]{article}
\usepackage[T1]{fontenc}
\usepackage[utf8]{inputenc}
\usepackage[top=20mm, bottom= 15mm, left=15mm, right=15mm]{geometry}
\usepackage{amsmath}
\usepackage{dsfont}
\usepackage{calc} 
\usepackage{comment}
\usepackage{pythontex}
\newcommand{\mylabel}[1]{#1\hfill}
\renewenvironment{itemize}
  {\begin{list}{$\triangleright$}{%
   \setlength{\parskip}{0mm}
   \setlength{\topsep}{.4\baselineskip}
   \setlength{\rightmargin}{0mm}
   \setlength{\listparindent}{0mm}
   \setlength{\itemindent}{0mm}
   \setlength{\labelwidth}{2ex}
   \setlength{\itemsep}{.4\baselineskip}
   \setlength{\parsep}{0mm}
   \setlength{\partopsep}{0mm}
   \setlength{\labelsep}{1ex}
   \setlength{\leftmargin}{\labelwidth+\labelsep}
   \let\makelabel\mylabel}}{%
   \end{list}\vspace*{-1.3mm}}
\parindent0ex
\parskip1.5ex
\newcounter{quesito}
\newenvironment{question}{\addtocounter{quesito}{1}\par\textbf{Quesito \thequesito.\kern1ex}}{\vspace{0.5\parskip}}
\newenvironment{xquestion}{\bigskip\addtocounter{quesito}{1}\bigskip\bigskip\par\textbf{Quesito \thequesito.\kern1ex}}{\vspace{\parskip}}
\newenvironment{answer}{\par\textbf{Risposta\quad}}{\vspace{\parskip}}

\pagestyle{empty} 

\excludecomment{xquestion}
\excludecomment{answer}

\begin{document}
\colorbox{blue!10}{\begin{minipage}{\textwidth}
Matematica e BioStatistica con Applicazioni Informatiche\\
Esercitazione in aula del 11 dicembre 2018
\end{minipage}}



\begin{pycode}
import random
random.seed('daxtxsdsxssme')
ESAME = False
\end{pycode}

%7
\begin{question}
\begin{pycode}
from sympy import *
a = [Rational( i ) for i in random.sample([2,3,4,5],1) ]
b = [Rational( i ) for i in random.sample([-8,-7,-6,-5],1) ]
c = [Rational( i ) for i in random.sample([3,4,5],1) ]
d = [Rational( i ) for i in random.sample([2,3],1) ]
e = [Rational( i ) for i in random.sample([1,2,3,4],1) ]
x = symbols('x')
\end{pycode}
Si consideri la funzione definita a tratti 
\[f(x) = \begin{cases} \py{latex(a[0])} \quad 1 \leq x < \py{latex(c[0])} \cr \py{latex(b[0])} \quad \py{latex(c[0])} \leq x \leq 7 \end{cases}\]
\begin{itemize}
\item[1.] Determinare l'area (con segno) sottesa da tale funzione.
\item[2.] Determinare l'area (con segno) sottesa dalla funzione $f(\py{latex(d[0])}x + \py{latex(e[0])})$.
\end{itemize}
\begin{answer}

{\color{blue} Il valore dell'area è $\py{latex(a[0]*(c[0]-1)+b[0]*(7-c[0]))}$
\hfill Risposta 1\kern19ex}

\smallskip
{\color{blue} Il valore dell'area è $\py{latex(a[0]*(c[0]-1)/d[0]+b[0]*(7-c[0])/d[0])}$
\hfill Risposta 2\kern19ex}

\end{answer}
\end{question}
%9
\begin{question}
\begin{pycode}
from sympy import *
k = [Rational( i ) for i in random.sample([2,3,4,5,6],1) ]
m = [Rational( i ) for i in random.sample([2,3,4],1) ]
a = [Rational( i ) for i in random.sample([1,2,3,4,5],1) ]
x = symbols('x')
\end{pycode}
Si consideri la funzione $f(x) = x^{\py{latex(m[0])}} + \py{latex(k[0])}$
\begin{itemize}
\item[1.] Calcolare l'integrale indefinito $\displaystyle \int f(x) dx$.
\item[2.] Determinare l'area della parte di piano compresa tra la funzione $f$ e le due rette di equazioni $x = -1$ e $x = \py{latex(a[0])}$.
\end{itemize}
\begin{answer}

{\color{blue} $\cfrac{x^{\py{latex(m[0]+1)}}}{\py{latex(m[0]+1)}} + \py{latex(k[0])}x + C$. 
\hfill Risposta 1\kern19ex}

\smallskip
{\color{blue} Il valore dell'area è $\py{latex(a[0]**(m[0]+1)/(m[0]+1)+a[0]*k[0] - ((-1)**(m[0]+1)/(m[0]+1)-k[0]))}$.
\hfill Risposta 2\kern19ex}

\end{answer}
\end{question}



\begin{question}
\begin{pycode}
from scipy.stats import norm
N = random.choice([64, 81])
xbar = random.choice([2994])
conf = random.choice([90, 95, 99])
mu0 = random.choice([3300, 3350])
sigma = random.choice([800, 850])
i = sigma*N**(-1/2)
epsilon = norm.ppf(1- (100-conf)/200)*i
\end{pycode}
In 1996-97 there were $\py{N}$ cases of Sudden Infant Death Syndrome (SIDS) in King Co., WA. The average birthweight in this sample was $\py{xbar}$g. Based on nationwide surveys of millions of deliveries, the mean birthweight in the US is $\py{mu0}$g, with a standard deviation of $\py{sigma}$g. Suppose that this sample of $\py{N}$ babies is a random sample from the total population of SIDS cases. Find a $\py{conf}\%$ confidence interval for the population mean birthweight of SIDS cases in the US.
\begin{answer}
$\bar x = \py{xbar}$

$\sigma = \py{sigma}$

$n = \py{N}$ 

$\displaystyle\mu\ =\ \bar x\ \pm\ \varepsilon$

dove

$\varepsilon = 
{\tt norm.ppf(\py{1- (100-conf)/200})}\cdot\dfrac{\sigma}{\sqrt{n}}\ =\ {\color{blue}\tt norm.ppf(\py{1- (100-conf)/200})\cdot\py{round(i,2)}}\ =\ \py{round(epsilon,2)}${\color{blue}\hfill Risposta\kern19ex}


\end{answer}
\end{question}



\vfill\hrulefill\par
\begin{tabular}{@{}lll}
Formulario:& se $X\sim B({\tt n},{\tt p})$ & allora $E(X)=np$\\
           & se $X\sim NB({\tt n},{\tt p})$& allora $E(X)=n(1-p)/p$
\end{tabular}




Si assuma noto il valore delle seguenti funzioni della libreria {\tt scipy.stats\/} di  {\tt Python\/}\\
{\tt binom.pmf(k, n, p)} = $\Pr\big(X={\tt k}\big)$ dove $X\sim B({\tt n},{\tt p})$\\
{\tt binom.cdf(k, n, p)} = $\Pr\big(X\le{\tt k}\big)$ dove  $X\sim B({\tt n},{\tt p})$ \\
{\tt bimom.ppf(q, n, p)} = ${\tt k}$ dove ${\tt k}$ è tale che $\Pr\big(X\le{\tt k}\big)\cong{\tt q}$ per $X\sim B({\tt n},{\tt p})$ 

{\tt nbinom.xxx(k, n, p)}, è l'analogo per $X\sim NB({\tt n},{\tt p})$.

{\tt norm.xxx(z)}, è l'analogo per $Z\sim N(0,1)$.

{\tt t.xxx(t, $\nu$)}, è l'analogo per $T\sim t(\nu)$.
\end{document}


