\documentclass[11pt,twoside,a4paper]{article}
\usepackage[T1]{fontenc}
\usepackage[utf8]{inputenc}
\usepackage[top=20mm, head=6mm, headsep=6mm, foot=6mm, bottom= 15mm, left=15mm, right=15mm]{geometry}
\usepackage{amsmath}
\usepackage{dsfont}
\usepackage{calc} 
\usepackage{comment}
\usepackage{pythontex}
\newcommand{\mylabel}[1]{#1\hfill}
\renewenvironment{itemize}
  {\begin{list}{$\triangleright$}{%
   \setlength{\parskip}{0mm}
   \setlength{\topsep}{.4\baselineskip}
   \setlength{\rightmargin}{0mm}
   \setlength{\listparindent}{0mm}
   \setlength{\itemindent}{0mm}
   \setlength{\labelwidth}{2ex}
   \setlength{\itemsep}{.4\baselineskip}
   \setlength{\parsep}{0mm}
   \setlength{\partopsep}{0mm}
   \setlength{\labelsep}{1ex}
   \setlength{\leftmargin}{\labelwidth+\labelsep}
   \let\makelabel\mylabel}}{%
   \end{list}\vspace*{-1.3mm}}
\parindent0ex
\parskip2ex
\newcounter{quesito}
\newenvironment{question}{\addtocounter{quesito}{1}\bigskip\bigskip\par\textbf{Quesito \thequesito.\kern1ex}}{\vspace{\parskip}}
\newenvironment{xquestion}{\bigskip\addtocounter{quesito}{1}\bigskip\bigskip\par\textbf{Quesito \thequesito.\kern1ex}}{\vspace{\parskip}}
\newenvironment{answer}{\par\textbf{Risposta\quad}}{\vspace{\parskip}}

\pagestyle{empty} 

\excludecomment{answer}
\includecomment{answer}

\begin{document}
\colorbox{blue!10}{\begin{minipage}{\textwidth}
Matematica e BioStatistica con Applicazioni Informatiche\\
Esercitazione in aula del 25 ottobre 2018
\end{minipage}}

\bigskip


\begin{pycode}
import random
random.seed(258466445)
\end{pycode}


\begin{question}
\begin{pycode}
from scipy.stats import binom
n = random.choice([26,27,28,29])
k = random.choice([12,13,14,15])
t = random.choice(["t","u"])
risposta = round( 1- binom.cdf(k-1,n, 1/4), 4)
\end{pycode}
Consideriamo sequenze di $\py{n}$ caratteri dell'alfabeto $\{a,g,c,\py{t}\}$. Assumiamo che tutti i caratteri occorrano con la stessa probabilità indipendentemente dalla posizione. Qual è la probabilità che due sequenze coincidano in $\ge\py{k}$ posizioni?  

Esprimere il risutato numerico tramite (solo) le funzioni elencate in calce. 
\begin{answer}


$X\sim B(\py{n}, 1/4)$

$\Pr(X\ge \py{k})\ =\ 1-\Pr(X\le \py{k-1})\ =\ ${\tt{\color{blue}  1 -  binom.cdf( \py{k-1}, \py{n}, 1/4 )} = \py{risposta}}{\color{blue}\hfill Risposta}
\end{answer}
\end{question}


\vfill
\hrulefill

Si assuma noto il valore delle seguenti funzioni della libreria {\tt scipy.stats\/} di  {\tt Python\/}

{\tt binom.pmf(k,n,p)} = $\Pr\big(X={\tt k}\big)$ dove $X\sim B({\tt n},{\tt p})$ 

{\tt binom.cdf(k)} = $\Pr\big(X\le{\tt k}\big)$ dove  $X\sim B({\tt n},{\tt p})$ 

{\tt bimom.ppf($\alpha$, n, p)} = ${\tt x_\alpha}$ dove ${\tt x_\alpha}$ è tale che $\Pr\big(X\le{\tt x_\alpha}\big)=\alpha$ per $X\sim B({\tt n},{\tt p})$ 







\end{document}

