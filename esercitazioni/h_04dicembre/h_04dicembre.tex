\documentclass[11pt,twoside,a4paper]{article}
\usepackage[T1]{fontenc}
\usepackage[utf8]{inputenc}
\usepackage[top=20mm, bottom= 15mm, left=15mm, right=15mm]{geometry}
\usepackage{amsmath}
\usepackage{dsfont}
\usepackage{calc} 
\usepackage{comment}
\usepackage{pythontex}
\newcommand{\mylabel}[1]{#1\hfill}
\renewenvironment{itemize}
  {\begin{list}{$\triangleright$}{%
   \setlength{\parskip}{0mm}
   \setlength{\topsep}{.4\baselineskip}
   \setlength{\rightmargin}{0mm}
   \setlength{\listparindent}{0mm}
   \setlength{\itemindent}{0mm}
   \setlength{\labelwidth}{2ex}
   \setlength{\itemsep}{.4\baselineskip}
   \setlength{\parsep}{0mm}
   \setlength{\partopsep}{0mm}
   \setlength{\labelsep}{1ex}
   \setlength{\leftmargin}{\labelwidth+\labelsep}
   \let\makelabel\mylabel}}{%
   \end{list}\vspace*{-1.3mm}}
\parindent0ex
\parskip1.5ex
\newcounter{quesito}
\newenvironment{question}{\addtocounter{quesito}{1}\par\textbf{Quesito \thequesito.\kern1ex}}{\vspace{0.5\parskip}}
\newenvironment{xquestion}{\bigskip\addtocounter{quesito}{1}\bigskip\bigskip\par\textbf{Quesito \thequesito.\kern1ex}}{\vspace{\parskip}}
\newenvironment{answer}{\par\textbf{Risposta\quad}}{\vspace{\parskip}}

\pagestyle{empty} 

\excludecomment{xquestion}
\excludecomment{answer}

\begin{document}
\colorbox{blue!10}{\begin{minipage}{\textwidth}
Matematica e BioStatistica con Applicazioni Informatiche\\
Esercitazione in aula del 4 dicembre 2018
\end{minipage}}



\begin{pycode}
import random
random.seed('daxtxsdsxssme')
ESAME = False
\end{pycode}


\bigskip\bigskip
\begin{question}
Di una v.a.\@ $X\sim N(\mu,\sigma^2)$ con media ignota e deviazione standard $\sigma=2$ vogliamo stimare un intervallo di confidenza per $\mu$ di raggio $\varepsilon=1$ e livello di confidenza $95\%$. Quant'è la dimensione del campione necessaria?

Esprimere il risutato numerico tramite (solo) le funzioni elencate in calce.

\begin{answer}
$\dfrac{\varepsilon}{\sigma/\sqrt{n}}\ =\ ${\tt norm.ppf( $\alpha$/2 ) }

$\sqrt{n}\ =\ \dfrac{\sigma}{\varepsilon} \cdot\ ${\tt norm.ppf( $\alpha$/2 ) }

$n\ =\ ${\tt {\color{blue} ( 2 * norm.ppf(0.025) )**2} =  16}\hfill{\color{blue} Risposta}
\end{answer}
\end{question}




\bigskip\bigskip
\begin{question} %3
\begin{pycode}
from scipy.stats import norm
cosa = 'una data cultura'
errore = 'scartare'
mu    = random.choice( [6, 7, 8, 9] )
sigma = random.choice( [5, 7] )
n = 2
a = mu - 3 
b = a - 2
if ESAME:
    cosa = random.choice( ['suolo', 'acqua marina', 'gas di scarico', 'sangue'] )
    sigma = random.choice( [5,7] )
    n_campioni = random.choice( [15,16,17,18,19] )
    n = random.choice( [3,4] )
    a = mu - random.choice( [3,4,5,6] )
    b = a - random.choice( [1,2] )
    errore =  random.choice( ['accettare', 'scartare'] )
def pr(x):
    return "{0:.3f}".format(round(x,3))

if errore=='accettare':
    dis_eq=r'\le'
    risultato='norm.cdf'
    risultato_numerico = round( norm.cdf(n * (b-a)/sigma), 3)
elif errore=='scartare':
    dis_eq=r'\ge'
    risultato='1 - norm.cdf'
    risultato_numerico = round( 1 -  norm.cdf(n * (b-a)/sigma), 3)
\end{pycode}
Abbiamo prelevato vari campioni di \py{cosa}. Ci interessa selezionare quei campioni che hanno una concentrazione $\le\py{a}$ di una data sostanza. La misura produce risultati che differiscono dal valore corretto per un errore distribuito normalmente con media $0$ e deviazione standard $\py{sigma}$. Consideriamo la seguente procedura: se la media di $\py{n**2}$ misure è $\le\py{b}$ concludiamo che il campione è come desiderato altrimenti lo scartiamo.

Calcolare (nel caso più sfavorevole) la probabilità di \py{errore} erroneamente un campione.

Esprimere il risutato numerico tramite (solo) le funzioni elencate in calce.
\begin{answer}

Il caso più sfavorevole occorre quando la concentrazione vera nel campione è $\mu=\py{a}$ 


$\bar X\sim N(\mu,\sigma^2/n)$\hfill media di $\py{n**2}$ misure\kern22ex

$\displaystyle\Pr\big(\bar X\py{dis_eq}\py{b}\big)\ =\ \Pr\bigg(\frac{\bar X-\mu}{\sigma/\sqrt{\py{n**2}}}\py{dis_eq}\frac{\py{b}-\mu}{\sigma/\sqrt{\py{n**2}}}\bigg)\ =\  \Pr\big(Z\py{dis_eq}\py{n * (b-a) } /\py{sigma} \big)$

$\phantom{\Pr\big(\bar X\le\py{b}\big)}\ =\ ${\color{blue}{\tt\ \py{risultato}(\py{n * (b-a)}/\py{sigma})}}{\tt\ =\ \py{ risultato_numerico } }\hfill {\color{blue}\hfill Risposta}

\end{answer}
\end{question}



\vfill\hrulefill\par
\begin{tabular}{@{}lll}
Formulario:& se $X\sim B({\tt n},{\tt p})$ & allora $E(X)=np$\\
           & se $X\sim NB({\tt n},{\tt p})$& allora $E(X)=n(1-p)/p$
\end{tabular}

Si assuma noto il valore delle seguenti funzioni della libreria {\tt scipy.stats\/} di  {\tt Python\/}\\
{\tt binom.pmf(k, n, p)} = $\Pr\big(X={\tt k}\big)$ dove $X\sim B({\tt n},{\tt p})$\\
{\tt binom.cdf(k, n, p)} = $\Pr\big(X\le{\tt k}\big)$ dove  $X\sim B({\tt n},{\tt p})$ \\
{\tt bimom.ppf(q, n, p)} = ${\tt k}$ dove ${\tt k}$ è tale che $\Pr\big(X\le{\tt k}\big)\cong{\tt q}$ per $X\sim B({\tt n},{\tt p})$ 

{\tt nbinom.xxx(k, n, p)}, è l'analogo per $X\sim NB({\tt n},{\tt p})$.

{\tt norm.xxx(z)}, è l'analogo per $Z\sim N(0,1)$.
\end{document}


