\documentclass[11pt,twoside,a4paper]{article}
\usepackage[T1]{fontenc}
\usepackage[utf8]{inputenc}
\usepackage[top=20mm, bottom= 15mm, left=15mm, right=15mm]{geometry}
\usepackage{amsmath}
\usepackage{dsfont}
\usepackage{calc} 
\usepackage{comment}
\usepackage{pythontex}
\newcommand{\mylabel}[1]{#1\hfill}
\renewenvironment{itemize}
  {\begin{list}{$\triangleright$}{%
   \setlength{\parskip}{0mm}
   \setlength{\topsep}{.4\baselineskip}
   \setlength{\rightmargin}{0mm}
   \setlength{\listparindent}{0mm}
   \setlength{\itemindent}{0mm}
   \setlength{\labelwidth}{2ex}
   \setlength{\itemsep}{.4\baselineskip}
   \setlength{\parsep}{0mm}
   \setlength{\partopsep}{0mm}
   \setlength{\labelsep}{1ex}
   \setlength{\leftmargin}{\labelwidth+\labelsep}
   \let\makelabel\mylabel}}{%
   \end{list}\vspace*{-1.3mm}}
\parindent0ex
\parskip1.5ex
\newcounter{quesito}
\newenvironment{question}{\addtocounter{quesito}{1}\par\textbf{Quesito \thequesito.\kern1ex}}{\vspace{0.5\parskip}}
\newenvironment{xquestion}{\bigskip\addtocounter{quesito}{1}\bigskip\bigskip\par\textbf{Quesito \thequesito.\kern1ex}}{\vspace{\parskip}}
\newenvironment{answer}{\par\textbf{Risposta\quad}}{\vspace{\parskip}}

\pagestyle{empty} 

\excludecomment{xquestion}
\excludecomment{answer}

\begin{document}
\colorbox{blue!10}{\begin{minipage}{\textwidth}
Matematica e BioStatistica con Applicazioni Informatiche\\
Esercitazione in aula del 28 novembre 2018
\end{minipage}}



\begin{pycode}
import random
random.seed('daxtxsdsxssme')
ESAME = False
\end{pycode}


%4
\begin{question}
\begin{pycode}
from sympy import *
k = [Rational( i ) for i in random.sample([2,3,4,5,6],1) ]
h = [Rational( i ) for i in random.sample([1,2,3,4,5,6],1) ]
m = [Rational( i ) for i in random.sample([2,3],1) ]
x = symbols('x')
\end{pycode}
Si consideri la funzione $f(x) = (\py{latex(k[0])}x+\py{latex(h[0])})^{\py{latex(m[0])}}$.
\begin{itemize}
\item[1.] Calcolare l'integrale indefinito $\displaystyle \int f(x) dx$.
\item[2.] Determinare l'area (con segno) sottesa alla funzione $f$ nell'intervallo $[0,1]$.
\end{itemize}
\begin{answer}

{\color{blue}
$\displaystyle \int f(x) dx = \frac{(\py{latex(k[0])}x+\py{latex(h[0])})^{\py{latex(m[0]+1)}}}{\py{latex(k[0]*(m[0]+1))}} + C$.
\hfill Risposta 1\kern19ex}

\smallskip
{\color{blue} Il valore dell'area è $\py{latex((h[0]**(m[0]+1)-(h[0]+k[0])**(m[0]+1))/(k[0]*(m[0]+1)))}$.
\hfill Risposta 2\kern19ex}

\end{answer}
\end{question}






\begin{question}
Un campione di pazienti assume un certo farmaco. Ad ognuno di questo viene misurata

1. Glicemia

2. Colesterolo totale

3. Emoglobina

\ldots

\ldots ecc. ecc. (100 parametri fisiologici in totale)

I valori vengono confrontati con quelli di un campione di controllo (o con quelli in letteratura).

Per ognuna di queste misure consideriamo un test di ipotesi:

$H_{0,i}:$ il valore del parametro $i$ è uguale a quello del campione di controllo

$H_{A,i}:$ il valore del parametro $i$ è diverso a quello del campione di controllo

Assumiamo che tutte le ipotesi nulle siano vere e che questi 100 parametri si comportino come v.a. stocasticamente indipendenti. Quant'è la probabilità di rigettare (erroneamente) almeno $1$ ipotesi nulla con una significatività del $5\%$?

Quant'è la probabilità di rigettare almeno $5$ ipotesi nulla con una significatività del $5\%$?
\end{question}





\begin{question} %1
\begin{pycode}
from sympy import *
from scipy.stats import norm
mu = random.choice([6,7,8,9])
sigma = random.choice([3,5])
sqrt_n = random.choice([4,8])
n = sqrt_n**2
a = mu - random.choice([4,6])
b = mu + random.choice([1,2])
std_a = Rational( a-mu, sigma )
std_b = Rational( b-mu, sigma )

def pr(x):
    return "{0:.3f}".format(round(x,3))
\end{pycode}
La variabile aleatoria $X$ ha distribuzione normale con media $\mu=\py{mu}$ e deviazione standard $\sigma=\py{sigma}$ 

1. Calcolare la probabilità dell'evento $X\in [\py{a},\py{b}]$ 

2. Calcolare la probabilità che da un campione di rango $n=\py{n}$ si ottenga una media in $[\py{a},\py{b}]$. 

Esprimere il risutato numerico tramite (solo) le funzioni elencate in calce.
\begin{answer}

% RISPOSTA 1

$\displaystyle \Pr(\py{a}\le X\le\py{b})
\ =\ 
\Pr\bigg(\dfrac{\py{a}-\mu}{\sigma}\le Z\le\dfrac{\py{b}-\mu}{\sigma}\bigg)
\ =\  
\Pr\bigg(Z\le\py{latex(std_b)}\bigg) -  \Pr\bigg(Z\le\py{latex(std_a)}\bigg)$ 

$\phantom{\Pr\py{a}\le X\le\py{b})}
\ =\ 
${\color{blue}{\tt\ norm.cdf( \py{ std_b} ) -  norm.cdf( \py{ std_a} )} 
\hfill Risposta 1}

$\phantom{\Pr\py{a}\le X\le\py{b})}
\ =\ 
${\tt\ norm.cdf( \py{ float( std_b ) } ) -  norm.cdf( \py{ float( std_a ) } )
= 
\py{pr(norm.cdf( (b-mu)/sigma ) -  norm.cdf( (a-mu)/sigma ))}}

% RISPOSTA 2
$\bar X$ v.a.\@ media campionaria

$\displaystyle \Pr(\py{a}\le\bar X\le\py{b})
\ =\ 
\Pr\bigg(\dfrac{\py{a}-\mu}{\sigma/\sqrt{n}}\le Z\le\dfrac{\py{b}-\mu}{\sigma/\sqrt{n}}\bigg)\ =\  \Pr\bigg(Z\le\py{latex(sqrt_n*std_b)}\bigg) -  \Pr\bigg(Z\le\py{latex(sqrt_n*std_a)}\bigg)$

$\phantom{\Pr\py{a}\le\bar X\le\py{b})}
\ =\ 
${\color{blue}{\tt\ norm.cdf( \py{sqrt_n*std_a } ) -  norm.cdf( \py{ sqrt_n*std_a } ) }\hfill Risposta 2}

$\phantom{\Pr\py{a}\le\bar X\le\py{b})}
\ =\ 
${\tt\ norm.cdf( \py{float( sqrt_n*std_b)} ) -  norm.cdf( \py{float(sqrt_n*std_a) } )
= 
\py{pr(norm.cdf( sqrt_n*(b-mu)/sigma ) -  norm.cdf( sqrt_n*(a-mu)/sigma ))}}

\end{answer}
\end{question}














\begin{question} %5
\begin{pycode}
from sympy import *
import random
import math
from scipy.stats import norm
n = random.choice( [2,3, 4, 5] )
sigma = random.choice( [35, 45, 57] )
err = random.choice( [5, 10, 15] )
xbar = random.choice( [83, 98, 102] )

frazione =  Rational(n * err, sigma)
def pr(x):
    return "{0:.3f}".format(round(x,3))

\end{pycode}
Da una popolazione con distribuzione normale con media $\mu$ ignota e deviazione standard $\py{sigma}$ estraiamo un campione di $\py{n**2}$ individui. Qual è la probabilità che la media campionaria risulti $>\mu+ \py{err}$~? 
    
Esprimere il risutato numerico tramite (solo) le funzioni elencate in calce.

\begin{answer}
  
  $n = \py{n**2}$\hfill dimensione del campione\kern22ex
  
  $\varepsilon = \py{err}$ 
  
  $\sigma = \py{sigma}$\hfill deviazione standard della popolazione\kern22ex
  
  $\sigma/\sqrt{n}$\hfill errore standard (deviazione standard della media)\kern22ex
  
  $\displaystyle P\bigg(Z>\dfrac {\varepsilon}{\sigma/\sqrt{n}}\bigg)\ =\ P\bigg(Z>\py{latex(frazione)}\bigg)\ =\ ${\color{blue}\tt\ 1 - norm.cdf(\py{frazione})}{\tt\ = \py{pr(1 - norm.cdf( n * err/sigma ) ) } }{\color{blue}\hfill Risposta}

\end{answer}
\end{question}






























\vfill\hrulefill\par
\begin{tabular}{@{}lll}
Formulario:& se $X\sim B({\tt n},{\tt p})$ & allora $E(X)=np$\\
           & se $X\sim NB({\tt n},{\tt p})$& allora $E(X)=n(1-p)/p$
\end{tabular}

Si assuma noto il valore delle seguenti funzioni della libreria {\tt scipy.stats\/} di  {\tt Python\/}\\
{\tt binom.pmf(k, n, p)} = $\Pr\big(X={\tt k}\big)$ dove $X\sim B({\tt n},{\tt p})$\\
{\tt binom.cdf(k, n, p)} = $\Pr\big(X\le{\tt k}\big)$ dove  $X\sim B({\tt n},{\tt p})$ \\
{\tt bimom.ppf(q, n, p)} = ${\tt k}$ dove ${\tt k}$ è tale che $\Pr\big(X\le{\tt k}\big)\cong{\tt q}$ per $X\sim B({\tt n},{\tt p})$ 

{\tt nbinom.xxx(k, n, p)}, è l'analogo per $X\sim NB({\tt n},{\tt p})$.

{\tt norm.xxx(k, n, p)}, è l'analogo per $X\sim N(\mu,\sigma)$.
\end{document}

