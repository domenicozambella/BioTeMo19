\documentclass[11pt,twoside,a4paper]{article}
\usepackage[T1]{fontenc}
\usepackage[utf8]{inputenc}
\usepackage[top=20mm, bottom= 15mm, left=15mm, right=15mm]{geometry}
\usepackage{amsmath}
\usepackage{dsfont}
\usepackage{calc} 
\usepackage{comment}
\usepackage{pythontex}
\newcommand{\mylabel}[1]{#1\hfill}
\renewenvironment{itemize}
  {\begin{list}{$\triangleright$}{%
   \setlength{\parskip}{0mm}
   \setlength{\topsep}{.4\baselineskip}
   \setlength{\rightmargin}{0mm}
   \setlength{\listparindent}{0mm}
   \setlength{\itemindent}{0mm}
   \setlength{\labelwidth}{2ex}
   \setlength{\itemsep}{.4\baselineskip}
   \setlength{\parsep}{0mm}
   \setlength{\partopsep}{0mm}
   \setlength{\labelsep}{1ex}
   \setlength{\leftmargin}{\labelwidth+\labelsep}
   \let\makelabel\mylabel}}{%
   \end{list}\vspace*{-1.3mm}}
\parindent0ex
\parskip1.5ex
\newcounter{quesito}
\newenvironment{question}{\addtocounter{quesito}{1}\par\textbf{Quesito \thequesito.\kern1ex}}{\vspace{0.5\parskip}}
\newenvironment{xquestion}{\bigskip\addtocounter{quesito}{1}\bigskip\bigskip\par\textbf{Quesito \thequesito.\kern1ex}}{\vspace{\parskip}}
\newenvironment{answer}{\par\textbf{Risposta\quad}}{\vspace{\parskip}}

\pagestyle{empty} 

\excludecomment{xquestion}
\excludecomment{answer}

\begin{document}
\colorbox{blue!10}{\begin{minipage}{\textwidth}
Matematica e BioStatistica con Applicazioni Informatiche\\
Esercitazione in aula del 15 novembre 2018
\end{minipage}}


\begin{pycode}
import random
random.seed(258466445)
ESAME = False
\end{pycode}

\begin{question}
Sia $a_i$ una successione di cui sappiamo che $a_0=2$, $a_1=1$, $a_2=4$ e \ 
\smash{$\displaystyle\sum^{\infty}_{i=2}a^2_i
\ =\ 
14$}.

Calcolare 
$\quad\displaystyle\sum^{\infty}_{i=0}a^2_i$.

\begin{answer}

$\quad\displaystyle\sum^{\infty}_{i=0}a^2_i
\ =\ \sum^{\infty}_{i=2}a^2_i+2^2+1^2
\ =\ 14+4+1={\color{blue}19}$.
\end{answer}
\end{question}


\begin{question}
Sia $a_i$ una successione di cui sappiamo che $a_0=2$, $a_1=1$, $a_2=4$ e \ 
\smash{$\displaystyle\sum^{\infty}_{i=2} (a_i+a_{i+1})
\ =\ 
14$}.

Calcolare
$\quad\displaystyle\sum^{\infty}_{i=0}a_i$.

\begin{answer}

$\quad\displaystyle\sum^{\infty}_{i=2}(a_i+a_{i+1})
\ =\ 
\sum^{\infty}_{i=2}a_i+ \sum^{\infty}_{i=3}a_i
\ =\ 
2\left(\sum^{\infty}_{i=0}a_i\right) - 2a_0 -2a_1 -a_2$

Quindi

$\quad\displaystyle\sum^{\infty}_{i=0}a_i
\ =\ 
\dfrac{14+4+ 2 + 4}{2}
\ =\ 
{\color{blue}12}$.
\end{answer}
\end{question}

%4
\begin{question}
\begin{pycode}
from sympy import *
k = [Rational( i ) for i in random.sample([2,3,4,5,6],1) ]
h = [Rational( i ) for i in random.sample([1,2,3,4,5,6],1) ]
m = [Rational( i ) for i in random.sample([2,3],1) ]
x = symbols('x')
\end{pycode}
Si consideri la funzione $f(x) = (\py{latex(k[0])}x+\py{latex(h[0])})^{\py{latex(m[0])}}$.
\begin{itemize}
\item[1.] Calcolare l'integrale indefinito $\displaystyle \int f(x) dx$.
\item[2.] Determinare l'area (con segno) sottesa alla funzione $f$ nell'intervallo $[0,1]$.
\end{itemize}
\begin{answer}

{\color{blue}
$\displaystyle \int f(x) dx = \frac{(\py{latex(k[0])}x+\py{latex(h[0])})^{\py{latex(m[0]+1)}}}{\py{latex(k[0]*(m[0]+1))}} + C$.
\hfill Risposta 1\kern19ex}

\smallskip
{\color{blue} Il valore dell'area è $\py{latex((h[0]**(m[0]+1)-(h[0]+k[0])**(m[0]+1))/(k[0]*(m[0]+1)))}$.
\hfill Risposta 2\kern19ex}

\end{answer}
\end{question}


%9
\begin{question}
Si consideri la funzione $f(x) = x^3 + 1$
\begin{itemize}
\item[1.] Calcolare l'integrale indefinito $\displaystyle \int f(x) dx$.
\item[2.] Determinare l'area della parte di piano compresa tra la funzione $f$ e le due rette di equazioni $x = -1$ e $x = 3$.
\end{itemize}
\end{question}



\begin{question}
\def\Pr{{\rm Pr\,}}
\begin{pycode}
from numpy import arange
from scipy.stats import binom

p = random.choice([0.6, 0.7])

n_monete = random.choice([30, 35, 40])
n_sample = n_monete - 5

pH0 = random.choice([0.6, 0.8])
n_eq = n_monete * pH0
n_eq = int( n_eq )
n_di = n_monete - n_eq
k = n_sample * random.choice([0.6, 0.7])
k = int( k )
prev = n_di / n_monete


sens = sum( binom.pmf(arange(k,n_sample+1), n_sample, p) )
spec = sum( binom.pmf(arange(0,k), n_sample, 0.5) )

pos =  sens * prev + (1-spec) * (1-prev)
ppv =  prev * sens  / pos
pfn =  (1-spec) * (1-prev) 

sens = round(sens, 3)
spec = round(spec, 3)
prev = round(prev, 3)
ppv = round(ppv, 3)
pos = round(pos, 3)
pfn = round(pfn, 3)
\end{pycode}
Abbiamo $\py{n_eq+n_di}$ monete di cui $\py{n_eq}$ sono equilibrate, le altre sono difettose e hanno probabilità $\py{p}$ di dare come risultato {\tt Testa}. Scegliamo a caso una di queste $\py{n_eq+n_di}$ monete. Per decidere se è equilibrata o difettosa, la lanciamo $\py{n_sample}$ volte. Se otteniamo $\ge\py{k}$ volte {\tt Testa\/} diremo che è difettosa. Dei seguenti dati si usino quelli pertinenti

\begin{itemize}
\item[1.] Qual è la probabilità di dichiarare difettosa una moneta che non lo è? 

\item[2.] Qual è la probabilità che una moneta dichiarata difettosa lo sia veramente?
\end{itemize}


$\Pr(X\ge \py{k})\ =\ \py{1-spec}$\kern4ex se $X\sim B(\py{n_sample},0.5)$\kern13ex%
$=\ \py{round(1-binom.cdf(k-1, n_monete,0.5), 3)}$\hfill se $X\sim B(\py{n_monete},0.5)$\kern10ex

$\phantom{\Pr(X\ge \py{k})}\ =\ \py{sens}$\kern4ex se $X\sim B(\py{n_sample},\py{p})$\kern13ex%
$=\ \py{round(1-binom.cdf(k-1, n_monete,  p), 3)}$\hfill se $X\sim B(\py{n_monete},\py{p})$\kern10ex






\begin{answer}

$D$\hfill insieme degli esperimenti fatti con monete sbilanciate\kern22ex

$T_{\ge\py{k}}$\hfill insieme degli esperimenti con risultato $\ge\py{k}$\kern22ex

$\Pr(D)=\py{prev}$\hfill prevalenza del difetto\kern22ex

$\Pr(\neg D)=1-\Pr(D)=\py{1-prev}$

$\Pr(T_{\ge\py{k}}| \neg D)\ =\kern1ex${\color{blue} $\py{1-spec}$\hfill Risposta 1} 

$\Pr(T_{\ge\py{k}}| D)\ =\ \py{sens}$

$\Pr(T_{\ge\py{k}})=\Pr(T_{\ge\py{k}}|D)\ \Pr(D) + \Pr(T_{\ge\py{k}}|\neg D)\ \Pr(\neg D)\ =\ \py{pos}$

$\displaystyle\Pr(D ~|~T_{\ge\py{k}})\ =\ \frac{\Pr(T_{\ge\py{k}}| D)\ \Pr(D)}{\Pr(T_{\ge\py{k}})}\ =\kern1ex${\color{blue}$\py{ppv}$\hfill Risposta 2}

\end{answer}
\end{question}



\begin{question}
The CEO of a large electric utility claims that $80\%$ of his $1\,000\,000$ customers are very satisfied with the service they receive. To test this claim, the local newspaper surveyed $100$ customers, using simple random sampling. Among the sampled customers, $73$ say they are very satisified. 

\begin{itemize}
\item[1.] Based on these findings, compute the p-value of the following hypothesis test.

$H_0:\quad p = 0.80$,\qquad $H_1:\quad p \neq 0.80$\hfill  where $p$ is the proportion of very satisfied customers

\item[2.] A second survey is planned on $200$ customers. Assuming $H_0$, what is the probability the new p-value will be $\le$ of the p-value obtained before?
\end{itemize}
\begin{answer} Sia $X\sim B(100,0.8)$

$\Pr(|X-80|\ge|80-73|)
\ =\ 
\Pr(X\ge 87) + \Pr(X\le 73) $

$\phantom{\Pr(|X-80|\ge|80-73|)}
\ =\ $
{\color{blue}{\tt 1 - bimom.cdf(86, 100, 0.8) + bimom.cdf(72, 100, 0.8)}\hfill Risposta 1}

$\phantom{\Pr(|X-80|\ge|80-73|)}\llap{(Uguale a riposta 1.)\quad}
\ =\ $
{\color{blue}{\tt 1 - bimom.cdf(86, 100, 0.8) + bimom.cdf(72, 100, 0.8)} \hfill Risposta 2}


\end{answer}
\end{question}
\vfill
\hrulefill

\begin{tabular}{@{}lll}
Formulario:& se $X\sim B({\tt n},{\tt p})$ & allora $E(X)=np$\\
           & se $X\sim NB({\tt n},{\tt p})$& allora $E(X)=n(1-p)/p$
\end{tabular}

Si assuma noto il valore delle seguenti funzioni della libreria {\tt scipy.stats\/} di  {\tt Python\/}\\
{\tt nbinom.pmf(k, n, p)} = $\Pr\big(X={\tt k}\big)$ dove $X\sim NB({\tt n},{\tt p})$\\
{\tt nbinom.cdf(k, n, p)} = $\Pr\big(X\le{\tt k}\big)$ dove  $X\sim NB({\tt n},{\tt p})$ \\
{\tt nbimom.ppf(q, n, p)} = ${\tt k}$ dove ${\tt k}$ è tale che $\Pr\big(X\le{\tt k}\big)\cong{\tt q}$ per $X\sim NB({\tt n},{\tt p})$ 

{\tt binom.pmf(k, n, p)}, {\tt }, {\tt bimom.ppf(q, n, p)} sono l'analogo per $X\sim B({\tt n},{\tt p})$.
\end{document}

