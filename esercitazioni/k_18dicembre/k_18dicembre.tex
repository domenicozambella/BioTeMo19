\documentclass[11pt,twoside,a4paper]{article}
\usepackage[T1]{fontenc}
\usepackage[utf8]{inputenc}
\usepackage[top=20mm, bottom= 15mm, left=15mm, right=15mm]{geometry}
\usepackage{amsmath}
\usepackage{dsfont}
\usepackage{calc}
\usepackage{comment}
\usepackage{pythontex}
\newcommand{\mylabel}[1]{#1\hfill}
\renewenvironment{itemize}
  {\begin{list}{$\triangleright$}{%
   \setlength{\parskip}{0mm}
   \setlength{\topsep}{.4\baselineskip}
   \setlength{\rightmargin}{0mm}
   \setlength{\listparindent}{0mm}
   \setlength{\itemindent}{0mm}
   \setlength{\labelwidth}{2ex}
   \setlength{\itemsep}{.4\baselineskip}
   \setlength{\parsep}{0mm}
   \setlength{\partopsep}{0mm}
   \setlength{\labelsep}{1ex}
   \setlength{\leftmargin}{\labelwidth+\labelsep}
   \let\makelabel\mylabel}}{%
   \end{list}\vspace*{-1.3mm}}
\parindent0ex
\parskip1.5ex
\newcounter{quesito}
\newenvironment{question}{\addtocounter{quesito}{1}\par\textbf{Quesito \thequesito.\kern1ex}}{\vspace{0.5\parskip}}
\newenvironment{xquestion}{\bigskip\addtocounter{quesito}{1}\bigskip\bigskip\par\textbf{Quesito \thequesito.\kern1ex}}{\vspace{\parskip}}
\newenvironment{answer}{\par\textbf{Risposta\quad}}{\vspace{\parskip}}

\pagestyle{empty}
\excludecomment{xquestion}
\excludecomment{answer}

\begin{document}
\colorbox{blue!10}{\begin{minipage}{\textwidth}
Matematica e BioStatistica con Applicazioni Informatiche\\
Esercitazione in aula del 18 dicembre 2018
\end{minipage}}



\begin{pycode}
import random
random.seed('daxtxsdsxssme')
ESAME = False
\end{pycode}

%11
\begin{question}
\begin{pycode}
from sympy import *
k = [Rational( i ) for i in random.sample([2,3,4,5],1) ]
m = [Rational( i ) for i in random.sample([5,6,7],1) ]
n = [Rational( i ) for i in random.sample([2,3,4],1) ]
x = symbols('x')
\end{pycode}
Si considerino le funzioni $f(x) = \py{latex(m[0])} x$ e $g(x) = \py{latex(k[0])} x^3 + \py{latex(n[0])} x$
\begin{itemize}
\item[1.] Calcolare gli integrali indefiniti $\displaystyle \int f(x) dx$ e $\displaystyle \int g(x) dx$.
\item[2.] Determinare l'area della parte di piano compresa tra le funzioni $f$ e $g$.
\end{itemize}
\begin{answer}

{\color{blue} $\displaystyle \int f(x) dx = \cfrac{\py{latex(m[0])} x^2}{2} + C$, $\displaystyle \int g(x) dx = \cfrac{\py{latex(k[0])}x^4}{4} +\cfrac{\py{latex(n[0])} x^2}{2} + C$.,  \hfill Risposta 1\kern19ex}

\smallskip
{\color{blue} Il valore dell'area è $\py{latex(((m[0]-n[0])**2)/(2*k[0]))}$.
\hfill Risposta 2\kern19ex}

\end{answer}
\end{question}
%12
\begin{question}
Si considerino le funzioni $f(x) = x^2$ e $g(x) = -x^3 - x^2$
\begin{itemize}
\item[1.] Calcolare gli integrali indefiniti $\displaystyle \int f(x) dx$ e $\displaystyle \int g(x) dx$.
\item[2.] Determinare l'area della parte di piano compresa tra le due funzioni nell'intervallo $[-2, 0]$.
\end{itemize}
\end{question}
%13
\begin{question}
Si consideri la funzione $f(x) = \sqrt[3]{x}$.
\begin{itemize}
\item[1.] Scrivere l'approssimazione lineare di $f(x)$ in 1.
\item[2.] Usare il risultato precedente per approssimare i valori di $\sqrt[3]{1.1}$ e $\sqrt[3]{1.2}$.
\end{itemize}
\begin{answer}

L'approssimazione lineare di $f(x)$ in 1 è data da $f'(1)(x-1) + f(1)$, essendo $f'(x) = \frac{1}{3\sqrt[3]{x^2}}$ si ha
{\color{blue}
$\frac{1}{3}x+\frac{2}{3}$,  \hfill Risposta 1\kern19ex}

{\color{blue}
$\sqrt[3]{1.1} \cong 1.0\overline{3}$ e $\sqrt[3]{1.2} \cong 1.0\overline{6}$
\hfill Risposta 2\kern19ex}

\end{answer}
\end{question}

\begin{question}
In a packing plant, a machine packs cartons with jars. It is supposed that a new machine will pack faster on the average than the machine currently used. To test that hypothesis, the times it takes each machine to pack ten cartons are recorded. The results, in seconds, are shown in the following table.

$\begin{array}{llllllllllll}
\textrm{New }[x_i:i=1,\dots,10] &=\  [\,42.1,&  41.3,&  42.4,&  43.2,&  41.8,&  41.0,&  41.8,&  42.8,&  42.3,&  42.7\,]
\\[1ex]
\textrm{Old }\hfill [y_i:i=1,\dots,10] &=\ [\,42.7,& 43.8,& 42.5,& 43.1,& 44.0,& 43.6,& 43.3,& 43.5,& 41.7,& 44.1\,]
\end{array}$\smallskip

$\quad\bar x = 42.14$,\quad$s_x= 0.683$

$\quad\bar y = 43.23$,\quad$s_y=  0.750$

We want to kow if the data provide sufficient evidence to conclude that, 
on the average, the new machine packs faster. 
Which test is required? What is the p-value associated to the data?
\begin{answer}

\end{answer}
\end{question}

\vfill\hrulefill\par
\begin{tabular}{@{}lll}
Formulario:& se $X\sim B({\tt n},{\tt p})$ & allora $E(X)=np$\\
           & se $X\sim NB({\tt n},{\tt p})$& allora $E(X)=n(1-p)/p$\\
\end{tabular}
\\[1ex]
%dati $\bar X\sim N(\mu_x,\sigma/n_x)$, \ 
%$\bar Y\sim N(\mu_y,\sigma/n_y)$, \ 
$T=\dfrac{\bar X-\bar Y}{S\cdot\sqrt{1/n_x+1/n_y}}$
\hfill dove
$S^2\ =\ \dfrac{n_x-1}{n_x+n_y-2}\cdot S_x^2 + \dfrac{n_y-1}{n_x+n_y-2}\cdot S_y^2$
\hfill
ha distribuzione $t(n_x+n_y-2)$\\

Si assuma noto il valore delle seguenti funzioni della libreria {\tt scipy.stats\/} di  {\tt Python\/}\\
{\tt binom.pmf(k, n, p)} = $\Pr\big(X={\tt k}\big)$ dove $X\sim B({\tt n},{\tt p})$\\
{\tt binom.cdf(k, n, p)} = $\Pr\big(X\le{\tt k}\big)$ dove  $X\sim B({\tt n},{\tt p})$ \\
{\tt bimom.ppf(q, n, p)} = ${\tt k}$ dove ${\tt k}$ è tale che 
                           $\Pr\big(X\le{\tt k}\big)\cong{\tt q}$ per $X\sim B({\tt n},{\tt p})$\\
{\tt nbinom.xxx(\ldots)}, è l'analogo per $X\sim NB({\tt n},{\tt p})$.\\
{\tt norm.xxx(\ldots)}, è l'analogo per $Z\sim N(0,1)$.\\
\hfill{\tt t.xxx(\ldots, $\nu$)}, è l'analogo per $T\sim t(\nu)$.
\end{document}


