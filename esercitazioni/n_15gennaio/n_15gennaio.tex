\documentclass[11pt,twoside,a4paper]{article}
\usepackage[T1]{fontenc}
\usepackage[utf8]{inputenc}
\usepackage[top=20mm, bottom= 15mm, left=15mm, right=15mm]{geometry}
\usepackage{amsmath}
\usepackage{dsfont}
\usepackage{calc}
\usepackage{comment}
\usepackage{pythontex}
\usepackage{tikz}
\usetikzlibrary{automata,positioning}
\newcommand{\mylabel}[1]{#1\hfill}
\renewenvironment{itemize}
  {\begin{list}{$\triangleright$}{%
   \setlength{\parskip}{0mm}
   \setlength{\topsep}{.4\baselineskip}
   \setlength{\rightmargin}{0mm}
   \setlength{\listparindent}{0mm}
   \setlength{\itemindent}{0mm}
   \setlength{\labelwidth}{2ex}
   \setlength{\itemsep}{.4\baselineskip}
   \setlength{\parsep}{0mm}
   \setlength{\partopsep}{0mm}
   \setlength{\labelsep}{1ex}
   \setlength{\leftmargin}{\labelwidth+\labelsep}
   \let\makelabel\mylabel}}{%
   \end{list}\vspace*{-1.3mm}}
\parindent0ex
\parskip1.5ex
\newcounter{quesito}
\newenvironment{question}{\addtocounter{quesito}{1}\par\textbf{Quesito \thequesito.\kern1ex}}{\vspace{0.5\parskip}}
\newenvironment{xquestion}{\bigskip\addtocounter{quesito}{1}\bigskip\bigskip\par\textbf{Quesito \thequesito.\kern1ex}}{\vspace{\parskip}}
\newenvironment{answer}{\par\textbf{Risposta\quad}}{\vspace{\parskip}}

\pagestyle{empty}
\excludecomment{xquestion}
\excludecomment{answer}

\begin{document}
\colorbox{blue!10}{\begin{minipage}{\textwidth}
Matematica e BioStatistica con Applicazioni Informatiche\\
Esercitazione in aula del 15 gennaio 2018
\end{minipage}}



\begin{pycode}
import random
random.seed('daxtxsdsxssme')
ESAME = False
\end{pycode}

\begin{question}
\begin{pycode}
from scipy.stats import nbinom
n = random.choice([26,27,28,29])
k = random.choice([12,13,14,15])
t = random.choice(["t","u"])
risposta = round( 1- nbinom.cdf(k-1,n, 1/4), 4)
\end{pycode}
Consideriamo sequenze di $\py{n}$ caratteri dell'alfabeto $\{a,g,c,\py{t}\}$. 
Assumiamo che tutti i caratteri occorrano con la stessa probabilità indipendentemente dalla posizione.
Fissata una sequenza $s_0$, qual è la probabilità che un'altra sequenza $s_1$ 
scelta in modo indipendente coincida con $s_0$ in $\ge\py{k}$ posizioni?   

Esprimere il risutato numerico tramite (solo) le funzioni elencate in calce.
\begin{answer}
    
  $X\sim NB(\py{n}, 1/4)$
  
  $\Pr(X\ge \py{k})\ =\ 1-\Pr(X\le \py{k-1})$
  \quad =\ 
  {\tt{\color{blue}  1 -  nbinom.cdf( $\py{k-1}$, $\py{n}$, 1/4 )}
  \ =\ 
  \py{risposta}}{\color{blue}\hfill Risposta}
\end{answer}
\end{question}


\vfill\hrulefill\par
\begin{tabular}{@{}lll}
Formulario:& se $X\sim B({\tt n},{\tt p})$ & allora $E(X)=np$\\
           & se $X\sim NB({\tt n},{\tt p})$& allora $E(X)=n(1-p)/p$\\
\end{tabular}
\\[1ex]
%dati $\bar X\sim N(\mu_x,\sigma/n_x)$, \ 
%$\bar Y\sim N(\mu_y,\sigma/n_y)$, \ 
$T=\dfrac{\bar X-\bar Y}{S\cdot\sqrt{1/n_x+1/n_y}}$
\hfill dove
$S^2\ =\ \dfrac{n_x-1}{n_x+n_y-2}\cdot S_x^2 + \dfrac{n_y-1}{n_x+n_y-2}\cdot S_y^2$
\hfill
ha distribuzione $t(n_x+n_y-2)$\\

Si assuma noto il valore delle seguenti funzioni della libreria {\tt scipy.stats\/} di  {\tt Python\/}\\
{\tt binom.pmf(k, n, p)} = $\Pr\big(X={\tt k}\big)$ dove $X\sim B({\tt n},{\tt p})$\\
{\tt binom.cdf(k, n, p)} = $\Pr\big(X\le{\tt k}\big)$ dove  $X\sim B({\tt n},{\tt p})$ \\
{\tt bimom.ppf(q, n, p)} = ${\tt k}$ dove ${\tt k}$ è tale che 
                           $\Pr\big(X\le{\tt k}\big)\cong{\tt q}$ per $X\sim B({\tt n},{\tt p})$\\
{\tt nbinom.xxx(\ldots)}, è l'analogo per $X\sim NB({\tt n},{\tt p})$.\\
{\tt norm.xxx(\ldots)}, è l'analogo per $Z\sim N(0,1)$.\\
\hfill{\tt t.xxx(\ldots, $\nu$)}, è l'analogo per $T\sim t(\nu)$.
\end{document}


