\documentclass[11pt,twoside,a4paper]{article}
\usepackage[T1]{fontenc}
\usepackage[utf8]{inputenc}
\usepackage[top=20mm, bottom= 15mm, left=15mm, right=15mm]{geometry}
\usepackage{amsmath}
\usepackage{dsfont}
\usepackage{calc}
\usepackage{comment}
\usepackage{pythontex}
\usepackage{tikz}
\usetikzlibrary{automata,positioning}
\newcommand{\mylabel}[1]{#1\hfill}
\renewenvironment{itemize}
  {\begin{list}{$\triangleright$}{%
   \setlength{\parskip}{0mm}
   \setlength{\topsep}{.4\baselineskip}
   \setlength{\rightmargin}{0mm}
   \setlength{\listparindent}{0mm}
   \setlength{\itemindent}{0mm}
   \setlength{\labelwidth}{2ex}
   \setlength{\itemsep}{.4\baselineskip}
   \setlength{\parsep}{0mm}
   \setlength{\partopsep}{0mm}
   \setlength{\labelsep}{1ex}
   \setlength{\leftmargin}{\labelwidth+\labelsep}
   \let\makelabel\mylabel}}{%
   \end{list}\vspace*{-1.3mm}}
\parindent0ex
\parskip1.5ex
\newcounter{quesito}
\newenvironment{question}{\addtocounter{quesito}{1}\par\textbf{Quesito \thequesito.\kern1ex}}{\vspace{0.5\parskip}}
\newenvironment{xquestion}{\bigskip\addtocounter{quesito}{1}\bigskip\bigskip\par\textbf{Quesito \thequesito.\kern1ex}}{\vspace{\parskip}}
\newenvironment{answer}{\par\textbf{Risposta\quad}}{\vspace{\parskip}}

\pagestyle{empty}
\excludecomment{xquestion}
\excludecomment{answer}

\begin{document}
\colorbox{blue!10}{\begin{minipage}{\textwidth}
Matematica e BioStatistica con Applicazioni Informatiche\\
Esercitazione in aula del 15 gennaio 2018
\end{minipage}}



\begin{pycode}
import random
random.seed('daxtxsdsxssme')
ESAME = False
\end{pycode}

%2
\begin{question}
Si consideri il seguente problema di Cauchy:
\[\begin{cases} y' = x^3 y^2 \cr y(0) = 4 \end{cases}\]
\begin{itemize}
\item[1.] Trovare la soluzione del problema di Cauchy.
\item[2.] Determinare l'intervallo massimale di esistenza della soluzione.

\end{itemize}
\begin{answer}

{\color{blue}
La soluzione del problema di Cauchy \`e data dalla funzione $y(x) = \cfrac{4}{1-x^4}$.
\hfill Risposta 1\kern19ex}

\smallskip
{\color{blue} L'intervallo massimale \`e $(-1,1)$.
\hfill Risposta 2\kern19ex}

\end{answer}
\end{question}


%5
\begin{question}
\begin{pycode}
from sympy import *
x = symbols('x')
h = [Rational( i ) for i in random.sample([2,3,4],1) ]
\end{pycode}
Si consideri il seguente problema di Cauchy:
\[\begin{cases} y' =-xe^{-y} \cr y(0) = \py{latex(h[0])} \end{cases}\]
\begin{itemize}
\item[1.] Trovare la soluzione del problema di Cauchy.
\item[2.] Determinare l'intervallo massimale di esistenza della soluzione.

\end{itemize}
\begin{answer}

{\color{blue}
La soluzione del problema di Cauchy \`e data dalla funzione $y(x) = ln(-\cfrac{x^2}{2}+ e^{\py{latex(h[0])}})$.
\hfill Risposta 1\kern19ex}

\smallskip
{\color{blue} L'intervallo massimale \`e $(-\sqrt{2e^{\py{latex(h[0])}}}, \sqrt{2e^{\py{latex(h[0])}}})$.
\hfill Risposta 2\kern19ex}

\end{answer}
\end{question}


\begin{question} % 222222222222
\begin{pycode}
from sympy import Rational, Integer
from scipy.stats import norm
mu = random.choice( [ [1,4], [2,7], [2,8] ] )
n = random.choice( [3, 4, 5 ] ) # = sqrt(n)
sigma = random.choice( [2, 3] )
alpha = random.choice( [0.1, 0.05, 0.02, 0.01] )
zalpha = - norm.ppf(alpha) 
xalpha1 = round( mu[0] - zalpha*sigma / n, 3 )
xalpha2 = round( mu[0] + zalpha*sigma / n, 3 )
delta = random.choice( [0.5, 1, 1.5, 2] )
muA = mu[0] + delta
risposta_numerica1 = round( norm.cdf(n * (xalpha2-(mu[0]+delta) ) / sigma ), 4 )
risposta_numerica2 = 1 - risposta_numerica1
\end{pycode}
Preleviamo un campione di rango $n=\py{n**2}$ da una popolazione con distribuzione $N(\mu,\sigma^2)$. Sappiamo che la deviazione standard è $\sigma=\py{sigma}$. La media $\mu$ invece potrebbe avere uno qualsiasi valori dei nell'intervallo $[\py{mu[0]},\ \py{mu[1]}  ]$.

Vogliamo testare $H_0:\mu=\py{mu[0]}$ contro $H_A:\mu\in (\py{mu[0]},\py{mu[1]}]$. Fissiamo come significatività $\alpha=\py{alpha}$ otteniamo che per uno z-test a coda superiore la zona di rifiuto è $[\py{xalpha2}, \ +\infty)$.

\begin{itemize}
\item[1.] Nel caso $H_A:\mu\in [\py{muA},\ \py{mu[1]}]$ qual'è la massima probabilità $\beta$ di non rigettare $H_0$ (errore II tipo)?
\item[2.] Calcolare la potenza del test con l'effect-size suggerito nel punto precedente.
\end{itemize}

Esprimere il risutato numerico tramite (solo) le funzioni elencate in calce.

\begin{answer}

Il caso più sfavorevole si ottiene quando $\mu = \py{muA}$. Sia $\bar X\sim N(\py{muA},\ \sigma^2/n)$

$\beta
\ =\ 
\Pr\big(\bar X < \py{xalpha2} \big)
\ =\ 
\Pr\bigg( \dfrac{\bar X - \py{muA} }{\sigma/\sqrt{n}}\ <\ \dfrac{\py{xalpha2-muA} }{\sigma/\sqrt{n} }\bigg)
\ =\ 
\Pr\big( Z<  \py{round( n * (xalpha2-muA ) / sigma, 4) }\big)$


{\color{blue}$\beta
\ =\ $}
{\color{blue}\tt norm.cdf(\py{ round( n * (xalpha2-muA ) / sigma , 4) }) }{\tt\ =\ \py{risposta_numerica1} }{\color{blue}\hfill Risposta 1}

\medskip
Con un effect size $\delta=\py{delta}$ la potenza del test è $1-\beta=${\color{blue}\tt\ 1 - norm.cdf(\py{round( n * (xalpha2-muA ) / sigma , 4) }) }{\tt\ =\ \py{risposta_numerica2 } }{\color{blue}\hfill Risposta 2}

\end{answer}
\end{question}

  


\vfill\hrulefill\par
\begin{tabular}{@{}lll}
Formulario:& se $X\sim B({\tt n},{\tt p})$ & allora $E(X)=np$\\
           & se $X\sim NB({\tt n},{\tt p})$& allora $E(X)=n(1-p)/p$\\
\end{tabular}
\\[1ex]
%dati $\bar X\sim N(\mu_x,\sigma/n_x)$, \ 
%$\bar Y\sim N(\mu_y,\sigma/n_y)$, \ 
$T=\dfrac{\bar X-\bar Y}{S\cdot\sqrt{1/n_x+1/n_y}}$
\hfill dove
$S^2\ =\ \dfrac{n_x-1}{n_x+n_y-2}\cdot S_x^2 + \dfrac{n_y-1}{n_x+n_y-2}\cdot S_y^2$
\hfill
ha distribuzione $t(n_x+n_y-2)$\\

Si assuma noto il valore delle seguenti funzioni della libreria {\tt scipy.stats\/} di  {\tt Python\/}\\
{\tt binom.pmf(k, n, p)} = $\Pr\big(X={\tt k}\big)$ dove $X\sim B({\tt n},{\tt p})$\\
{\tt binom.cdf(k, n, p)} = $\Pr\big(X\le{\tt k}\big)$ dove  $X\sim B({\tt n},{\tt p})$ \\
{\tt bimom.ppf(q, n, p)} = ${\tt k}$ dove ${\tt k}$ è tale che 
                           $\Pr\big(X\le{\tt k}\big)\cong{\tt q}$ per $X\sim B({\tt n},{\tt p})$\\
{\tt nbinom.xxx(\ldots)}, è l'analogo per $X\sim NB({\tt n},{\tt p})$.\\
{\tt norm.xxx(\ldots)}, è l'analogo per $Z\sim N(0,1)$.\\
\hfill{\tt t.xxx(\ldots, $\nu$)}, è l'analogo per $T\sim t(\nu)$.
\end{document}


