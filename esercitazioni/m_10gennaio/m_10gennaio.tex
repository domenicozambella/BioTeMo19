\documentclass[11pt,twoside,a4paper]{article}
\usepackage[T1]{fontenc}
\usepackage[utf8]{inputenc}
\usepackage[top=20mm, bottom= 15mm, left=15mm, right=15mm]{geometry}
\usepackage{amsmath}
\usepackage{dsfont}
\usepackage{calc}
\usepackage{comment}
\usepackage{pythontex}
\usepackage{tikz}
\usetikzlibrary{automata,positioning}
\newcommand{\mylabel}[1]{#1\hfill}
\renewenvironment{itemize}
  {\begin{list}{$\triangleright$}{%
   \setlength{\parskip}{0mm}
   \setlength{\topsep}{.4\baselineskip}
   \setlength{\rightmargin}{0mm}
   \setlength{\listparindent}{0mm}
   \setlength{\itemindent}{0mm}
   \setlength{\labelwidth}{2ex}
   \setlength{\itemsep}{.4\baselineskip}
   \setlength{\parsep}{0mm}
   \setlength{\partopsep}{0mm}
   \setlength{\labelsep}{1ex}
   \setlength{\leftmargin}{\labelwidth+\labelsep}
   \let\makelabel\mylabel}}{%
   \end{list}\vspace*{-1.3mm}}
\parindent0ex
\parskip1.5ex
\newcounter{quesito}
\newenvironment{question}{\addtocounter{quesito}{1}\par\textbf{Quesito \thequesito.\kern1ex}}{\vspace{0.5\parskip}}
\newenvironment{xquestion}{\bigskip\addtocounter{quesito}{1}\bigskip\bigskip\par\textbf{Quesito \thequesito.\kern1ex}}{\vspace{\parskip}}
\newenvironment{answer}{\par\textbf{Risposta\quad}}{\vspace{\parskip}}

\pagestyle{empty}
\excludecomment{xquestion}
%\excludecomment{answer}

\begin{document}
\colorbox{blue!10}{\begin{minipage}{\textwidth}
Matematica e BioStatistica con Applicazioni Informatiche\\
Esercitazione in aula del 10 gennaio 2018
\end{minipage}}



\begin{pycode}
import random
random.seed('daxtxsdsxssme')
ESAME = False
\end{pycode}


%5
\begin{question}
\def\Pr{{\rm Pr\,}}
\def\Ex{{\rm E\,}}
\def\Var{{\rm Var\,}}
\begin{pycode}
from sympy import *
x = random.sample([2,3,4,5],4)
ax = x[0]
bx = x[1]
ay = x[2]
by = x[3]

p = random.sample([2,3,4,5],3)
d = random.choice(list(range(1,p[0])))
px = Rational(d,p[0])
d = random.choice(list(range(1,p[1])))
pay = Rational(d,p[1])
d = random.choice(list(range(1,p[1])))
pby = Rational(d,p[2])

\end{pycode}
Della v.a.\@ discreta $X$ conosciamo la distribizione di probabilità

\noindent\rlap{\kern15ex$\displaystyle\Pr(X=\py{ax}) =\py{latex(px)}$}\kern64ex   $\displaystyle\Pr(X=\py{bx}) =\py{latex(1-px)}$ 

Della v.a.\@ discreta $Y$ conosciamo la distribuzione condizionata a $X$

\noindent\rlap{\kern15ex$\displaystyle\Pr(Y=\py{ay}\ |\ X=\py{ax}) =\py{latex(pay)}$}\kern64ex     $\displaystyle\Pr(Y=\py{ay}\ |\ X=\py{bx}) =\py{latex(pby)}$ 

\noindent\rlap{\kern15ex$\displaystyle\Pr(Y=\py{by}\ |\ X=\py{ax}) =\py{latex(1-pay)}$}\kern64ex   $\displaystyle\Pr(Y=\py{by}\ |\ X=\py{bx}) =\py{latex(1-pby)}$ 

Calcolare la distribuzione di probablità di $Y$

Esprimere i numeri razionali come frazioni.
  
\begin{answer}
  
  {\color{blue}$\displaystyle\Pr(Y=\py{ay})$}$\displaystyle\ =\ \Pr(Y=\py{ay}\ |\ X=\py{ax})\cdot\Pr(X=\py{ax}) \ + \ \Pr(Y=\py{ay}\ |\ X=\py{bx})\cdot\Pr(X=\py{bx})${\color{blue}\ =\ \ $\displaystyle\py{latex(pay*px +  pby*(1-px)) }$}\hfill\smash{\raisebox{-\baselineskip}{\color{blue}$\Bigg\}$ \ \ Risposta}}
  
  {\color{blue}$\displaystyle\Pr(Y=\py{by})$}$\displaystyle\ =\ 1 - \Pr(Y=\py{ay})${\color{blue}\ =\ \ $\displaystyle\py{latex((1-pay)*px +  (1-pby)*(1-px)) }$}
  
  %{\color{blue}$\displaystyle\Pr(Y=\py{by})$}$\displaystyle\ =\ \Pr(Y=\py{by}\ |\ X=\py{ax})\cdot\Pr(X=\py{ax}) \ + \ \Pr(Y=\py{by}\ |\ X=\py{bx})\cdot\Pr(X=\py{bx})\ =\ \ ${\color{blue}$\displaystyle\py{latex((1-pay)*px +  (1-pby)*(1-px)) }$}
  
\end{answer}
\end{question}



\vfill\hrulefill\par
\begin{tabular}{@{}lll}
Formulario:& se $X\sim B({\tt n},{\tt p})$ & allora $E(X)=np$\\
           & se $X\sim NB({\tt n},{\tt p})$& allora $E(X)=n(1-p)/p$\\
\end{tabular}
\\[1ex]
%dati $\bar X\sim N(\mu_x,\sigma/n_x)$, \ 
%$\bar Y\sim N(\mu_y,\sigma/n_y)$, \ 
$T=\dfrac{\bar X-\bar Y}{S\cdot\sqrt{1/n_x+1/n_y}}$
\hfill dove
$S^2\ =\ \dfrac{n_x-1}{n_x+n_y-2}\cdot S_x^2 + \dfrac{n_y-1}{n_x+n_y-2}\cdot S_y^2$
\hfill
ha distribuzione $t(n_x+n_y-2)$\\

Si assuma noto il valore delle seguenti funzioni della libreria {\tt scipy.stats\/} di  {\tt Python\/}\\
{\tt binom.pmf(k, n, p)} = $\Pr\big(X={\tt k}\big)$ dove $X\sim B({\tt n},{\tt p})$\\
{\tt binom.cdf(k, n, p)} = $\Pr\big(X\le{\tt k}\big)$ dove  $X\sim B({\tt n},{\tt p})$ \\
{\tt bimom.ppf(q, n, p)} = ${\tt k}$ dove ${\tt k}$ è tale che 
                           $\Pr\big(X\le{\tt k}\big)\cong{\tt q}$ per $X\sim B({\tt n},{\tt p})$\\
{\tt nbinom.xxx(\ldots)}, è l'analogo per $X\sim NB({\tt n},{\tt p})$.\\
{\tt norm.xxx(\ldots)}, è l'analogo per $Z\sim N(0,1)$.\\
\hfill{\tt t.xxx(\ldots, $\nu$)}, è l'analogo per $T\sim t(\nu)$.
\end{document}


