\documentclass[11pt,twoside,a4paper]{article}
\usepackage[T1]{fontenc}
\usepackage[utf8]{inputenc}
\usepackage[top=20mm, bottom= 15mm, left=15mm, right=15mm]{geometry}
\usepackage{amsmath}
\usepackage{dsfont}
\usepackage{calc} 
\usepackage{comment}
\usepackage{pythontex}
\newcommand{\mylabel}[1]{#1\hfill}
\renewenvironment{itemize}
  {\begin{list}{$\triangleright$}{%
   \setlength{\parskip}{0mm}
   \setlength{\topsep}{.4\baselineskip}
   \setlength{\rightmargin}{0mm}
   \setlength{\listparindent}{0mm}
   \setlength{\itemindent}{0mm}
   \setlength{\labelwidth}{2ex}
   \setlength{\itemsep}{.4\baselineskip}
   \setlength{\parsep}{0mm}
   \setlength{\partopsep}{0mm}
   \setlength{\labelsep}{1ex}
   \setlength{\leftmargin}{\labelwidth+\labelsep}
   \let\makelabel\mylabel}}{%
   \end{list}\vspace*{-1.3mm}}
\parindent0ex
\parskip1.5ex
\newcounter{quesito}
\newenvironment{question}{\addtocounter{quesito}{1}\par\textbf{Quesito \thequesito.\kern1ex}}{\vspace{0.5\parskip}}
\newenvironment{xquestion}{\bigskip\addtocounter{quesito}{1}\bigskip\bigskip\par\textbf{Quesito \thequesito.\kern1ex}}{\vspace{\parskip}}
\newenvironment{answer}{\par\textbf{Risposta\quad}}{\vspace{\parskip}}

\pagestyle{empty} 

\excludecomment{xquestion}
\excludecomment{answer}

\begin{document}
\colorbox{blue!10}{\begin{minipage}{\textwidth}
Matematica e BioStatistica con Applicazioni Informatiche\\
Esercitazione in aula del 8 novembre 2018
\end{minipage}}

\begin{pycode}
import random
random.seed(258466445)
\end{pycode}

\begin{question}
Al momento un paziente prende giornalmente $2$ unità al giorno di una medicina. La concentrazione massima nel sangue con questa dose è di $3.4$. Si rende necessario farla salire a $4.2$. Quanto dovrà far aumentare la concentrazione la nuova dose giornaliera? Ricordiamo che l'equazione $x_{n+1}=ax_n +b$ ha come soluzione generale $Ca^n+b/(1-a)$.
\begin{answer}

La concentrazione massima a regime è $b/(1-a)=2/(1-a)=3.4$. Quindi $a=0.41$. La nuova dose $b'$ deve soddisfare $b'/(1-a)=4.2$. Quindi $b'=2.48$
\end{answer}
\end{question}

\end{document}

