\documentclass[11pt,twoside,a4paper]{article}
\usepackage[T1]{fontenc}
\usepackage[utf8]{inputenc}
\usepackage[top=20mm, head=6mm, headsep=6mm, foot=6mm, bottom= 15mm, left=15mm, right=15mm]{geometry}
\usepackage{amsmath}
\usepackage{calc} 
\usepackage{pythontex}
\newcommand{\mylabel}[1]{#1\hfill}
\renewenvironment{itemize}
  {\begin{list}{$\triangleright$}{%
   \setlength{\parskip}{0mm}
   \setlength{\topsep}{.4\baselineskip}
   \setlength{\rightmargin}{0mm}
   \setlength{\listparindent}{0mm}
   \setlength{\itemindent}{0mm}
   \setlength{\labelwidth}{2ex}
   \setlength{\itemsep}{.4\baselineskip}
   \setlength{\parsep}{0mm}
   \setlength{\partopsep}{0mm}
   \setlength{\labelsep}{1ex}
   \setlength{\leftmargin}{\labelwidth+\labelsep}
   \let\makelabel\mylabel}}{%
   \end{list}\vspace*{-1.3mm}}
\def\Ex{{\rm E\,}}
\def\Var{{\rm Var\,}}
\parindent0ex
\parskip2ex
\newcounter{quesito}
\newenvironment{question}{\bigskip\addtocounter{quesito}{1}\bigskip\bigskip\par\textbf{Quesito \thequesito.\kern1ex}}{\vspace{\parskip}}
\newenvironment{xquestion}{\bigskip\addtocounter{quesito}{1}\bigskip\bigskip\par\textbf{Quesito \thequesito.\kern1ex}}{\vspace{\parskip}}
\newenvironment{answer}{\par\textbf{Risposta\quad}}{\vspace{\parskip}}

\pagestyle{empty} 

\begin{document}
\colorbox{blue!10}{\begin{minipage}{\textwidth}
Domande  per verificare il riconoscimento di esperimenti che si modellano con distribuzione binomiale.
\end{minipage}}

\bigskip\bigskip


\begin{pycode}
import random
random.seed(258466445)
ESAME = False
\end{pycode}




\begin{question}
\def\Pr{{\rm Pr\,}}
\def\Ex{{\rm E\,}}
\def\Var{{\rm Var\,}}
\begin{pycode}
from scipy.stats import nbinom
p = random.choice([0.5,0.6])
n = random.choice([5, 6])
case = random.choice([10,15])
if not ESAME :
   p = 0.5
   n = 6
   case = 10
risultato = nbinom.cdf(case, n, p )
risultato = round(risultato, 6)
\end{pycode}
Pat is required to sell candy bars to raise money for the 6th grade field trip. There are \py{case} houses in the neighborhood, and Pat is not supposed to return home until \py{n} candy bars have been sold. So the child goes door to door, selling candy bars. At each house, there is a $\py{p}$ probability of selling one candy bar and a $\py{1-p}$ probability of selling nothing.
\begin{itemize}
\item[1.] What is the probability of selling the last candy bar at the $i$-th house? 
\item[2.] What is the probability of selling all 5 candies in the neighborhood? 
\end{itemize}

Esprimere il risutato numerico tramite (solo) le funzioni elencate in calce. 
\begin{answer}
Sia $Y\sim NB(\py{n},\py{p})$.

The probability of selling the last candy bar at the $i$-th house is 

$\Pr(Y+\py{n}=i)\ =\ ${\tt{\color{blue}\ nbinom.pmf(i-\py{n}, \py{n}, \py{p} ) } } {\color{blue}\hfill Risposta 1}

The probability of selling all \py{n} candies in the neighborhood is 

$\Pr(Y+\py{n}\le\py{case})\ =\ ${\tt{\color{blue}\ nbinom.cdf(\py{case-n}, \py{n}, \py{p} ) }= \py{risultato} } {\color{blue}\hfill Risposta 2}\par
\end{answer}
\end{question}





\begin{question}
\def\Pr{{\rm Pr\,}}
\def\Ex{{\rm E\,}}
\def\Var{{\rm Var\,}}
\begin{pycode}
from scipy.stats import nbinom
p  = random.choice([0.15,0.1])
ps = random.choice([0.99,0.995])
n = random.choice([8,9])
max = random.choice([11,12])
if not ESAME :
   p = 0.05
   n = 9
   max = 11
   ps = 0.99
risultato1 = round(n*p/(1-p) + n,3)
risultato2 = nbinom.cdf(max-n, n, 1-p )
risultato2 = round(risultato2, 6)
risultato3 = n + nbinom.ppf(ps, n, 1-p )
risultato3 = int(risultato3)
\end{pycode}
Abbiamo una scatola di viti di cui il $\py{int(100*p)}\%$ è difettoso. I nostri prodotti richiedono $\py{n}$ viti per pezzo. Procediamo nel seguente modo. Prendiamo una vite a caso e, se difettosa, la sostituiamo con un'altra, anche scelta a caso, fino ad avere le $\py{n}$ viti richieste.
\begin{itemize}
\item[1.] In media quante viti dovremmo provare per completare un pezzo?
\item[2.] Qual è la probabilità di riuscire a finire un pezzo scegliendo $\le\py{max}$ viti? 
\item[3.] Quanti tentativi dovremo preventivare perchè la probabilità di finire un pezzo sia almeno il $\py{int(100*ps)}\%$? 
\end{itemize}

Esprimere il risutato numerico tramite (solo) le funzioni elencate in calce. 
\begin{answer}
Consideriamo successo una vite non difettosa. Sia $Y\sim NB(\py{n},\py{1-p})$.

$\Ex(Y+\py{n})\ =\ \dfrac{\py{n}\cdot(1-\py{1-p})}{\py{1-p}}+\py{n}\ =\ ${\tt{\color{blue}\  \py{ risultato1} } } {\color{blue}\hfill Risposta 1}

$\Pr(Y+\py{n}\le\py{max})\ =\ ${\tt{\color{blue}\ nbinom.cdf(\py{max-n}, \py{n}, \py{1-p} ) }= \py{risultato2} } {\color{blue}\hfill Risposta 2}

\smallskip
$\Pr(Y+\py{n}\le x)\ =\ \py{ps}\ $ quindi {\tt{\color{blue}\ x $\cong$ \py{n} + nbinom.ppf(\py{ps}, \py{n}, \py{1-p} ) }= \py{risultato3} } {\color{blue}\hfill Risposta 3}\par

\end{answer}
\end{question}


\vfill
\hrulefill

\begin{tabular}{@{}llll}
Formulario:& se $X\sim B({\tt n},{\tt p})$ & allora $\Ex(X)=np$&e \ $\Var(X)=np(1-p)$\\
           & se $X\sim NB({\tt n},{\tt p})$& allora $E(X)=n(1-p)/p$&e \ $\Var(X)=n(1-p)/p^2$\\
\end{tabular}

Si assuma noto il valore delle seguenti funzioni della libreria {\tt scipy.stats\/} di  {\tt Python\/}\\
{\tt nbinom.pmf(k, n, p)} = $\Pr\big(X={\tt k}\big)$ dove $X\sim NB({\tt n},{\tt p})$\\
{\tt nbinom.cdf(k, n, p)} = $\Pr\big(X\le{\tt k}\big)$ dove  $X\sim NB({\tt n},{\tt p})$ \\
{\tt nbimom.ppf(q, n, p)} = ${\tt k}$ dove ${\tt k}$ è tale che $\Pr\big(X\le{\tt k}\big)\cong{\tt q}$ per $X\sim NB({\tt n},{\tt p})$ 


\end{document}
