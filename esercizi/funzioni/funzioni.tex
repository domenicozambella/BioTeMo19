\documentclass[11pt,twoside,a4paper]{article}
\usepackage[T1]{fontenc}
\usepackage[utf8]{inputenc}
\usepackage[top=20mm, head=6mm, headsep=6mm, foot=6mm, bottom= 15mm, left=15mm, right=15mm]{geometry}
\usepackage{amsmath}
\usepackage{calc} 
\usepackage{pythontex}
\usepackage{dsfont}
\newcommand{\mylabel}[1]{#1\hfill}
\renewenvironment{itemize}
  {\begin{list}{$\triangleright$}{%
   \setlength{\parskip}{0mm}
   \setlength{\topsep}{.4\baselineskip}
   \setlength{\rightmargin}{0mm}
   \setlength{\listparindent}{0mm}
   \setlength{\itemindent}{0mm}
   \setlength{\labelwidth}{2ex}
   \setlength{\itemsep}{.4\baselineskip}
   \setlength{\parsep}{0mm}
   \setlength{\partopsep}{0mm}
   \setlength{\labelsep}{1ex}
   \setlength{\leftmargin}{\labelwidth+\labelsep}
   \let\makelabel\mylabel}}{%
   \end{list}\vspace*{-1.3mm}}
\parindent0ex
\parskip2ex
\newcounter{quesito}
\newenvironment{question}{\bigskip\addtocounter{quesito}{1}\bigskip\bigskip\par\textbf{Quesito \thequesito.\kern1ex}}{\vspace{\parskip}}
\newenvironment{xquestion}{\bigskip\addtocounter{quesito}{1}\bigskip\bigskip\par\textbf{Quesito \thequesito.\kern1ex}}{\vspace{\parskip}}
\newenvironment{answer}{\par\textbf{Risposta\quad}}{\vspace{\parskip}}

\pagestyle{empty} 

\begin{document}


\begin{pycode}
import random
random.seed(2548445)
\end{pycode}
%1
\begin{question}
\def\RR{{\mathds R}}
\def\dom{{\rm dom}}
\def\range{{\rm im}}
\begin{pycode}
from sympy import *
x = symbols('x')
n =[Rational( i ) for i in random.sample([1,2,3,4,5],4) ]
g = ( n[0]*x - n[1] ) / ( n[2]*x + n[3] )
f = cos( abs( g ) )
\end{pycode}
Si consideri la funzione $\displaystyle f(x) = \py{latex(f)}$.
\begin{itemize}
\item[1.] Determinare dominio e immagine della funzione.
\item[2.] Determinare il punto di massimo assoluto per $x \geq 0$.
\end{itemize}
\begin{answer}

{\color{blue}
$\displaystyle \dom f \ =\ \RR\ \setminus\ \left\{\py{latex(-n[3]/n[2])}\right\} $
\quad e\quad 
$\range f \ =\ \RR$ 
\hfill Risposta 1\kern19ex}

{\color{blue}
$\displaystyle x\ =\ \py{latex(solve(g,x)[0])}$
\hfill Risposta 2\kern19ex}

\end{answer}
\end{question}
%2
\begin{question}
\def\RR{{\mathds R}}
\def\dom{{\rm dom}}
\def\range{{\rm im}}
\begin{pycode}
from sympy import *
x = symbols('x')
n =[Rational( i ) for i in random.sample([1,2,3,4,5],3) ]
f = exp( n[0]*x ) / ( n[1]*abs(x)-n[2] )
\end{pycode}
Si consideri la funzione $\displaystyle f(x) = \py{latex(f)} $.
\begin{itemize}
\item[1.] Determinare il dominio di $f$.
\item[2.] Determinare $\displaystyle f^{-1}\left(-\py{latex(1/n[2])}\right)$.
\end{itemize}
\begin{answer}

{\color{blue}
$\displaystyle  \dom f \ =\ \RR\ \setminus\ \left\{\pm\py{latex(n[2] / n[1])}\right\} $ 
\hfill Risposta 1\kern19ex}

{\color{blue}
$\displaystyle x\ =\ 0$
\hfill Risposta 2\kern19ex}

\end{answer}
\end{question}
%3
\begin{question}
\def\RR{{\mathds R}}
\def\dom{{\rm dom}}
\def\range{{\rm im}}
\begin{pycode}
from sympy import *
x = symbols('x')
n =[Rational( i ) for i in random.sample([1,2,3,4,5],1) ]
f = abs(x) - abs(x + n[0])
\end{pycode}
Si consideri la funzione $\displaystyle f(x) = \py{latex(f)}$.
\begin{itemize}
\item[1.] Determinare dominio e immagine della funzione.
\item[2.] Determinare $f^{-1}(\py{latex(n[0])})$.
\end{itemize}

\begin{answer}

$\displaystyle f(x) = \begin{cases} -\py{latex(n[0])}, \quad x \geq \py{latex(n[0])} \cr - \py{latex(n[0])} x + \py{latex(n[0])}, \quad - \py{latex(n[0])} \leq x < 0 \cr \py{latex(n[0])}, \quad x < -\py{latex(n[0])} \end{cases}$ 

{\color{blue} $\displaystyle \dom f \ =\ \RR$\quad e\quad $\displaystyle \range\ f \ =\ [-\py{latex(n[0])}, \py{latex(n[0])}]$
\hfill Risposta 1\kern19ex}

{\color{blue}
$\displaystyle f^{-1}(\py{latex(n[0])}) = -\py{latex(n[0])}$
\hfill Risposta 2\kern19ex}

\end{answer}
\end{question}

\begin{question}
\def\dom{{\rm dom}}
\def\range{{\rm im}}
\begin{pycode}
from sympy import *
x = symbols('x')
n = [Rational( i ) for i in random.sample([1,2,3,4,5],4) ]
f = (x + n[0])/(n[1]*x+n[2])
g = sqrt( n[3] -x )
\end{pycode}
Si considerino le funzioni $\displaystyle f(x) = \py{latex(f)}$ e $\displaystyle g(x) =\py{latex(g)}$
\begin{itemize}
\item[1.] Scrivere esplicitamente le funzioni $f \circ g$ e $g \circ f$.
\item[2.] Determinare dominio di $f \circ g$.
\end{itemize}
\begin{answer}

{\color{blue}
$\displaystyle (f \circ g) (x)\ =\ \py{latex(f.subs(x,g))}$
\qquad e\qquad 
$\displaystyle (g \circ f) (x)\ =\ \py{latex(g.subs(x,f))}$
\hfill Risposta 1\kern19ex}

\smallskip
{\color{blue}
$\dom (f \circ g)\ =\ (-\infty,\ \py{n[3]}]$
\hfill Risposta 2\kern19ex}

\end{answer}
\end{question}

\begin{question}
\def\RR{{\mathds R}}
\def\dom{{\rm dom}}
\def\range{{\rm im}}
\begin{pycode}
from sympy import *
e, x = symbols('e x')
n = [Rational( i ) for i in random.sample([1,2,3,4,5],3) ]
f = n[0] + log( n[1]*x+n[2] )
\end{pycode}
Si consideri la funzione $f(x) = \py{ latex(f) } )$.
\begin{itemize}
\item[1.] Determinare dominio e immagine della funzione. 
\item[2.] Per quali valori si annulla la funzione $f(-x)$?
\end{itemize}
Esprimere il risultato come frazione di interi, ed eventualmente multipli di $e$.
\begin{answer}

{\color{blue}
$\displaystyle\dom f\ =\ \left(\py{latex(-n[2]/n[1])},\ +\infty\right)$
\quad e\quad
$\range f\ =\ \RR$
\hfill Risposta 1\kern19ex}

{\color{blue}
$\displaystyle x\ =\ \py{latex(-n[1]/(n[2]*e**n[0]))}$
\hfill Risposta 2\kern19ex}

\end{answer}
\end{question}

\begin{question}
Si considerino le funzioni $f(x) = \cfrac{1}{x}$ e $g(x) = x^2 - 2x$.
\begin{itemize}
\item[1.] Scrivere esplicitamente le funzioni $f \circ g$ e $g \circ f$.
\item[2.] Determinare dominio di $f \circ g$ e $g \circ f$.
\end{itemize}
\begin{answer}

\end{answer}
\end{question}

\begin{question}
Si considerino le funzioni $f(x) = \cfrac{1}{x}$ e $g(x) = \log x$.
\begin{itemize}
\item[1.] Scrivere esplicitamente le funzioni $f \circ g$ e $g \circ f$.
\item[2.] Determinare dominio e immagine di $f \circ g$ e $g \circ f$.
\end{itemize}
\end{question}

\begin{question}
Si considerino le funzioni $f(x) = \sqrt{x}$ e $g(x) = \cfrac{x^2 + 1}{x^2 - 1}$.
\begin{itemize}
\item[1.] Scrivere esplicitamente le funzioni $f \circ g$ e $g \circ f$.
\item[2.] Determinare dominio e immagine di $f \circ g$ e $g \circ f$.
\end{itemize}
\end{question}

\begin{question}
Si consideri la funzione $f(x) = \bigg| \cfrac{1}{x} + 1 \bigg|$.
\begin{itemize}
\item[1.] Determinare dominio e immagine della funzione.
\item[2.] In quale punto si annulla la funzione $f(x+3)$?
\end{itemize}
\end{question}

\begin{question}
Si considerino le funzioni $f(x) = 2 + x^2$ e $g(x) = \cfrac{1}{x}$.
\begin{itemize}
\item[1.] Scrivere esplicitamente le funzioni $f \circ g$ e $g \circ f$.
\item[2.] Determinare dominio e immagine di $f \circ g$ e $g \circ f$.
\end{itemize}
\end{question}

\begin{question}
Sia $f(x)$ la funzione che misura in parti per milione (ppm) la concentrazione di anidride carbonica nell'atmosfera nell'anno $x$. Si supponga che tale concentrazione abbia una crescita lineare di fattore $m = 0.8$ ppm.
\begin{itemize}
\item[1.] In quanti anni la concentrazione di anidride carbonica aumenter\`a di 5 ppm?
\item[2.] Sapendo che nel 1960 la concentrazione di anidride carbonica nell'atmosfera era di 316 ppm, scrivere esplicitamente la funzione $f(x)$.
\end{itemize}
\end{question}

\begin{question}
Sia $f(x)$ la funzione che misura in parti per milione (ppm) la concentrazione di anidride carbonica nell'atmosfera nell'anno $x$. Si supponga che tale concentrazione abbia una crescita lineare di fattore $m = 0.8$ ppm.
\begin{itemize}
\item[1.] Sapendo che nel 1960 la concentrazione di anidride carbonica nell'atmosfera era di 316 ppm, scrivere esplicitamente la funzione $f(x)$.
\item[2.] Quale sar\`a la concentrazione di anidride carbonica nel 2025?
\end{itemize}
\end{question}

\begin{question}
Sia $f(x)$ la funzione che misura in parti per milione (ppm) la concentrazione di anidride carbonica nell'atmosfera nell'anno $x$. Si supponga che tale concentrazione abbia una crescita lineare di fattore $m = 0.8$ ppm.
\begin{itemize}
\item[1.] Sapendo che nel 2015 la concentrazione di anidride carbonica nell'atmosfera era di 360 ppm, scrivere esplicitamente la funzione $f(x)$.
\item[2.] Quale era la concentrazione di anidride carbonica nel 1990?
\end{itemize}
\end{question}

\begin{question}
Le funzioni trigonometriche $\sin$ e $\cos$ sono periodiche di periodo $2 \pi$.
\begin{itemize}
\item[1.] Qual \`e il periodo delle funzioni $\sin \left( x+3 \right)$ e $\cos \bigg( \cfrac{x+1}{3} \bigg)$?
\item[2.] Qual \`e il periodo della funzione $\tan \bigg( \cfrac{x}{2} \bigg)$?
\end{itemize}
\end{question}



\end{document}
