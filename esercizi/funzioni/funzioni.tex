\documentclass[11pt,twoside,a4paper]{article}
\usepackage[T1]{fontenc}
\usepackage[utf8]{inputenc}
\usepackage[top=20mm, head=6mm, headsep=6mm, foot=6mm, bottom= 15mm, left=15mm, right=15mm]{geometry}
\usepackage{amsmath}
\usepackage{calc} 
\usepackage{pythontex}
\usepackage{dsfont}
\newcommand{\mylabel}[1]{#1\hfill}
\renewenvironment{itemize}
  {\begin{list}{$\triangleright$}{%
   \setlength{\parskip}{0mm}
   \setlength{\topsep}{.4\baselineskip}
   \setlength{\rightmargin}{0mm}
   \setlength{\listparindent}{0mm}
   \setlength{\itemindent}{0mm}
   \setlength{\labelwidth}{2ex}
   \setlength{\itemsep}{.4\baselineskip}
   \setlength{\parsep}{0mm}
   \setlength{\partopsep}{0mm}
   \setlength{\labelsep}{1ex}
   \setlength{\leftmargin}{\labelwidth+\labelsep}
   \let\makelabel\mylabel}}{%
   \end{list}\vspace*{-1.3mm}}
\parindent0ex
\parskip2ex
\newcounter{quesito}
\newenvironment{question}{\bigskip\addtocounter{quesito}{1}\bigskip\bigskip\par\textbf{Quesito \thequesito.\kern1ex}}{\vspace{\parskip}}
\newenvironment{xquestion}{\bigskip\addtocounter{quesito}{1}\bigskip\bigskip\par\textbf{Quesito \thequesito.\kern1ex}}{\vspace{\parskip}}
\newenvironment{answer}{\par\textbf{Risposta\quad}}{\vspace{\parskip}}

\pagestyle{empty} 

\begin{document}


\begin{pycode}
import random
random.seed(2548445)
\end{pycode}

\begin{question}
Si consideri la funzione $f(x) = \cos \bigg|\cfrac{3x-1}{2x+2} \bigg|$.
\begin{itemize}
\item[1.] Determinare dominio e immagine della funzione.
\item[2.] Determinare il punto di massimo assoluto per $x \geq 0$
\end{itemize}
\end{question}

\begin{question}
Si consideri la funzione $f(x) = \cfrac{e^x}{\lvert x \rvert - 1}$.
\begin{itemize}
\item[1.] Determinare dominio e immagine della funzione.
\item[2.] Determinare $f^{-1}(-1)$.
\end{itemize}
\end{question}

\begin{question}
Si consideri la funzione $f(x) = \lvert x \rvert - \lvert x + 2 \rvert$.
\begin{itemize}
\item[1.] Determinare dominio e immagine della funzione.
\item[2.] Determinare $f^{-1}(2)$.
\end{itemize}
\end{question}

\begin{question}
\def\dom{{\rm dom}}
\def\range{{\rm im}}
\begin{pycode}
from sympy import *
x = symbols('x')
#f = Function('x')
n = random.sample([1,2,3,4,5],4)
f = (x + n[0])/(n[1]*x+n[2])
g = sqrt( n[3] -x )
\end{pycode}
Si considerino le funzioni $\displaystyle f(x) = \py{latex(f)}$ e $\displaystyle g(x) =\py{latex(g)}$
\begin{itemize}
\item[1.] Scrivere esplicitamente le funzioni $f \circ g$ e $g \circ f$.
\item[2.] Determinare dominio di $f \circ g$.
\end{itemize}
\begin{answer}

{\color{blue}$\displaystyle (f \circ g) (x)\ =\ \py{latex(f.subs(x,g))}$\qquad e\qquad $\displaystyle (g \circ f) (x)\ =\ \py{latex(g.subs(x,f))}$\hfill Risposta 1}

\smallskip
{\color{blue}$\dom (f \circ g)\ =\ (-\infty,\ \py{n[3]}]$\hfill Risposta 2}

\end{answer}
\end{question}

\begin{question}
\def\RR{{\mathds R}}
\begin{pycode}
from sympy import *
e, x = symbols('e x')
n = random.sample([1,2,3,4,5],3)
n0 = Rational(n[0])
n1 = Rational(n[1])
n2 = Rational(n[2])
\end{pycode}
Si consideri la funzione $f(x) = \py{n0} + \log(\py{ latex( n1*x+n2 ) } )$.
\begin{itemize}
\item[1.] Determinare dominio e immagine della funzione. 
\item[2.] Per quali valori si annulla la funzione $f(-x)$?
\end{itemize}
Esprimere il risultato come frazione di interi, ed eventualmente multipli di $e$.
\begin{answer}

\end{answer}

{\color{blue} Dominio=$\displaystyle\left(\py{latex(-n2/n1)},\ +\infty\right)$\qquad Immagine=$\RR$\hfill Risposta 1\kern19ex}

{\color{blue}$\displaystyle x=\py{latex(-n1/(n2*e**n0))}$\hfill Risposta 2\kern19ex}

\end{question}

\begin{question}
Si considerino le funzioni $f(x) = \cfrac{1}{x}$ e $g(x) = x^2 - 2x$.
\begin{itemize}
\item[1.] Scrivere esplicitamente le funzioni $f \circ g$ e $g \circ f$.
\item[2.] Determinare dominio e immagine di $f \circ g$ e $g \circ f$.
\end{itemize}
\end{question}

\begin{question}
Si considerino le funzioni $f(x) = \cfrac{1}{x}$ e $g(x) = \log x$.
\begin{itemize}
\item[1.] Scrivere esplicitamente le funzioni $f \circ g$ e $g \circ f$.
\item[2.] Determinare dominio e immagine di $f \circ g$ e $g \circ f$.
\end{itemize}
\end{question}

\begin{question}
Si considerino le funzioni $f(x) = \sqrt{x}$ e $g(x) = \cfrac{x^2 + 1}{x^2 - 1}$.
\begin{itemize}
\item[1.] Scrivere esplicitamente le funzioni $f \circ g$ e $g \circ f$.
\item[2.] Determinare dominio e immagine di $f \circ g$ e $g \circ f$.
\end{itemize}
\end{question}

\begin{question}
Si consideri la funzione $f(x) = \bigg| \cfrac{1}{x} + 1 \bigg|$.
\begin{itemize}
\item[1.] Determinare dominio e immagine della funzione.
\item[2.] In quale punto si annulla la funzione $f(x+3)$?
\end{itemize}
\end{question}

\begin{question}
Si considerino le funzioni $f(x) = 2 + x^2$ e $g(x) = \cfrac{1}{x}$.
\begin{itemize}
\item[1.] Scrivere esplicitamente le funzioni $f \circ g$ e $g \circ f$.
\item[2.] Determinare dominio e immagine di $f \circ g$ e $g \circ f$.
\end{itemize}
\end{question}

\begin{question}
Sia $f(x)$ la funzione che misura in parti per milione (ppm) la concentrazione di anidride carbonica nell'atmosfera nell'anno $x$. Si supponga che tale concentrazione abbia una crescita lineare di fattore $m = 0.8$ ppm.
\begin{itemize}
\item[1.] In quanti anni la concentrazione di anidride carbonica aumenter\`a di 5 ppm?
\item[2.] Sapendo che nel 1960 la concentrazione di anidride carbonica nell'atmosfera era di 316 ppm, scrivere esplicitamente la funzione $f(x)$.
\end{itemize}
\end{question}

\begin{question}
Sia $f(x)$ la funzione che misura in parti per milione (ppm) la concentrazione di anidride carbonica nell'atmosfera nell'anno $x$. Si supponga che tale concentrazione abbia una crescita lineare di fattore $m = 0.8$ ppm.
\begin{itemize}
\item[1.] Sapendo che nel 1960 la concentrazione di anidride carbonica nell'atmosfera era di 316 ppm, scrivere esplicitamente la funzione $f(x)$.
\item[2.] Quale sar\`a la concentrazione di anidride carbonica nel 2025?
\end{itemize}
\end{question}

\begin{question}
Sia $f(x)$ la funzione che misura in parti per milione (ppm) la concentrazione di anidride carbonica nell'atmosfera nell'anno $x$. Si supponga che tale concentrazione abbia una crescita lineare di fattore $m = 0.8$ ppm.
\begin{itemize}
\item[1.] Sapendo che nel 2015 la concentrazione di anidride carbonica nell'atmosfera era di 360 ppm, scrivere esplicitamente la funzione $f(x)$.
\item[2.] Quale era la concentrazione di anidride carbonica nel 1990?
\end{itemize}
\end{question}

\begin{question}
Le funzioni trigonometriche $\sin$ e $\cos$ sono periodiche di periodo $2 \pi$.
\begin{itemize}
\item[1.] Qual \`e il periodo delle funzioni $\sin \left( x+3 \right)$ e $\cos \bigg( \cfrac{x+1}{3} \bigg)$?
\item[2.] Qual \`e il periodo della funzione $\tan \bigg( \cfrac{x}{2} \bigg)$?
\end{itemize}
\end{question}

%\begin{question}
%\def\Pr{{\rm Pr\,}}
%\def\Ex{{\rm E\,}}
%\def\Var{{\rm Var\,}}
%\begin{pycode}
%from sympy import *
%x = random.sample([1,2,3],2)
%x1 = -x[0]
%x2 = x[1]
%x3 = x[0]
%p1 = Rational(1,random.choice([2,3,4]))
%p2 = Rational(1,random.choice([3,4]))
%p3 = 1- p1 - p2
%\end{pycode}
%La v.a.\@ discreta $X$ ha distribuzione di probabilità 
%
%\hfil$\displaystyle\Pr(X=\py{x1}) =\py{latex(p1)}$,\hfil  $\displaystyle\Pr(X=\py{x2}) =\py{latex(p2)}$,\hfil $\displaystyle\Pr(X=\py{x3}) =\py{latex(p3)}$. 
%
%\begin{itemize}
%\item[1.] Calcolare la distribuzione di probabilità di $X^2$
%\item[2.] Calcolare $\Var(X)$. 
%\end{itemize}
%
%Esprimere i numeri razionali come frazioni.
%
%
%\begin{answer}
%
%{\color{blue}$\displaystyle\Pr(X^2=\py{x1**2})\ =\ \py{latex(p1+p3)}$ 
%\ \ e \ \ 
%$\displaystyle\Pr(X^2=\py{x2**2})\ =\ \py{latex(p2)}$\hfill Risposta 1} 
%
%$\displaystyle\Ex(X) = \py{x1}\cdot\Pr(X=\py{x1})+\py{x2}\cdot\Pr(X=\py{x2})+\py{x3}\cdot\Pr(X=\py{x3})=\py{latex(x1*p1)} + \py{latex(x2*p2)} + \py{latex(x3*p3)}=\py{latex(x1*p1 + x2*p2 + x3*p3)}$
%
%$\displaystyle\Ex(X^2) = \py{x1**2}\cdot\Pr(X^2=\py{x1**2})\ +\ \py{x2**2}\cdot\Pr(X^2=\py{x2**2})=\py{latex((x1**2)*(p1+p3))} + \py{latex((x2**2)*p2)} = \py{latex((x1**2)*(p1+p3)+ (x2**2)*p2)}$
%
%$\displaystyle{\color{blue}\Var(X)}=\Ex(X^2)-\Ex(X)^2=\py{latex((x1**2)*(p1+p3)+ (x2**2)*p2)} - \py{latex((x1*p1 + x2*p2 + x3*p3)**2)}$ {\color{blue}$\ =\ \displaystyle\py{latex((x1**2)*(p1+p3)+ (x2**2)*p2 - (x1*p1 + x2*p2 + x3*p3)**2)} $\hfill Risposta 2} 
%\end{answer}
%\end{question}



\end{document}
