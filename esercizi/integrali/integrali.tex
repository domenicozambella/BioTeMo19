\documentclass[11pt,twoside,a4paper]{article}
\usepackage[T1]{fontenc}
\usepackage[utf8]{inputenc}
\usepackage[top=20mm, head=6mm, headsep=6mm, foot=6mm, bottom= 15mm, left=15mm, right=15mm]{geometry}
\usepackage{amsmath}
\usepackage{calc} 
\usepackage{pythontex}
\usepackage{dsfont}
\newcommand{\mylabel}[1]{#1\hfill}
\renewenvironment{itemize}
  {\begin{list}{$\triangleright$}{%
   \setlength{\parskip}{0mm}
   \setlength{\topsep}{.4\baselineskip}
   \setlength{\rightmargin}{0mm}
   \setlength{\listparindent}{0mm}
   \setlength{\itemindent}{0mm}
   \setlength{\labelwidth}{2ex}
   \setlength{\itemsep}{.4\baselineskip}
   \setlength{\parsep}{0mm}
   \setlength{\partopsep}{0mm}
   \setlength{\labelsep}{1ex}
   \setlength{\leftmargin}{\labelwidth+\labelsep}
   \let\makelabel\mylabel}}{%
   \end{list}\vspace*{-1.3mm}}
\parindent0ex
\parskip2ex
\newcounter{quesito}
\newenvironment{question}{\bigskip\addtocounter{quesito}{1}\bigskip\bigskip\par\textbf{Quesito \thequesito.\kern1ex}}{\vspace{\parskip}}
\newenvironment{xquestion}{\bigskip\addtocounter{quesito}{1}\bigskip\bigskip\par\textbf{Quesito \thequesito.\kern1ex}}{\vspace{\parskip}}
\newenvironment{answer}{\par\textbf{Risposta\quad}}{\vspace{\parskip}}

\pagestyle{empty} 

\begin{document}
\begin{pycode}
import random
random.seed(2548445)
\end{pycode}
%1
\begin{question}
\begin{pycode}
from sympy import *
x = symbols('x')
m = [Rational( i ) for i in random.sample([3,4],1) ]
n = [Rational( i ) for i in random.sample([5,7],1) ]
k = [Rational( i ) for i in random.sample([2,3,4,5,6],1) ]
\end{pycode}
Si consideri la funzione $f(x) = \sqrt[\py{latex(m[0])}]{x^{\py{latex(n[0])}}} - \py{latex(k[0])}\sin x$.
\begin{itemize}
\item[1.] Calcolare l'integrale indefinito $\displaystyle \int f(x) dx$.
\item[2.] Determinare l'area (con segno) sottesa alla funzione $f$ nell'intervallo $[0,1]$.
\end{itemize}
\begin{answer}

{\color{blue}
$\displaystyle \int f(x) dx = \frac{\py{latex(m[0])}}{\py{latex(m[0]+n[0])}}x^{\py{latex(m[0])}/\py{latex(m[0]+n[0])}} + \py{latex(k[0])}\cos x + C$.
\hfill Risposta 1\kern19ex}

\smallskip
{\color{blue} Il valore dell'area è
$\py{latex(k[0])}\cos 1 \py{latex(m[0]/(n[0]+m[0])-k[0])}$.
\hfill Risposta 2\kern19ex}

\end{answer}
\end{question}
%2
\begin{question}
\begin{pycode}
from sympy import *
a = [Rational( i ) for i in random.sample([2,4],2) ]
k = [Rational( i ) for i in random.sample([4,8],1) ]
x = symbols('x')
\end{pycode}
Si consideri la funzione $f(x) = \cos(\py{latex(k[0])}x)$.
\begin{itemize}
\item[1.] Calcolare l'integrale indefinito $\displaystyle \int f(x) dx$.
\item[2.] Determinare l'area (con segno) sottesa alla funzione $f$ nell'intervallo $[-\frac{\pi}{\py{latex(a[0])}},\frac{\pi}{\py{latex(a[1])}}]$.
\end{itemize}
\begin{answer}

{\color{blue}
$\displaystyle \int f(x) dx = \cfrac{\sin(\py{latex(k[0])}x)}{\py{latex(k[0])}} + C$.
\hfill Risposta 1\kern19ex}

\smallskip
{\color{blue} Il valore dell'area è 0.
\hfill Risposta 2\kern19ex}

\end{answer}
\end{question}
%3
\begin{question}
\begin{pycode}
from sympy import *
a = [Rational( i ) for i in random.sample([2,3,4,5,6,7],2) ]
x = symbols('x')
\end{pycode}
Si consideri la funzione $f(x) = e^{\py{latex(a[0])}x}$.
\begin{itemize}
\item[1.] Calcolare l'integrale indefinito $\displaystyle \int f(x) dx$.
\item[2.] Determinare l'area (con segno) sottesa alla funzione $f$ nell'intervallo $[0,\py{latex(a[1])}]$.
\end{itemize}
\begin{answer}

{\color{blue}
$\displaystyle \int f(x) dx = \cfrac{e^{\py{latex(a[0])}x}}{\py{latex(a[0])}} + C$.
\hfill Risposta 1\kern19ex}

\smallskip
{\color{blue} Il valore dell'area è} $e^{\py{latex(a[0]*a[1])}}/\py{latex(a[0])} - 1/\py{latex(a[0])}=$ {\color{blue}$(e^{\py{latex(a[0]*a[1])}}-1)/\py{latex(a[0])}$.
\hfill Risposta 2\kern19ex}

\end{answer}
\end{question}
%4
\begin{question}
\begin{pycode}
from sympy import *
k = [Rational( i ) for i in random.sample([2,3,4,5,6],1) ]
h = [Rational( i ) for i in random.sample([1,2,3,4,5,6],1) ]
m = [Rational( i ) for i in random.sample([2,3],1) ]
x = symbols('x')
\end{pycode}
Si consideri la funzione $f(x) = (\py{latex(k[0])}x+\py{latex(h[0])})^{\py{latex(m[0])}}$.
\begin{itemize}
\item[1.] Calcolare l'integrale indefinito $\displaystyle \int f(x) dx$.
\item[2.] Determinare l'area (con segno) sottesa alla funzione $f$ nell'intervallo $[0,1]$.
\end{itemize}
\begin{answer}

{\color{blue}
$\displaystyle \int f(x) dx = \frac{(\py{latex(k[0])}x+\py{latex(h[0])})^{\py{latex(m[0]+1)}}}{\py{latex(k[0]*(m[0]+1))}} + C$.
\hfill Risposta 1\kern19ex}

\smallskip
{\color{blue} Il valore dell'area è $\py{latex(-(h[0]**(m[0]+1)-(h[0]+k[0])**(m[0]+1))/(k[0]*(m[0]+1)))}$.
\hfill Risposta 2\kern19ex}

\end{answer}
\end{question}
%5
\begin{question}
\begin{pycode}
from sympy import *
k = [Rational( i ) for i in random.sample([2,3,4],1) ]
x = symbols('x')
\end{pycode}
Si consideri la funzione $f(x) = \py{latex(k[0])}x^2$ nell'intervallo $[0,4]$.
\begin{itemize}
\item[1.] Suddividere tale intervallo in $8$ parti e scrivere gli intervalli in cui \`e stato diviso. Calcolare la funzione $f$ nel punto medio di ciascuno di tali intervalli.
\item[2.] Calcolare la somma di Riemann della funzione $f$ relativa alla suddivisione e ai punti di campionamento trovati al punto precedente.
\end{itemize}
\begin{answer}

{\color{blue}
Gli intervalli sono $[0, 0.5], [0.5, 1], [1, 1.5], [1.5, 2], [2, 2.5], [2.5, 3], [3, 3.5], [3.5, 4]$.
Inoltre, $f(0.25) = \py{latex(k[0]*0.25**2)}$, $f(0.75) = \py{latex(k[0]*0.75**2)}$, $f(1.25) = \py{latex(k[0]*1.25**2)}$, $f(1.75) = \py{latex(k[0]*1.75**2)}$, $f(2.25) = \py{latex(k[0]*2.25**2)}$, $f(2.75) = \py{latex(k[0]*2.75**2)}$, $f(3.25) = \py{latex(k[0]*3.25**2)}$, $f(3.75) = \py{latex(k[0]*3.75**2)}$.
\hfill Risposta 1\kern19ex}

\smallskip
{\color{blue}La somma di Riemann vale $\py{latex(k[0]*(0.25**2+0.75**2+1.25**2+1.75**2+2.25**2+2.75**2+3.25**2+3.75))}$.
\hfill Risposta 2\kern19ex}

\end{answer}
\end{question}
%6
\begin{question}
\begin{pycode}
from sympy import *
k = [Rational( i ) for i in random.sample([3,4,5,6,7,8],1) ]
x = symbols('x')
\end{pycode}
Si consideri la funzione $f(x) = x^2 - \py{latex(k[0])}x$.
\begin{itemize}
\item[1.] Determinare l'area (con segno) sottesa da tale funzione nell'intervallo $[0, 10]$.
\item[2.] Determinare l'area (con segno) sottesa dalla funzione $\lvert f(x) \rvert$ nell'intervallo $[0, 10]$.
\end{itemize}
\begin{answer}

{\color{blue} L'area è} $\displaystyle  \int_0^{10} x^2 - \py{latex(k[0])}x dx = \left[ \frac{x^3}{3} - \py{latex(k[0]/2)} x^2 \right]_0^{10} =$ $\frac{1000}{3} - \py{latex(100*k[0]/2)} = $ 
{\color{blue}
$\py{latex(1000/3 - 100*k[0]/2)}$.
\hfill Risposta 1\kern19ex}

\smallskip
{\color{blue} L'area è} $\displaystyle \int_0^{10} |x^2 - \py{latex(k[0])}x| dx = \int_0^{\py{latex(k[0])}} -x^2 + \py{latex(k[0])}x dx + \int_{\py{latex(k[0])}}^{10} x^2 - \py{latex(k[0])}x dx = -\frac{2\py{latex(k[0])}^3}{3} + \py{latex(k[0])}^3 + \frac{1000}{3} -\frac{\py{latex(k[0])}}{2}\cdot 100 = $
{\color{blue} $\py{latex(-2*(k[0]**3)/3 + k[0]**3 + 1000/3 -k[0]*100/2)}$
\hfill Risposta 2\kern19ex}

\end{answer}
\end{question}
%7
\begin{question}
\begin{pycode}
from sympy import *
a = [Rational( i ) for i in random.sample([2,3,4,5],1) ]
b = [Rational( i ) for i in random.sample([-8,-7,-6,-5],1) ]
c = [Rational( i ) for i in random.sample([3,4,5],1) ]
d = [Rational( i ) for i in random.sample([2,3],1) ]
e = [Rational( i ) for i in random.sample([1,2,3,4],1) ]
x = symbols('x')
\end{pycode}
Si consideri la funzione definita a tratti 
\[f(x) = \begin{cases} \py{latex(a[0])} \quad 1 \leq x < \py{latex(c[0])} \cr \py{latex(b[0])} \quad \py{latex(c[0])} \leq x \leq 7 \end{cases}\]
\begin{itemize}
\item[1.] Determinare l'area (con segno) sottesa da tale funzione.
\item[2.] Determinare l'area (con segno) sottesa dalla funzione $f(\py{latex(d[0])}x + \py{latex(e[0])})$.
\end{itemize}
\begin{answer}

{\color{blue} Il valore dell'area è $\py{latex(a[0]*(c[0]-1)+b[0]*(7-c[0]))}$
\hfill Risposta 1\kern19ex}

\smallskip
{\color{blue} Il valore dell'area è $\py{latex(a[0]*(c[0]-1)/d[0]+b[0]*(7-c[0])/d[0])}$
\hfill Risposta 2\kern19ex}

\end{answer}
\end{question}
%8
\begin{question}
\begin{pycode}
from sympy import *
k = [Rational( i ) for i in random.sample([2,3,4,5,6],1) ]
a = [Rational( i ) for i in random.sample([1,2,3,4,5],1) ]
x = symbols('x')
\end{pycode}
Si consideri la funzione $f(x) = e^x - \py{latex(k[0])}$
\begin{itemize}
\item[1.] Calcolare l'integrale indefinito $\displaystyle \int f(x) dx$.
\item[2.] Determinare l'area della parte di piano compresa tra la funzione $f$ e le due rette di equazioni $x = 0$ e $x = \py{latex(a[0])}$.
\end{itemize}
\begin{answer}

{\color{blue} $e^x - \py{latex(k[0])} x + C$. 
\hfill Risposta 1\kern19ex}

\smallskip
{\color{blue} Il valore dell'area è $e^{\py{latex(a[0])}} - \py{latex(k[0]*a[0]-1)}$.
\hfill Risposta 2\kern19ex}

\end{answer}
\end{question}
%9
\begin{question}
\begin{pycode}
from sympy import *
k = [Rational( i ) for i in random.sample([2,3,4,5,6],1) ]
m = [Rational( i ) for i in random.sample([2,3,4],1) ]
a = [Rational( i ) for i in random.sample([1,2,3,4,5],1) ]
x = symbols('x')
\end{pycode}
Si consideri la funzione $f(x) = x^{\py{latex(m[0])}} + \py{latex(k[0])}$
\begin{itemize}
\item[1.] Calcolare l'integrale indefinito $\displaystyle \int f(x) dx$.
\item[2.] Determinare l'area della parte di piano compresa tra la funzione $f$ e le due rette di equazioni $x = -1$ e $x = \py{latex(a[0])}$.
\end{itemize}
\begin{answer}

{\color{blue} $\cfrac{x^{\py{latex(m[0]+1)}}}{\py{latex(m[0]+1)}} + \py{latex(k[0])}x + C$. 
\hfill Risposta 1\kern19ex}

\smallskip
{\color{blue} Il valore dell'area è $\py{latex(a[0]**(m[0]+1)/(m[0]+1)+a[0]*k[0] - ((-1)**(m[0]+1)/(m[0]+1)-k[0]))}$.
\hfill Risposta 2\kern19ex}

\end{answer}
\end{question}
%10
\begin{question}
\begin{pycode}
from sympy import *
k = [Rational( i ) for i in random.sample([2,3,4,5,6],1) ]
x = symbols('x')
\end{pycode}
Si consideri la funzione $f(x) = \py{latex(k[0])}\sin(x)$
\begin{itemize}
\item[1.] Calcolare l'integrale indefinito $\displaystyle \int f(x) dx$.
\item[2.] Determinare l'area della parte di piano compresa tra la funzione $f$ e le due rette di equazioni $x = \pi/2$ e $x = 2 \pi$.
\end{itemize}
\begin{answer}

{\color{blue} L'integrale indefinito è $-\py{latex(k[0])} \cos x + C$. 
\hfill Risposta 1\kern19ex}

\smallskip
{\color{blue} Il valore dell'area è $\py{latex(3*k[0])}$.
\hfill Risposta 2\kern19ex}

\end{answer}
\end{question}
%11
\begin{question}
\begin{pycode}
from sympy import *
k = [Rational( i ) for i in random.sample([2,3,4,5],1) ]
m = [Rational( i ) for i in random.sample([5,6,7],1) ]
n = [Rational( i ) for i in random.sample([2,3,4],1) ]
x = symbols('x')
\end{pycode}
Si considerino le funzioni $f(x) = \py{latex(m[0])} x$ e $g(x) = \py{latex(k[0])} x^3 + \py{latex(n[0])} x$
\begin{itemize}
\item[1.] Calcolare gli integrali indefiniti $\displaystyle \int f(x) dx$ e $\displaystyle \int g(x) dx$.
\item[2.] Determinare l'area della parte di piano compresa tra le funzioni $f$ e $g$.
\end{itemize}
\begin{answer}

{\color{blue} $\displaystyle \int f(x) dx = \cfrac{\py{latex(m[0])} x^2}{2} + C$, $\displaystyle \int g(x) dx = \cfrac{\py{latex(k[0])}x^4}{4} +\cfrac{\py{latex(n[0])} x^2}{2} + C$. 
\hfill Risposta 1\kern19ex}

\smallskip
{\color{blue} Il valore dell'area è $\py{latex(((m[0]-n[0])**2)/(2*k[0]))}$.
\hfill Risposta 2\kern19ex}

\end{answer}
\end{question}
%12
\begin{question}
\begin{pycode}
from sympy import *
k = [Rational( i ) for i in random.sample([0,1,2,3],1) ]
x = symbols('x')
\end{pycode}
Si considerino le funzioni $f(x) = \py{latex((x+k[0])**2)}$ e $g(x) =  \py{latex(-(x+k[0])**3-(x+k[0])**2)}$
\begin{itemize}
\item[1.] Calcolare gli integrali indefiniti $\displaystyle \int f(x) dx$ e $\displaystyle \int g(x) dx$.
\item[2.] Determinare l'area della parte di piano compresa tra le due funzioni nell'intervallo $[-2, 0]$.
\end{itemize}
\begin{answer}

{\color{blue} $\displaystyle \int f(x) dx = \cfrac{\py{latex((x+k[0])**3)}}{3} + C$, $\displaystyle \int g(x) dx = -\cfrac{\py{latex((x+k[0])**4)}}{4} -\cfrac{\py{latex((x+k[0])**3)}}{3} + C$. 
\hfill Risposta 1\kern19ex}

\smallskip
{\color{blue} Il valore dell'area è .
\hfill Risposta 2\kern19ex}

\end{answer}
\end{question}
%13
\begin{question}
\begin{pycode}
from sympy import *
k = [Rational( i ) for i in random.sample([2,3,4,5,6],1) ]
h = [Rational( i ) for i in random.sample([1,2,3,4,5,6],1) ]
a = [Rational( i ) for i in random.sample([1,2,3],1) ]
b = [Rational( i ) for i in random.sample([4,5,6,7],1) ]
x = symbols('x')
\end{pycode}
Si consideri la funzione $v(t) = \py{latex(k[0])} t^2 - t +\py{latex(h[0])}$ che descrive la velocit\`a di un corpo ad ogni istante di tempo $t$.
\begin{itemize}
\item[1.] Determinare lo spostamento netto di tale corpo nell'intervallo di tempo $[\py{latex(a[0])}, \py{latex(b[0])}]$.
\item[2.] Determinare lo spostamento netto di un corpo la cui velocit\`a \`e descritta dalla funzione $v(t/2)$.
\end{itemize}
\begin{answer}

{\color{blue} Lo spostamento netto è $\py{latex(k[0]*((b[0]**3-a[0]**3)/3)+(a[0]**2-b[0]**2)/2+h[0]*(b[0]-a[0]))}$. 
\hfill Risposta 1\kern19ex}

\smallskip
{\color{blue} Lo spostamento netto è $\py{latex(k[0]*((b[0]**3-a[0]**3)/12)+(a[0]**2-b[0]**2)/4+h[0]*(b[0]-a[0]))}$.
\hfill Risposta 2\kern19ex}

\end{answer}
\end{question}
%14
\begin{question}
\begin{pycode}
from sympy import *
k = [Rational( i ) for i in random.sample([2,4],1) ]
h = [Rational( i ) for i in random.sample([6,8,10],1) ]
a = [Rational( i ) for i in random.sample([1,2,3,4,5,6],1) ]
x = symbols('x')
\end{pycode}
Si consideri una funzione $f(x)$ tale che $\displaystyle \int_{\py{latex(k[0])}}^{\py{latex(h[0])}} f(2x) dx = \py{latex(a[0])}$
\begin{itemize}
\item[1.] Determinare l'area sottesa dalla funzione $f(x)$ nell'intervallo $[\py{latex(2*k[0])},\py{latex(2*h[0])}]$.
\item[2.] Determinare l'area sottesa dalla funzione $f(4x)$ nell'intervallo $[\py{latex(k[0]/2)},\py{latex(h[0]/2)}]$.
\end{itemize}
\begin{answer}

{\color{blue} L'area vale $\py{latex(2*a[0])}$. 
\hfill Risposta 1\kern19ex}

\smallskip
{\color{blue}L'area vale $\py{latex(4*a[0])}$.
\hfill Risposta 2\kern19ex}

\end{answer}
\end{question}


\end{document}