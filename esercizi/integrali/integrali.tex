\documentclass[11pt,twoside,a4paper]{article}
\usepackage[T1]{fontenc}
\usepackage[utf8]{inputenc}
\usepackage[top=20mm, head=6mm, headsep=6mm, foot=6mm, bottom= 15mm, left=15mm, right=15mm]{geometry}
\usepackage{amsmath}
\usepackage{calc} 
\usepackage{pythontex}
\usepackage{dsfont}
\newcommand{\mylabel}[1]{#1\hfill}
\renewenvironment{itemize}
  {\begin{list}{$\triangleright$}{%
   \setlength{\parskip}{0mm}
   \setlength{\topsep}{.4\baselineskip}
   \setlength{\rightmargin}{0mm}
   \setlength{\listparindent}{0mm}
   \setlength{\itemindent}{0mm}
   \setlength{\labelwidth}{2ex}
   \setlength{\itemsep}{.4\baselineskip}
   \setlength{\parsep}{0mm}
   \setlength{\partopsep}{0mm}
   \setlength{\labelsep}{1ex}
   \setlength{\leftmargin}{\labelwidth+\labelsep}
   \let\makelabel\mylabel}}{%
   \end{list}\vspace*{-1.3mm}}
\parindent0ex
\parskip2ex
\newcounter{quesito}
\newenvironment{question}{\bigskip\addtocounter{quesito}{1}\bigskip\bigskip\par\textbf{Quesito \thequesito.\kern1ex}}{\vspace{\parskip}}
\newenvironment{xquestion}{\bigskip\addtocounter{quesito}{1}\bigskip\bigskip\par\textbf{Quesito \thequesito.\kern1ex}}{\vspace{\parskip}}
\newenvironment{answer}{\par\textbf{Risposta\quad}}{\vspace{\parskip}}

\pagestyle{empty} 

\begin{document}
\begin{pycode}
import random
random.seed(2548445)
\end{pycode}
%1
\begin{question}
Si consideri la funzione $f(x) = \sqrt[4]{x^5} - \sin x$.
\begin{itemize}
\item[1.] Calcolare l'integrale indefinito $\displaystyle \int f(x) dx$.
\item[2.] Determinare l'area (con segno) sottesa alla funzione $f$ nell'intervall $[0,2]$.
\end{itemize}
\end{question}
%2
\begin{question}
Si consideri la funzione $f(x) = \cos(4x)$.
\begin{itemize}
\item[1.] Calcolare l'integrale indefinito $\displaystyle \int f(x) dx$.
\item[2.] Determinare l'area (con segno) sottesa alla funzione $f$ nell'intervall $[-2,3]$.
\end{itemize}
\end{question}
%3
\begin{question}
Si consideri la funzione $f(x) = e^{2x}$.
\begin{itemize}
\item[1.] Calcolare l'integrale indefinito $\displaystyle \int f(x) dx$.
\item[2.] Determinare l'area (con segno) sottesa alla funzione $f$ nell'intervall $[-1,2]$.
\end{itemize}
\end{question}
%4
\begin{question}
Si consideri la funzione $f(x) = 2x\log (x^2) + x^5$.
\begin{itemize}
\item[1.] Calcolare l'integrale indefinito $\displaystyle \int f(x) dx$.
\item[2.] Determinare l'area (con segno) sottesa alla funzione $f$ nell'intervall $[8,10]$.
\end{itemize}
\end{question}
%5
\begin{question}
Si consideri la funzione $f(x) = 2x^2$ nell'intervallo $[0,4]$.
\begin{itemize}
\item[1.] Suddividere tale intervallo in $8$ parti e scrivere gli intervalli in cui \`e stato diviso. Calcolare la funzione $f$ nel punto medio di ciascuno di tali intervalli.
\item[2.] Calcolare la somma di Riemann della funzione $f$ relativa alla suddivisione e ai punti di campionamento trovati al punto precedente.
\end{itemize}
\end{question}
%6
\begin{question}
Si consideri la funzione $f(x) = x^2 - 8x$.
\begin{itemize}
\item[1.] Determinare l'area (con segno) sottesa da tale funzione nell'intervallo $[0, 10]$.
\item[2.] Determinare l'area (con segno) sottesa dalla funzione $\lvert f(x) \rvert$ nell'intervallo $[0, 10]$.
\end{itemize}
\end{question}
%7
\begin{question}
Si consideri la funzione definita a tratti 
\[f(x) = \begin{cases} 4 \quad 1 \leq x < 3 \cr -6 \quad 3 \leq x \leq 7 \end{cases}\]
\begin{itemize}
\item[1.] Determinare l'area (con segno) sottesa da tale funzione.
\item[2.] Determinare l'area (con segno) sottesa dalla funzione $f(2x + 3)$.
\end{itemize}
\end{question}
%8
\begin{question}
Si consideri la funzione $f(x) = e^x - 2$
\begin{itemize}
\item[1.] Calcolare l'integrale indefinito $\displaystyle \int f(x) dx$.
\item[2.] Determinare l'area della parte di piano compresa tra la funzione $f$ e le due rette di equazioni $x = 0$ e $x = 2$.
\end{itemize}
\end{question}
%9
\begin{question}
Si consideri la funzione $f(x) = x^3 + 1$
\begin{itemize}
\item[1.] Calcolare l'integrale indefinito $\displaystyle \int f(x) dx$.
\item[2.] Determinare l'area della parte di piano compresa tra la funzione $f$ e le due rette di equazioni $x = -1$ e $x = 3$.
\end{itemize}
\end{question}
%10
\begin{question}
Si consideri la funzione $f(x) = 3\sin(x)$
\begin{itemize}
\item[1.] Calcolare l'integrale indefinito $\displaystyle \int f(x) dx$.
\item[2.] Determinare l'area della parte di piano compresa tra la funzione $f$ e le due rette di equazioni $x = \pi$ e $x = 2 \pi$.
\end{itemize}
\end{question}
%11
\begin{question}
Si consideri la funzione $f(x) = 3\sin(x)$
\begin{itemize}
\item[1.] Calcolare l'integrale indefinito $\displaystyle \int f(x) dx$.
\item[2.] Determinare l'area della parte di piano compresa tra la funzione $f$ e le due rette di equazioni $x = \pi$ e $x = 2 \pi$.
\end{itemize}
\end{question}
%12
\begin{question}
Si considerino le funzioni $f(x) = x^2$ e $g(x) = -x^3 - x^2$
\begin{itemize}
\item[1.] Calcolare gli integrali indefiniti $\displaystyle \int f(x) dx$ e $\displaystyle \int g(x) dx$.
\item[2.] Determinare l'area della parte di piano compresa tra le due funzioni nell'intervallo $[-2, 0]$.
\end{itemize}
\end{question}
%13
\begin{question}
Si consideri la funzione $v(t) = 3 t^2 - t +1$ che descrive la velocit\`a di un corpo ad ogni istante di tempo $t$.
\begin{itemize}
\item[1.] Determinare lo spostamento netto di tale corpo nell'intervallo di tempo $[1, 6]$.
\item[2.] Determinare lo spostamento neto di un corpo la cui velocit\`a \`e descritta dalla funzione $v(t/2)$.
\end{itemize}
\end{question}
%14
\begin{question}
Si consideri una funzione $f(x)$ tale che $\displaystyle \int_1^4 f(2x) dx = 6$
\begin{itemize}
\item[1.] Determinare l'area sottesa dalla funzione $f(x)$ nell'intervallo $[2,8]$.
\item[2.] Determinare l'area sottesa dalla funzione $f(3x)$ nell'intervallo $[6,24]$.
\end{itemize}
\end{question}


\end{document}