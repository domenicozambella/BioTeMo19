\documentclass[11pt,twoside,a4paper]{article}
\usepackage[T1]{fontenc}
\usepackage[utf8]{inputenc}
\usepackage[top=20mm, head=6mm, headsep=6mm, foot=6mm, bottom= 15mm, left=15mm, right=15mm]{geometry}
\usepackage{amsmath}
\usepackage{calc} 
\usepackage{pythontex}
\newcommand{\mylabel}[1]{#1\hfill}
\renewenvironment{itemize}
  {\begin{list}{$\triangleright$}{%
   \setlength{\parskip}{0mm}
   \setlength{\topsep}{.4\baselineskip}
   \setlength{\rightmargin}{0mm}
   \setlength{\listparindent}{0mm}
   \setlength{\itemindent}{0mm}
   \setlength{\labelwidth}{2ex}
   \setlength{\itemsep}{.4\baselineskip}
   \setlength{\parsep}{0mm}
   \setlength{\partopsep}{0mm}
   \setlength{\labelsep}{1ex}
   \setlength{\leftmargin}{\labelwidth+\labelsep}
   \let\makelabel\mylabel}}{%
   \end{list}\vspace*{-1.3mm}}
\parindent0ex
\parskip2ex
\newcounter{quesito}
\newenvironment{question}{\bigskip\addtocounter{quesito}{1}\bigskip\bigskip\par\textbf{Quesito \thequesito.\kern1ex}}{\vspace{\parskip}}
\newenvironment{xquestion}{\bigskip\addtocounter{quesito}{1}\bigskip\bigskip\par\textbf{Quesito \thequesito.\kern1ex}}{\vspace{\parskip}}
\newenvironment{answer}{\par\textbf{Risposta\quad}}{\vspace{\parskip}}

\pagestyle{empty} 

\begin{document}

\begin{question}
Si consideri una funzione $f(x)$ la cui derivata prima è data dalla funzione $f'(x) = e^{-x}$.
\begin{itemize}
\item[1.] Indicare gli intervalli in cui la funzione $f(x)$ cresce e quelli in cui la funzione decresce.
\item[2.] Trovare massimi e minimi locali di $f(x)$.
\end{itemize}
\end{question}

\begin{question}
Si consideri una funzione $f(x)$ la cui derivata prima è data dalla funzione $f'(x) = \cfrac{1}{x}$.
\begin{itemize}
\item[1.] Indicare gli intervalli in cui la funzione $f(x)$ cresce e quelli in cui la funzione decresce.
\item[2.] Trovare massimi e minimi locali di $f(x)$.
\end{itemize}
\end{question}

\begin{question}
Si consideri una funzione $f(x)$ la cui derivata prima è data dalla funzione $f'(x) = \cfrac{x^2 - 1}{x}$.
\begin{itemize}
\item[1.] Indicare gli intervalli in cui la funzione $f(x)$ cresce e quelli in cui la funzione decresce.
\item[2.] Trovare massimi e minimi locali di $f(x)$.
\end{itemize}
\end{question}

\begin{question}
Si consideri una funzione $f(x)$ la cui derivata prima è data dalla funzione $f'(x) = \cfrac{x^2 + 1}{x}$.
\begin{itemize}
\item[1.] Indicare gli intervalli in cui la funzione $f(x)$ cresce e quelli in cui la funzione decresce.
\item[2.] Trovare massimi e minimi locali di $f(x)$.
\end{itemize}
\end{question}

\begin{question}
Si consideri una funzione $f(x)$ la cui derivata prima è data dalla funzione $f'(x) = \log (x)$.
\begin{itemize}
\item[1.] Indicare gli intervalli in cui la funzione $f(x)$ cresce e quelli in cui la funzione decresce.
\item[2.] Trovare massimi e minimi locali di $f(x)$.
\end{itemize}
\end{question}

\begin{question}
Si consideri un corpo lasciato cadere da una torre alta 500 metri. Sia  $f(t) = 4.9 t^2$ la funzione che ne descrive l'altezza da terra ad ogni secondo.
\begin{itemize}
\item[1.] Qual è la velocità istantanea del corpo dopo 3 secondi?
\item[2.] Qual è la velocità istantanea del corpo quando tocca terra?
\end{itemize}
\end{question}

\begin{question}
Sia data la funzione $f(x) = x^2 + 3 x$
\begin{itemize}
\item[1.] Scrivere l'equazione della retta tangente nel punto (-5, 10).
\item[2.] In quali intervalli la funzione è decrescente?
\end{itemize}
\end{question}

\begin{question}
Sia data la funzione $f(x) = x^3 + 1$
\begin{itemize}
\item[1.] Scrivere l'equazione della retta tangente nel punto (1, 2).
\item[2.] In quali intervalli la funzione è crescente?
\end{itemize}
\end{question}

\begin{question}
Si consideri una particella che si muove lungo una retta. Sia $f(t) = 4 t^3 + 5 t$ la funzione che ne descrive la distanza in metri dal punto di partenza ogni secondo.
\begin{itemize}
\item[1.] Qual è la velocità istantanea del corpo dopo 2 secondi?
\item[2.] Quando la velocità del corpo è superiore a 101 metri al secondo?
\end{itemize}
\end{question}

\begin{question}
Si consideri la funzione $f(x) = x^3 + 2x + 1$.
\begin{itemize}
\item[1.] Determinare la derivata prima $f'(x)$.
\item[2.] Trovare massimi e minimi locali di $f(x)$.
\end{itemize}
\end{question}

\begin{question}
Si consideri la funzione $f(x) = x \cos (x)$.
\begin{itemize}
\item[1.] Determinare la derivata $f'(x)$.
\item[2.] Determinare la derivata di $f(2x)$.
\end{itemize}
\end{question}

\begin{question}
Si consideri la funzione $f(x) = \sin^2 x$.
\begin{itemize}
\item[1.] Determinare la derivata $f'(x)$.
\item[2.] Determinare la derivata di $f(2x)$.
\end{itemize}
\end{question}

\begin{question}
Si consideri la funzione $f(x) = \sqrt{x}$.
\begin{itemize}
\item[1.] Scrivere l'approssimazione lineare di $f(x)$ in 4.
\item[2.] Usare il risultato precedente per approssimare i valori di $\sqrt{4.1}$ e $\sqrt{3.9}$.
\end{itemize}
\end{question}

\begin{question}
Si consideri la funzione $f(x) = 3^x$.
\begin{itemize}
\item[1.] Scrivere l'approssimazione lineare di $f(x)$ in 1.
\item[2.] Usare il risultato precedente per approssimare i valori di $3^{0.9}$ e $3^{1.1}$.
\end{itemize}
\end{question}

\begin{question}
Si consideri la funzione $f(x) = \sqrt[3]{x}$.
\begin{itemize}
\item[1.] Scrivere l'approssimazione lineare di $f(x)$ in 1.
\item[2.] Usare il risultato precedente per approssimare i valori di $\sqrt[3]{1.1}$ e $\sqrt[3]{1.2}$.
\end{itemize}
\end{question}

\end{document}